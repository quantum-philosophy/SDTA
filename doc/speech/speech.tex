\documentclass[11pt,a4paper]{article}
\usepackage[english]{babel}
\usepackage[utf8]{inputenc}
\usepackage[T1]{fontenc}
\usepackage{amsmath}
\usepackage{amssymb}
\usepackage{amsthm}

\renewcommand{\thesection}{}

\title{Steady-State Dynamic Temperature Analysis and Reliability Optimization for Embedded Multiprocessor Systems: Presentation}
\author{Ivan Ukhov}

\begin{document}
\maketitle

\section{Title}
Good afternoon everybody. My name is Ivan Ukhov, and today I am grad to present you our work called Steady-State Dynamic Temperature Analysis and Reliability Optimization for Embedded Multiprocessor Systems.

\section{Slide 1}
First, let me introduce you three types of temperature analysis that I will be referring to through out the presentation.

The first type, and probably the simplest one, is known as the Steady-State Temperature Analysis. In this case, it is assumed that the power dissipation of the system is constant, for instance, it can be the average power. And one is looking for the corresponding temperature when the system reaches a thermal balance and temperature stops changing.

The second type is the Transient Temperature Analysis, TTA. Here we already have a power profile instead of just a single average value, and we are interested in the transient temperature, that is the evolution of temperature in time. As a result, we obtain a transient temperature profile.

The third type of temperature analysis is known as the Steady-State Dynamic Temperature Analysis, SSDTA, and today I am going to talk mostly about it. In this case, it is assumed that the power profile is periodic and, after a sufficiently long period of time, temperature also starts following some periodic pattern, called the Steady-State Dynamic Temperature Profile, SSDTP, which we would like to find. The result of this analysis is again a temperature profile, but this time periodic.

\section{Slide 2}
In brief, our work is about the following. We have a multiprocessor platform with a thermal package and a discrete dynamic power profile. And we are interested in the thermal behaviour of the system in the long run where temperatures exhibits a periodic pattern, the SSDTP, as I have just described.

In the second part of the work, we demonstrate the importance of the SSDTA to perform a reliability optimization based on the thermal cycling fatigue, which I shall explain later.

\section{Slide 3}
Now, how can one perform the SSDTA using the existing tools? So, the first option is to construct a very long power profile by repeating the same periodic power chunk a number of times. Then, the Transient Temperature Analysis of this long trace is performed, and it is assumed that, if the number of chunks is sufficiently large, temperature by the last chunk reaches the desired SSDTP. Unfortunately, depending on the application, this approach can burn a lot of time without any guarantees of accuracy.

\section{Slide 4}
Another option is known as the Steady-State Approximation (SSA). In this case, one employs the Steady-State Temperature Analysis, not Steady-State Dynamic, in order to approximate the SSDTP. The process goes like this. In order to calculate temperature at a certain moment of time, it is assumed that the power consumption at that moment of time stays forever constant and, when the system reaches a thermal balance, we obtain the desired temperature for this moment of time. The final temperature curve becomes stepwise, as it is illustrated in the slide, the orange curve. The approximation is fast, but, unfortunately, can be dramatically inaccurate since in general temperature does not have time to reach the assumed thermal balance.

\section{Slide 5}
Now, how do we propose to perform the SSDTA? Well, the beginning is standard. Given a multiprocessor platform with a thermal package, an equivalent thermal RC circuit is constructed. The circuit is composed of a number of thermal nodes that are characterized by thermal capacitance and thermal resistance. The nodes that belong to the die itself are said to be sources of heat in the sense that they dissipate power.

\section{Slide 6}
The thermal behaviour of the circuit is described with the well-known expression given at the top of the slide. $T$ is the temperature vector, $P$ is the power vector, $C$ and $G$ are the matrices of the capacitance and conductance, respectively. In general, it does not have a closed-form solution, but when the power on the right-hand side is constant, it becomes a system of ordinary differential equations, which can be solved analytically. Now, since the given power profile is discrete, it is reasonable to assume that between neighbour samples the power dissipation stays constant. In this case, we can solve this system of ordinary differential equations for each of the intervals, forming an iterative process shown in the slide. $K$ and $B$ here are the coefficients that connects the current temperature and power with the temperature at the next moment of time. This iterative solution is already employed in the literature to perform the TTA; we propose to employ it to perform the SSDTA as well. Since we are looking for a periodic curve, we have one additional equation, shown in the slide, that ensures that the values on both sides of the temperature profile are equal.

\section{Slide 7}
When we write this iterative equation for each step of the power profile and add the boundary condition mentioned above, we obtain a huge system of linear equation, which is to be solved. The dimensions of the system are the number of steps, $N_s$, in the power profile multiplied by the number nodes, $N_n$, in the thermal circuit. The problem here is that the system is so large that straight-forward solutions just do not work, as we discuss in the paper.

\section{Slide 8}
Fortunately, we have a fast and accurate solution to this problem. Although, I am not going into details right now, we can take them to the poster session or you can find everything in the paper. What we propose consists of two parts. First, we perform a one-time auxiliary transformation of the heat equation, which I shown you before. This transformation allows us to undertake all the computations in a more efficient manner. Secondly, we propose an efficient way of actually finding the solution, the SSDTP, by forming a so-called condensed system. The approach takes into consideration all the features of the problem, which are the structure and sparseness of the system. The solution is analytical and, therefore, accurate withing the thermal model. Also, it is relatively cheap to obtain. As an example, the computational complexity of direct solvers of linear systems is $N_s^3 N_n^3$. Our solution has just a linear dependency on $N_s$, the number of steps in the power profile, which usually by far dominates.

Also, it should be mentioned that the leakage modeling is included in the proposed solution as well. Please refer to the paper for further details.

\section{Slide 9}
Now, a few words about the experimental results. Again, in the paper one can find much more. Here is the comparison with HotSpot, a well-known thermal simulator, where we let it to successively perform the TTA until its normalized root mean square error becomes less or equal to 1\%. In the left graph we vary the number of steps in the power profile, in the right --- the number of processing elements that, according to some equation, corresponds to the number of thermal nodes. As you can see, we are 2000--5000 times faster.

\section{Blank}
The second part of the paper is dedicated to the reliability optimization where we demonstrate our technique in action. One of the failure mechanisms that cannot be properly addressed without the knowledge of the exact SSDTP is the thermal cycling fatigue.

\section{Slide 10}
Here I have a motivational example where you can see what I am talking about. The thermal cycling refers to the damage caused by the fluctuation of temperature, that is the amplitudes and the number of temperature changes in time. Here we have three different combinations of a mapping and schedule, on the left side, of the same application that consists of six tasks deployed onto two cores. The corresponding temperature fluctuation is shown on the right side; please pay your attention to the change of the blue curve. It can be seen that altering the mapping and schedule can significantly decrease the temperature variation and, therefore, prolong the lifetime of the system.

\section{Slide 11}
So, where do we plug in our SSDTA? Well, we employ a genetic algorithm that varies the mapping and schedule of the application in order to minimize the damage from the thermal cycling, and in order to compute the damage we employ out technique.

It is worth being mentioned once again that just the average temperature is by no means sufficient in this case to estimate the damage and guide the optimization process; the SSDTP is a must.

Also, we demonstrate in the paper that the energy efficiency of the system is not compromised by our optimization. For this particular purpose we conducted a multiobjective optimization with energy as the second goal.

\section{Slide 12}
And here is the last slide where I would like to show you some experimental results. Synthetic test cases can be found in the paper; here we have a real-life example, which is the MPEG2 decoder. The decoder was analysed using the MPARM platform and decomposed into 34 tasks deployed onto 2 cores. With our method the genetic algorithm managed to find a solution that increases the lifetime of the system by 24 times while when we replace our technique with the state of the art, it managed to find solutions that are only 5--11 times better than the initial one.

\section{Thanks}
I guess this is it. Thank you for your attention. If you have any questions, please feel free to ask.

\end{document}
