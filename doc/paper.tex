\documentclass[conference]{IEEEtran}
\usepackage[english]{babel}
\usepackage[utf8]{inputenc}
\usepackage[T1]{fontenc}
\usepackage{mathtools}
\usepackage{graphicx}
\usepackage{appendix}
\usepackage{hyperref}
\usepackage{color}

\newcommand{\equref}[1]{Equation~(\ref{#1})}
\newcommand{\figref}[1]{Figure~\ref{#1}}
\newcommand{\apref}[1]{Appendix~\ref{#1}}
\newcommand{\secref}[1]{Section~\ref{#1}}
\newcommand{\dindex}[3]{#1_{#2,\;#3}}

\newcommand{\note}[1]{\textcolor{red}{NOTICE: #1}}
\newcommand{\todo}[1]{\textcolor{blue}{TODO: #1}}


\newcommand{\ssdtc}{Steady-State Dynamic Temperature Curve}

\title{Accurate and Fast Steady-State\\Dynamic Temperature Analysis}
\author{Min Bao, Petru Eles, Zebo Peng, and Ivan Ukhov}

\begin{document}
  \maketitle

  \begin{abstract}
    The temperature analysis is one of the most important components of a well-established design of an embedded system. Whenever it is possible, system architects are trying to take it into account in order to produce results of a higher quality in the sense of, for example, reliability and energy consumption. An accurate and fast temperature estimation is not an easy goal to achieve, since the temperature variation is a considerably complicated process that depends on a lot of different parameters, such as the dynamic power consumption and power leakage of the die. The problem becomes much more severe when we make a little step further and put it inside an optimization loop where this type of estimation needs to be performed thousands of times. In this case, the solution should be not only accurate, but also computationally cheap to find. In this paper we propose such a technique that satisfies both criteria, accuracy and speed. Our particular focus of interest is \emph{multiprocessor} system-on-chips that have \emph{periodic} power and, consequently, temperature profiles (for example, the system may execute a periodic application). The proposed solution is analytical, therefore, it is exact from the perspective of the underlying model, also it is fast enough to be used in such search heuristics as the genetic algorithms, that we have chosen to demonstrate our approach.

  \end{abstract}

  \section{Introduction}
  \subsection{Temperature Variation}
Morden policies for prevening temperature runaways and decreasing energy consumption, within such tecniques as dynamic power management (DPM) and dynamic voltage and frequency scaling (DVFS), keep puzzling embedded system architects with a constantly increasing strength. These and similar approaches may cause considerable temperature fluctuations within a multiprocessor system-on-chip (MPSoC), therefore, dramatically decreasing its reliability \cite{mihic2004}, \cite{simunic2005}.

The importance of the temperature distribution over ICs has been widely studied in the literature \cite{lu2004}. Hence, a large number of different methods for performance and energy optimization imposes the maximal temperature constrain. In this essence, the need of fast and accurate methods for obtaining temperature profiles becomes urgent.

In this paper we consider the HotSpot thermal model \cite{huang2006} and propose an extremely fast way to calculate the steady-state dynamic temperature curve (SSDTC) of an embedded system that executes a set of periodic tasks.

In order to demonstrate our approach, we perform the energy optimization with a constrain on the spatial temperature gradient within a die. The constrain is satisfied with the help of SSDTC that delivers the diapason of the temperature fluctuation. The optimization problem is solved through the mapping and scheduling based on genetic algorithms and the list scheduler described in \cite{schmitz2004}.

\subsection{A Motivational Example}
TODO.


  \section{Preliminaries}
  \label{sec:architecture-model}
We consider a heterogeneous multicore architecture with a set of processing elements $\Pi$ defined as the following:
\[
  \Pi = \{ \pi_i = (V_i, \: f_i, \: N_{gate \: i}): \; i = \range{0}{N_p - 1} \}
\]
where $V_i$, $f_i$, and $N_{gate \: i}$ are the supply voltage, frequency, and number of gates \cite{liao2005} of the $i$th core, respectively.

\label{sec:power-model}
The total power dissipation of a processing element is defined as the sum of the dynamic and leakage power: $P = P_{dyn} + P_{leak}$. The dynamic part is modeled as $P_{dyn} = C_{eff} \cdot f \cdot V^2$ where $C_{eff}$ is the effective switched capacitance, $V$ and $f$ are the supply voltage and frequency, respectively. The leakage part of the power dissipation is defined as \cite{liao2005}:
\begin{equation} \label{eq:total-power}
  P_{leak}(T) = N_{gate} \: V \: I_0 \left[ A \: T^2 e^{\frac{\alpha \: V + \beta}{T}} + B e^{(\gamma \: V + \delta)} \right]
\end{equation}
where $T$ and $V$ are the current temperature and supply voltage, respectively, $N_{gate}$ is the number of gates in the circuit, $I_0$ is the average leakage current at the reference temperature and supply voltage. $A$, $B$, $\alpha$, $\beta$, $\gamma$, and $\delta$ are the technology dependent constants found in \cite{liao2005}.

\label{sec:thermal-model}
Our proposed technique is based on the RC thermal model that employs the analogy between electrical and thermal circuits \cite{kreith2000}. Heat transfer is modeled with the following system of differential equations:
\begin{equation} \label{eq:fourier-model}
  \m{C} \: \frac{d\v{T}(t)}{dt} + \m{G} \: (\v{T}(t) - \v{T}_{amb})= \v{P}(t)
\end{equation}
where $\v{T}$ is the temperature vector, $\v{T}_{amb}$ is the ambient temperature vector, $\m{C}$ is the thermal capacitance matrix, $\m{G}$ is the thermal conductance matrix, and $\v{P}$ is the power dissipation vector. The dimensions of the system are $N_n \times N_n$, where $N_n$ is the number of nodes in the equivalent RC thermal circuit, which is further discussed in \appref{ap:thermal-circuits}.


  \section{Prior Work and Motivation}
  \iimage{task-graph}{0 0 0 0}{Motivational example with 6 tasks, labeled from ``T0'' to ``T5'', running on a heterogeneous dual-core architecture. The execution times of the tasks vary between cores and are given on the figure along with the period of the application.}
\image{motivation}{100 240 100 240}{Three alternative combinations of mapping and scheduling (the left side) of the application shown in \figref{fig:task-graph} onto a dual-core architecture and corresponding steady-state dynamic temperature curves (the right side) for each of the cores.}
Consider a toy application with six tasks, denoted ``T0''--``T5'', and a heterogeneous architecture with two cores, labeled ``PE0'' and ``PE1''. The task graph of the application is given in \figref{fig:task-graph}. The initial mapping, schedule, and resulting SSDTP are shown at the top of \figref{fig:motivation}, where PE0 is experiencing three thermal cycles. If we change the allocation of T5 and move it to PE1, we achieve two thermal cycles of PE0 instead of three. Finally, if we vary the schedule as well and change the order of T1 and T3, the number of cycles of PE0 becomes one. Using the reliability model from \secref{sec:reliability}, we observe improvements in the MTTF of 45\% and 54\%, respectively, relative to the initial configuration.


  \section{Steady-State Dynamic Temperature Profile}
  \subsection{Problem Formulation}
We consider a multiprocessor system-on-chip with a set of processing elements $\Pi = \{ \pi_i: i = 1 \dots N_p \}$. The system executes a periodic application with the overall period $\mathcal{T}$ which is discretized into $N_s$ intervals $\triangle t_j$ for $j = 1 \dots N_s$. The time intervals $\triangle t_j$ are small enough to assume that the power consumption and the temperature of each processing element $\Pi_i$ are constant within these intervals. We assume that the power profile $P = \{ P_{ij}: i = 1 \dots N_p, \; j = 1 \dots N_s \}$, where $P_{ij}$ is the power consumption of the $i$th core during the $j$th time interval, is also periodic. After the stabilization process, the temperature profile of the system becomes periodic as well and defined as $T = \{ T_{ij}: i = 1 \dots N_p, \; j = 1 \dots N_s \}$, ${T_{ij}}$ corresponds to the temperature of the ${i}$th core on the $j$th time interval. Such periodic temperature profile is called the steady-state dynamic temperature curve (SSDTC).

Now we can formulate the problem. Given:
\begin{itemize}
  \item A \emph{periodic} power profile $P$ of a multiprocessor system-on-chip with a set of processing elements $\Pi$.
  \item The floorplan of the chip, i.e., the location and size of each processing element $\pi_i \in \Pi$.
  \item The configuration of the package including the thermal interface material, heat spreader, and heat sink.
  \item The thermal parameters of the die and package (the thermal conductivity, thermal capacitance, etc.).
\end{itemize}
Find:
\begin{itemize}
  \item The corresponding \emph{periodic} temperature profile $T$ of the system in its steady state (when the temperature stabilization process is finished).
\end{itemize}

Furthermore, we want to be able to perform this procedure as quick as possible with a desired accuracy in order to use it for solving different optimization problems where such temperature curves have to be calculated a very large number of times. An example of such problem is the reliability-aware task allocation and scheduling given in \secref{sec:results}.

\subsection{Iterative Solution with HotSpot}
One approach to find the SSDTC is to call the HotSpot simulator with a very long power profile composed of one repeating chunk. In this case, we get a huge temperature profile, and we can take only its tail that corresponds to the last chunk in the power profile. Another way to obtain the SSDTC is to call the simulator a number of times each time starting from the temperature that we got during the previous call.

\image{hotspot-one-realization}{50 200 50 200}{An example of one HotSpot simulation for one processing element running an application of $1 s$. The ground truth is the real steady-state dynamic temperatures curve that HotSpot is supposed to reach after a number of iterations. The power profile contains $10^4$ steps that corresponds to $10^{-4} s$ sampling interval.}

The number of iterations required to get the steady-state temperature profile depends on the accuracy that we are trying to achieve. It also depends a lot on the application itself and, consequently, on its power profile. This is an especially urgent issue for tasks that have a short execution time relative to the thermal time constant of the die. In this case temperature does not have time to reach its steady-state value during executing of a task, but it is still gradually increasing on average. This leads to a large number iterations that the simulator is required to perform. An example is given on \figref{fig:hotspot-one-realization}, where the blue curve represents the ground truth (SSDTC), and the orange one shows one first simulation of HotSpot. The sampling interval was chosen to be $10^{-4} \; s$ to capture all power fluctuations due to the task switching activity, resulting in a power profile with $10^4$ steps. The situation is getting \emph{much worse} if we want to take into consideration the leakage power, since in this case we have a loop between temperature and power, and we need to perform extra iterations (indifferent of the chosen solution approach) for temperature and power to converge \cite{liu2007}.

\subsection{Direct Analytical Solution}
In order to deal with the problem, we use an analytical approach. The direct solution of \equref{eq:initial} is given as the following:
\begin{equation} \label{eq:solution}
  T(t) = e^{C^{-1}A t} \; T_0 + (C^{-1} A)^{-1}(e^{C^{-1}A t} - I)C^{-1} B
\end{equation}

Since we want to be as efficient as possible, we shall go into details of the solution. First, we notice that here we need to calculate the matrix exponential of the matrix $C^{-1} A t$, which would be much easier to accomplish if the matrix were symmetric, because a real symmetric matrix is \emph{diagonalizable} and has \emph{independent} (orthogonal) real eigenvectors:
\begin{equation} \label{eq:eigenvalue-decomposition}
  M = U \Lambda U^T
\end{equation}
where $M$ is a real symmetric matrix, $U$ is the eigenvectors of $M$, $\Lambda$ is a diagonal matrix of the eigenvalues of $M$ ($\lambda_i$). Therefore, the matrix exponential can be computed using the obtained eigenvalue decomposition:
\begin{align}
  & e^M = e^{U \Lambda U^T} = U \: e^{\Lambda} \: U^T \nonumber \\
  & e^{\Lambda} = \left[
      \begin{array}{ccc}
        e^{\lambda_1} & \cdots & 0 \\
        \vdots & \ddots & \vdots \\
        0 & \cdots & e^{\lambda_{n}}
      \end{array}
    \right] \nonumber
\end{align}

Hence, instead of $C^{-1} A t$ in front of the temperature vector we want to have a symmetry matrix. In order to achieve this, we perform the following substitution:
\begin{align*}
  Y & = C^{\frac{1}{2}} T \\
  D & = C^{-\frac{1}{2}} A \: C^{-\frac{1}{2}} \\
  E & = C^{-\frac{1}{2}} B
\end{align*}
with the result:
\begin{align}
  \frac{dY}{dt} & = D \: Y + E \nonumber \\
  Y(t) & = e^{D t} Y_0 + D^{-1} (e^{D t} - I) E \label{eq:modified-solution} \\
  T(t) & = C^{-\frac{1}{2}} Y(t) \nonumber
\end{align}

In this case, $D$ is a symmetric matrix, therefore, it will be easier to find the matrix exponential of $D \: t$ using the above-mentioned eigenvalue decomposition (\equref{eq:eigenvalue-decomposition}):
\[
  e^{D t} = U \: e^{\Lambda t} \: U^T = U \left[
      \begin{array}{ccc}
        e^{t \lambda_1} & \cdots & 0 \\
        \vdots & \ddots & \vdots \\
        0 & \cdots & e^{t \lambda_{N_n}}
      \end{array}
    \right] U^T
\]

Now we are going further and look at the power profile $B$. Each row of $B$ corresponds to a particular time interval $\triangle t_i$ and represents the power consumption $B_i$ during this interval of all processing elements. Each step $i = 1 \dots N_s$ of the iterative process we have a pair $(\triangle t_i, B_i)$ which gives us a temperature vector $T_i$ according to \equref{eq:modified-solution} where $t = \triangle t_i$. The iterative process can be described as the following:
\begin{align}
  & Y_{i+1} = K_i \: Y_i + G_i \: B_i \label{eq:recurrent-equation} \\
  & K_i = e^{D \: \triangle t_i} \nonumber \\
  & G_i = D^{-1} \left( e^{D \triangle t_i} - I \right) C^{-\frac{1}{2}} \nonumber
\end{align}

Since we perform the eigenvalue decomposition of D (\equref{eq:eigenvalue-decomposition}), $D^{-1}$ can be efficiently computed in the following way:
\[
  D^{-1} = U \: \Lambda^{-1} \: U^T = U \left[
      \begin{array}{ccc}
        \frac{1}{\lambda_1} & \cdots & 0 \\
        \vdots & \ddots & \vdots \\
        0 & \cdots & \frac{1}{\lambda_{N_n}}
      \end{array}
    \right] U^T \\
\]
therefore:
\begin{align*}
  G_i & = U \: \Lambda^{-1} \: U^T \left(U \: e^{\Lambda \triangle t_i} \: U^T - U \: U^T \right) C^{-\frac{1}{2}} = \\
      & = U \left[
        \begin{array}{ccc}
          \frac{e^{\triangle t_i \: \lambda_1} - 1}{\lambda_1} & \cdots & 0 \\
          \vdots & \ddots & \vdots \\
          0 & \cdots & \frac{e^{\triangle t_i \: \lambda_{N_n}} - 1}{\lambda_{N_n}}
        \end{array}
      \right] U^T \: C^{-\frac{1}{2}}
\end{align*}

Therefore, in order to find SSDTC, we need to solve the following system of linear equations:
\[
  \begin{cases}
    K_1 \: Y_1 - Y_2 & = -Q_1 \\
    ... \\
    K_{N_s} \: Y_{N_s} - Y_{N_s + 1} & = -Q_{N_s}
  \end{cases}
\]
where $Q_i = G_i \: B_i$. Also we should take into account the boundary condition which ensures that the temperature has the same values on both sides of the curve:
\begin{equation} \label{eq:boundary-condition}
  Y_1 = Y_{N_s + 1}
\end{equation}

Hence, the system of linear equations takes the following form:
\[
  \begin{cases}
    K_1 \: Y_1 - Y_2 & = -Q_1 \\
    ... \\
    -Y_1 + K_{N_s} \: Y_{N_s} & = -Q_{N_s}
  \end{cases}
\]

To get the whole picture, the system can be written as:
\begin{align}
  & \mathbb{A} \: \mathbb{Y} = \mathbb{B} \label{eq:system} \\
  & \mathbb{A} = \left[
    \begin{array}{ccccc}
      K_1 & -I & 0 & \cdots & 0 \\
      0 & K_2 & -I &  & \vdots \\
      \vdots &  & \ddots & -I & 0 \\
      0 &  &  & K_{N_s - 1} & -I \\
      -I & 0 & \cdots & 0 & K_{N_s}
    \end{array}
  \right] \nonumber \\
  & \mathbb{Y} = \left[
    \begin{array}{c}
      Y_1 \\
      \vdots \\
      Y_{N_s}
    \end{array}
  \right] \nonumber \\
  & \mathbb{B} = \left[
    \begin{array}{c}
      -Q_1 \\
      \vdots \\
      -Q_{N_s}
    \end{array}
  \right] \nonumber
\end{align}

$\mathbb{A}$ is a square matrix of the dimensions $N_n N_s \times N_n N_s$. $\mathbb{Y}$ and $\mathbb{B}$ are vectors of the length $N_n N_s$. This is the system that can give us the desired SSDTP.

Such systems could be extremely big, especially when we want to achieve a high level of accuracy and, therefore, the power profile contains a lot of steps $N_s$. Each new step is $N_n$ new equations in the system given by \equref{eq:system}. Also the complexity grows very rapidly with the number of processing elements $N_p$, since in the HotSpot thermal model the number of thermal nodes $N_n$ is dependent on it according to the equation \cite{rao2008}:
\[
  N_n = 4 N_p + 12
\]

Therefore, \emph{each} new processing element increases \emph{each} matrix $K_i$ by 4 rows and 4 columns, and \emph{each} vector $Y_i$ and $Q_i$ by 4 elements. All in all, a fast and accurate approach to solve \equref{eq:system} is required.

\subsection{Condensed Equation}
The system that we have is described with the following recurrent equation:
\begin{equation} \label{eq:ce-recurrent}
  Y_{i + 1} = K_i \: Y_i + Q_i, \; i = 1 \dots N_s
\end{equation}

The iterative repetition of this equation leads us to:
\begin{align}
  Y_i & = \prod_{j = 1}^{i} K_j \: Y_1 + P_{i - 1}, \; i = 2 \dots N_s + 1 \label{eq:y-recurrent} \\
  P_1 & = Q_1 \nonumber \\
  P_i & = \sum_{l = 2}^i \prod_{j = l}^i K_j \: Q_{l - 1} + Q_i, \: i = 2 \dots N_s \nonumber
\end{align}

The recurrent version of the last equation:
\begin{equation} \label{eq:p-recurrent}
  P_i = K_i \: P_{i - 1} + Q_i, \; i = 2 \dots N_s
\end{equation}

Therefore, we can calculate the final value $Y_{N_s + 1}$ from \equref{eq:y-recurrent}:
\[
  Y_{N_s + 1} = \prod_{j = 1}^{N_s} K_j \: Y_1 + P_{N_s}
\]

Taking into account the boundary condition given by \equref{eq:boundary-condition}, we obtain the following system of linear equations:
\[
  (I - \prod_{j = 1}^{N_s} K_j) \: Y_1 = P_{N_s}
\]

Solving this system, we obtain the first component of the vector $\mathbb{Y}$, that is $Y_1$. ($P_{N_s}$ can be calculated using \equref{eq:p-recurrent}.) All other vectors $Y_i$ for $i = 2 \dots N_s$ are successively found with help of \equref{eq:ce-recurrent}. We also recall that $K_i$ is the matrix exponential, therefore, we use the following simplification:
\[
  \prod_{j = i}^l K_j = \prod_{j = i}^l e^{D t_j} = e^{D \sum_{j = i}^l t_j}
\]
since the product of each pair $D \: t_j$ and $D \: t_k$ is commutative.

\subsection{Sampling Interval}
Now, we make an important assumption about the time intervals $\triangle t_i$ in order to perform all the calculations in a much more efficient manner. We assume that \emph{the time intervals are equal}, $\triangle t_i = \triangle t$ for $i = 1 \dots N_s$, i.e., the distance in time between two successive power measurements stays constant. We refer to this distance as \emph{sampling interval}. The preferable size of this sampling interval depends on a particular application and the level of accuracy that we want to achieve. Having this assumption, the iterative process (\equref{eq:recurrent-equation}) turns into:
\[
  Y_{i+1} = K \: Y_i + G \: B_i
\]
where:
\begin{align*}
  & K = e^{D \: \triangle t} \\
  & G = D^{-1} \left( e^{D \: \triangle t} - I \right) C^{-\frac{1}{2}}
\end{align*}

It should be noted that $K$ and $G$ are constants, since they depend only on the matrices $D$, $C$, and the sampling interval $\triangle t$. Both $Y_i$ and $B_i$ are vectors of the length $N_n$. In this case, the block diagonal of the matrix $\mathbb{A}$ in \equref{eq:system} is composed of the same block, since $K_i = K$ for $i = 1 \dots N_s$.

Let us come back to the condensed equation presented in the previous section. The recurrent expressions in case of equal time intervals are the following:
\begin{align}
  & Y_{i + 1} = K \: Y_i + Q_i, \; i = 1 \dots N_s \nonumber \\
  & P_i = K \: P_{i - 1} + Q_i, \; i = 2 \dots N_s \nonumber \\
  & (I - K^{N_s}) Y_1 = P_{N_s} \label{eq:linear-system}
\end{align}

Again thank to the matrix exponential $K$ and eigenvalue decomposition, $K^{N_s}$ can be found very efficiently:
\begin{align*}
  K^{N_s} & = U \: e^{N_s \triangle t \: \Lambda} \: U^T = U \: e^{\mathcal{T} \Lambda} \: U^T \\
    & = U \left[
      \begin{array}{ccc}
        e^{\mathcal{T} \lambda_1} & \cdots & 0 \\
        \vdots & \ddots & \vdots \\
        0 & \cdots & e^{\mathcal{T} \lambda_{N_n}}
      \end{array}
    \right] U^T
\end{align*}
where $U$ is a square matrix of the eigenvectors (orthogonal) of $D \triangle t$, $\Lambda$ is a diagonal matrix of the eigenvalues, and $\mathcal{T}$ is the period of the application.

Substituting $K^{N_s}$ from the last equation into \equref{eq:linear-system}, we get:
\[
  (I - U \: e^{\mathcal{T} \Lambda} \: U^T) Y_1 = P_{N_s}
\]

The identity matrix $I$ can be thought as $U U^T$, therefore:
\begin{align*}
  & U (I - e^{\mathcal{T} \Lambda}) U^T \: Y_1 = P_{N_s} \\
  & Y_1 = U (I - e^{\mathcal{T} \Lambda})^{-1} U^T P_{N_s} \\
  & Y_1 = U M U^T P_{N_s}
\end{align*}
where $M$ is a diagonal matrix with the following structure:
\[
  M = \left[
    \begin{array}{ccc}
      \frac{1}{1 - e^{\mathcal{T} \lambda_1}} & \cdots & 0 \\
      \vdots & \ddots & \vdots \\
      0 & \cdots & \frac{1}{1 - e^{\mathcal{T} \lambda_{N_n}}}
    \end{array}
  \right]
\]

As we see, in this approach there is no need to inverse any matrix, the solution of the system is obtained by scalar divisions and a similarity transformation with $U$.


  \section{Reliability Model}
  The proposed solution of the steady-state dynamic temperature estimation can be used in a wide range of optimization procedures. One of them is the reliability optimization that we discuss in this section. We performing the temperature-aware task mapping and scheduling in order to address the thermal cycling aging effect while keeping the energy consumption on an appropriate level. Both mapping and scheduling are based on the genetic algorithms \cite{schmitz2004}. Let us start with the overall description of the system.

\subsection{Application Model}
The system executes a periodic application with a set of data-dependent tasks. The overall structure of the application is defined by a task graph:
\begin{align*}
  & G = (\mathcal{V}, \: E, \: \mathcal{T}) \\
  & \mathcal{V} = \{ v_i: \: i = 0 \dots N_t - 1 \} \\
  & E = \{ e_{ij} \}
\end{align*}
where $\mathcal{V}$ is a set of $N_t$ vertices of the graph (tasks), $E$ is a set of edges (data dependencies between tasks), and $\mathcal{T}$ is the period of the application. Each pair of a task $v_i$ and processing element $\pi_j$ is characterized by a tuple $(C_{eff \; ij}, N_{cycles \; ij})$, where $C_{eff \; ij}$ is the effective switched capacitance and $N_{cycles \: ij}$ is the number of clock cycles. These parameters determine the processor load and execution time of the task, correspondingly.

\subsection{Temperature-Aware Reliability Model}
In the paper we address temperature-driven failure mechanisms with the reliability model presented in \cite{huang2009}, \cite{xiang2010}. The model is based on the assumption that the failure rate has a Weibull distribution (e.g., the thermal cycling, electromigration, etc. \cite{jedec2010}):
\[
  R(t) = e^{-(\frac{t}{\eta})^\beta}
\]
where $\eta$ is the scaling parameter, $\beta$ is the shape (slope) parameter. The mean time to failure (MTTF) for the Weibull distribution is given by the following equation:
\begin{equation} \label{eq:general-mttf}
  MTTF = \eta \; \Gamma(1 + \frac{1}{\beta})
\end{equation}
where $\Gamma$ is the gamma function. The shape parameter is found to be independent on the temperature variation \cite{chang2006}, which is not the case with the scaling parameter $\eta$. Therefore, the distribution can vary from one set of conditions to another. We can use the same approach as it was shown previously and split the overall period of the application $\mathcal{T}$ into $N_m$ time intervals $\Delta t_i$, so that during each time interval $\Delta t_i$ all those conditions can be treated as constants, consequently, the corresponding $\eta_i$ is also constant. In this case, the cumulative distribution function by the end of the first execution of the application is the following \cite{huang2009}, \cite{xiang2010}:
\[
  R = e^{-(\sum_{i=0}^{N_m - 1} \frac{\Delta t_i}{\eta_i})^\beta}
\]

It can be shown that if $\eta_i$ are large enough\footnote{This is the case, since usually the MTTF is in order of tens of years (see \equref{eq:general-mttf}).}, the following continuous approximation can be applied \cite{xiang2010}:
\[
  R(t) = e^{-(\frac{t}{\mathcal{T}} \sum_{i=0}^{N_m - 1} \frac{\Delta t_i}{\eta_i})^\beta}
\]
The formula still keeps the form of the Weibull distribution with the following scaling parameter:
\[
  \eta = \frac{\mathcal{T}}{\sum_{i=0}^{N_m - 1} \frac{\Delta t_i}{\eta_i}}
\]

As it was mentioned earlier, the reliability model can be used to model different failure mechanisms. Let us now focus on one particular failure cause, the thermal cycling (TC) fatigue, that is a common packaging and interfacial failure mechanism \cite{jedec2010}. The number of cycles to failure can be estimated using a modified version of the well-known Coffin-Manson equation with the Arrhenius term \cite{jedec2010}, \cite{xiang2010}, \cite{ciappa2003}:
\begin{equation} \label{eq:cycles-to-failure}
  \mathcal{N} = A (\Delta T - \Delta T_0)^{-b} e^{\frac{E_a}{k T_{max}}}
\end{equation}
where $A$ is an empirically determined constant, $\Delta T$ is the thermal cycle amplitude, $\Delta T_0$ is the portion of the temperature range in the elastic region which does not cause damage, $b$ is the Coffin-Manson exponent which also empirically determined\footnote{This constant is found to be 6--9 for brittle fracture such as Si and its dielectrics \cite{jedec2010}.}, $E_{a}$ is the activation energy\footnote{For the thermal cycling failure mechanism the activation energy lies between 0.5 and 0.7~eV \cite{vigrass}.}, $k$ is the Boltzmann constant, and $T_{max}$ is the maximal temperature during the thermal cycle. Having the number of cycles to failure and the duration of one cycle $\Delta t$, we can compute the MTTF:
\[
  MTTF = \mathcal{N} \; \Delta t
\]

Since we consider the TC failure mechanism, the time intervals $\Delta t_i$ correspond to intervals of constant parameters of this particular mechanism that can be observed on \equref{eq:cycles-to-failure}. Each interval $\Delta t_i$ belongs to one thermal cycle with the scaling parameter $\eta_i$ derived using \equref{eq:general-mttf}:
\[
  \eta_i = \frac{MTTF_i}{\Gamma(1 + \frac{1}{\beta})}
\]
where $MTTF_i$ is the mean time to failure of the $i$th time interval as if we had the failure distribution of this interval all the time. Taking everything together, we get the following equation:
\begin{equation} \label{eq:one-mttf}
  MTTF = \frac{\mathcal{T}}{\sum_{i=0}^{N_m - 1} \frac{1}{\mathcal{N}_i}}
\end{equation}

\equref{eq:one-mttf} describes the MTTF of one component, which is a processing element in our case. We assume that each processing element is essential for the proper work of the system, therefore, a failure of any core leads to the total failure of the whole system. Consequently, the MTTF of the system can be estimated as the minimal MTTF among its components:
\begin{align*}
  & MTTF_{sys} = \min_{i=0}^{N_p - 1} \; MTTF_i \\
  & MTTF_i = \frac{\mathcal{T}}{\sum_{j=0}^{N_{m \: i} - 1} \frac{1}{\mathcal{N}_{ij}}}
\end{align*}
where $N_{m \: i}$ is the number of thermal cycles of the $i$th processing element within the application period $\mathcal{T}$ and $\mathcal{N}_{ij}$ is the number of thermal cycles to failure as if the $j$th cycle was being repeated all the time.

\subsection{Motivational Example} \label{sec:motivation}
\iimage{task-graph}{0 0 0 0}{Motivational example with 6 tasks, labeled from ``T0'' to ``T5'', running on a heterogeneous dual-core architecture. The execution times of the tasks vary between cores and are given on the figure along with the period of the application.}
\image{motivation}{100 240 100 240}{Three alternative combinations of mapping and scheduling (the left side) of the application shown in \figref{fig:task-graph} onto a dual-core architecture and corresponding steady-state dynamic temperature curves (the right side) for each of the cores.}
Consider a toy application with six tasks, denoted ``T0''--``T5'', and a heterogeneous architecture with two cores, labeled ``PE0'' and ``PE1''. The task graph of the application is given in \figref{fig:task-graph}. The initial mapping, schedule, and resulting SSDTP are shown at the top of \figref{fig:motivation}, where PE0 is experiencing three thermal cycles. If we change the allocation of T5 and move it to PE1, we achieve two thermal cycles of PE0 instead of three. Finally, if we vary the schedule as well and change the order of T1 and T3, the number of cycles of PE0 becomes one. Using the reliability model from \secref{sec:reliability}, we observe improvements in the MTTF of 45\% and 54\%, respectively, relative to the initial configuration.



  \section{Experimental Results} \label{sec:results}
  \subsection{Computation Performance} \label{sec:results-ssdtp}
In this subsection we investigate the scalability properties of the proposed solution for the SSDTP calculation and compare it with the approaches based on the FFT (\secref{sec:fast-fourier-transform}), TTA with the analytical solution (\secref{sec:tta-analytical}), and TTA with HotSpot (\secref{sec:hotspot-solution})\footnote{All the experiments are done on a Linux machine with Intel\textregistered\ Core\texttrademark\ i7-2600 (3.4GHz, 4 cores, 8 threads) and 8Gb of RAM.}. In the last two case, the TTA is run until the normalized RMSE relative to the SSDTP obtained with the CE method is less than 1\%. The sampling interval is equal to \mbox{1 $ms$}.

\subsubsection{Various Application Periods}
\iimage{scaling-time}{40 230 40 230}{Scalability with the application period for a quad-core architecture. The comparison is given on the semilogarithmic scale.}
\begin{itable}{scaling-time}{|r|r|r|r|r|r|r|}
  {Scalability with the application period shown in \figref{fig:scaling-time}.}
  {$\period$ --- application period, CE --- computational time of the CE method in $ms$, FFT --- speed-up relative to the FFT method, $TTA^{AS}$, $TTA^{HS}$ --- speed-up relative to the TTA with the analytical solution and TTA with HotSpot, respectively, along with the number of iterations.}
  \hline
  \multirow{2}{*}{$\period$, s} & \multirow{2}{*}{CE, ms} & \multirow{2}{*}{FFT, $\times$} & \multicolumn{2}{c|}{$TTA^{AS}$} & \multicolumn{2}{c|}{$TTA^{HS}$} \\ \cline{4-7}
  & & & $\times$ & Periods & $\times$ & Periods \\
  \hline
  \hline
  0.05 & 0.18 & 52.12 & 168.71 & 385 & 6750.96 & 689 \\
  0.10 & 0.39 & 41.64 &  77.91 & 191 & 5301.97 & 268 \\
  0.20 & 0.67 & 41.64 &  42.79 &  95 & 5188.80 & 122 \\
  0.30 & 1.03 & 39.09 &  27.97 &  63 & 5030.25 &  86 \\
  0.40 & 1.36 & 39.16 &  21.43 &  48 & 4887.90 &  63 \\
  0.50 & 1.70 & 43.56 &  16.99 &  38 & 4880.57 &  50 \\
  0.60 & 2.04 & 40.86 &  14.34 &  32 & 4899.20 &  42 \\
  0.70 & 2.32 & 42.46 &  12.87 &  28 & 4935.89 &  36 \\
  0.80 & 2.66 & 41.70 &  11.07 &  24 & 4842.62 &  31 \\
  0.90 & 2.98 & 42.54 &  10.21 &  22 & 4883.76 &  28 \\
  1.00 & 3.33 & 45.19 &   9.23 &  20 & 4892.86 &  25 \\
  \hline
\end{itable}
First, we vary the application period keeping the architecture fixed, which is a quad-core platform with the core area of 4 $mm^2$ and configuration shown in \tabref{tab:parameters}. The comparison is depicted in \figref{fig:scaling-time}. The computational time of the CE method and its speed-up relative to the rest are given in \tabref{tab:scaling-time}. It can be seen that the proposed technique is roughly 5000 times faster than the TTA with HotSpot and from 9 to 170 times faster than the TTA with the analytical solution. The number of application periods, over which the TTA is performed, is significantly larger for short application periods (\secref{sec:hotspot-iterative-solution}) while longer periods imply larger numbers of steps in the power profiles keeping the cost of the analysis high.

\subsubsection{Various Number of Cores}
\iimage{scaling-cores}{40 230 40 230}{Scalability with the number of cores. The comparison is given on the semilogarithmic scale.}
\begin{itable}{scaling-cores}{|r|r|r|r|r|r|r|}
  {Scalability with the number of cores shown in \figref{fig:scaling-cores}.}
  {$N_p$ --- number of cores, CE --- computational time of the CE method in $ms$, FFT --- speed-up relative to the FFT method, $TTA^{AS}$, $TTA^{HS}$ --- speed-up relative to the TTA with the analytical solution and TTA with HotSpot, respectively, along with the number of iterations.}
  \hline
  \multirow{2}{*}{$N_p$} & \multirow{2}{*}{CE, ms} & \multirow{2}{*}{FFT, $\times$} & \multicolumn{2}{c|}{$TTA^{AS}$} & \multicolumn{2}{c|}{$TTA^{HS}$} \\ \cline{4-7}
  & & & $\times$ & Periods & $\times$ & Periods \\
  \hline
  \hline
   2 &  0.77 & 70.79 & 15.74 & 33 & 3236.75 & 45 \\
   4 &  1.63 & 46.09 & 17.68 & 38 & 2906.30 & 50 \\
   8 &  4.67 & 30.08 & 20.40 & 44 & 2695.69 & 55 \\
  16 & 15.48 & 21.08 & 24.36 & 54 & 2434.74 & 62 \\
  32 & 56.44 & 13.13 & 27.24 & 62 & 2214.19 & 65 \\
  \hline
\end{itable}
The second part of the comparison is the scalability with the number of processing elements shown in \figref{fig:scaling-cores} and \tabref{tab:scaling-cores}. The configuration of the chip is similar to the previous experiment. In can be observed that the proposed solution provides a significant performance improvement relative to its competitors where in order to keep the same level of accuracy the TTA requires larger number of periods of the application to be analyzed.

\subsection{Reliability Optimization} \label{sec:reliability-results}
In this section we present the results of the reliability optimization described in \secref{sec:reliability-problem}. The experimental setup is the following. Heterogeneous platforms and periodic applications are generated randomly \cite{dick1998} in such a way that the execution time of tasks is uniformly distributed between 1 and 20 $ms$ and the leakage power accounts for 30--60\% of the total power dissipation. The area of one core is 4 $mm^2$, other parameters of the die and thermal package are given in \tabref{tab:parameters}. The temperature constraint $T_{max}$ (see \equref{eq:t-max}) is set to $100^\circ C$. In \equref{eq:cycles-to-failure} the Coffin-Manson exponent $b$ is set to 6 and the activation energy $E_a$ to 0.5, and the elastic temperature region $\Delta T_0$ to zero. The coefficient of proportionality $A$ is indifferent, since we are concerned about the relative improvement.

In each of the experiments, we compare the optimized solution with the initial temperature-aware solution described in \cite{xie2006}. First, we calculate the statical criticality (SC) of each task as the maximal distance from the task to the end of the task graph. Then, we schedule and map the application onto the platform where the next task from the ready list and an appropriate core are chosen based on their dynamic criticality (DC). The DC of a pair task/core depends on the SC, execution time, earliest possible start time on this particular core, and maximal steady-state temperature that the die can reach if the task is place on the core. This combination of mapping and scheduling is the starting point for the future optimization. The deadline of the application is set to the duration of the initial solution extended by 5\%.

\subsubsection{Various Number of Cores}
\begin{itable}{mttf-cores}{|r|r|r|r|r|}
  {Reliability optimization for different architectures}
  {$N_p$ --- number of cores, $N_t$ --- number of tasks, $t_{avg}$ --- computational time, $F_{avg}$ --- MTTF improvement, $E_{avg}$ --- decrease in the energy consumption.}
  \hline
  $N_p$ & $N_t$ & $t_{avg}, s$ & $F_{avg}$, $\times$ & $E_{avg}$, $\times$ \\
  \hline
  \hline
   2 &   40 &     7.84 &  39.41 & 0.97 \\
 % 4 &   80 &    65.76 & 144.08 & 0.98 \\ % 2 cases with unused PEs
 % 4 &   80 &    67.34 &  56.53 & 0.98 \\ % 1 close to
   4 &   80 &    65.76 &  37.11 & 0.99 \\
 % 8 &  160 &   784.88 &  44.10 & 0.95 \\ % 2 cases with unused PEs
   8 &  160 &   759.29 &  31.36 & 0.97 \\
  16 &  320 &  3484.59 &  13.51 & 0.98 \\
  32 &  640 &  7950.72 &   2.88 & 1.05 \\
  \hline
\end{itable}
In the first set of experiments, we change the number of cores while keeping the number of tasks per core constant and equal to 20. For each problem we have generated 20 random task graphs of a similar structure and found the average improvement of the MTTF. We also have measured the change in the consumed energy. The results are given in \tabref{tab:mttf-cores}. It can be observed that the thermal-cycling-unaware task allocation and scheduling dramatically decrease the lifetime of the device and the optimization based on the SSDTP is a must for an embedded system design framework. The complexity of the problem grows rapidly as it can be observed from the average number of evaluated solutions during the optimization of one task graph (\tabref{tab:mttf-cores}). Note that the energy efficiency of the system is not suffering from the optimization, on the contrary, typical solutions found by the GA have a lower level of the energy consumption, although, this is not the goal of this optimization.

\subsubsection{Various Number of Tasks} \label{sec:results-various-tasks}
\begin{itable}{mttf-tasks}{|r|r|r|r|r|}
  {Reliability optimization for different applications}
  {$N_p$ --- number of cores, $N_t$ --- number of tasks, $t_{avg}$ --- computational time, $F_{avg}$ --- MTTF improvement, $E_{avg}$ --- decrease in the energy consumption.}
  \hline
  $N_p$ & $N_t$ & $t_{avg}, s$ & $F_{avg}$, $\times$ & $E_{avg}$, $\times$ \\
  \hline
  \hline
  4 &  40 &  \todo{23.42} & 56.61 & 0.89 \\
  4 &  80 & \todo{101.44} & 32.46 & 0.99 \\
  4 & 160 & \todo{562.43} & 13.83 & 1.07 \\
  4 & 320 & \todo{657.54} & 11.97 & 1.05 \\
  4 & 640 & \todo{379.35} &  3.50 & 1.03 \\
  \hline
\end{itable}
For the second set of experiments, we keep the quad-core architecture and vary the number of tasks within the application. The number of randomly generated task graphs per problem is 20. The average improvement of the MTTF along with the change in the energy consumption are given in \tabref{tab:mttf-tasks}. The observations to be made here are similar to the previous ones: taking into consideration the SSDTP of the system during the design stage can significantly prolong the MTTF without sacrificing the energy efficiency of the system.

\subsubsection{Various Solution Techniques} \label{sec:results-various-techniques}
\begin{itable}{mttf-comparison}{|r|r|r|r|r|}
  {Reliability optimization for different solution techniques}
  {$N_p$ --- number of cores, $N_t$ --- number of tasks, $F^{CE}_{avg}$, $F^{HS}_{avg}$, and $F^{SS}_{avg}$ --- MTTF improvements obtained by the CE method, TTA with HotSpot, and SS approximation, respectively.}
  \hline
  $N_p$ & $N_t$ & $F^{CE}_{avg}$, $\times$ & $F^{HS}_{avg}$, $\times$ & $F^{SS}_{avg}$, $\times$ \\
  \hline
  \hline
  4 &  40 & 56.61 & 2.45 & 30.69 \\
  4 &  80 & 32.46 & 1.90 & 18.87 \\
  4 & 160 & 13.83 & \todo{0} &  4.74 \\
  4 & 320 & 11.97 & \todo{0} &  3.81 \\
  4 & 640 &  3.50 & \todo{0} &  2.46 \\
  \hline
\end{itable}
We compare the results of the optimization delivered by our method with the results obtained using the TTA with HotSpot (\secref{sec:hotspot-iterative-solution}) and steady-state approximation (\secref{sec:steady-state-approximation}) during the same computational time. In this case, the real SSDTP is assumed to be unknown for the TTA and HotSpot is stopped when the maximal difference between two successive iterations is less than $0.01^\circ C$ or the limit of 30 iterations is reached, although, it is not sufficient for a good accuracy (see \secref{sec:hotspot-iterative-solution}). The final solutions found by the later two methods are reevaluated using the CE method and compared with the solutions found only by the CE approach. The experimental setup is the same as in \secref{sec:results-various-tasks}. The results are summarized in \tabref{tab:mttf-comparison}. In can be seen that the proposed technique outperforms its competitors.

\subsubsection{Multi-Objective Optimization}
\iimage{average-pareto}{0 0 0 0}{The average Pareto front found by the multi-objective optimization for a quad-core architecture and 20 applications with 80 tasks each.}
In order to investigate the relation between the MTTF and energy in details, we have conducted experiments with a multi-objective optimization\footnote{The multi-objective optimization is based on the NSGA-II algorithm \cite{deb2002}.} where the energy minimization was added as the second goal. The result of such an optimization is a Pareto front, a set of non-dominant solutions for the designer to choose from. An example of a Pareto front averaged over 20 applications with 80 tasks deployed onto a quad-core platform is given in \figref{fig:average-pareto}. From the experiments we have observed that the difference in the energy consumption between the end points of the curves is typically low and usually bounded by 2--3\% while the MTTF range is much wider. Consequently, solutions optimized with respect to the cost function given by \equref{eq:fitness-function} do not compromise the energy efficiency of the system.

\subsubsection{Real-Life Example}
Finally, we have applied our optimization technique to a real-life example, namely the MPEG2 video decoder \cite{ffmpeg2011} that is deployed onto a dual-core architecture. The decoder was analysed and split into 21 tasks. The parameters of each task were obtained through a system-level simulation on the MPARM platform \cite{benini2005}. The deadline is set to 25 $ms$. The solution found by the GA with the CE method improves the lifetime of the system 32.64 times. The same optimization problem was solved using the steady-state approximation and TTA with HotSpot as it is described in \secref{sec:results-various-techniques}. The best found solutions are 23.79 and 19.36, respectively.


  \section{Conclusion}
  In this paper we shown the importance of the steady-state dynamic temperature analysis for different aspects of the multiprocessor system design. We formutated the problem and presented possible solutions. First, we discussed what can be achieved with the existing HotSpot simulator, and then we proposed an analytical solution to the problem. The solution is precise from the perspective of the underlying model and by far faster than possible alternatives with the same accuracy (to the best knowledge of the authors). We also considered the thermal cycling failure mechanism and conducted the reliability optimization based on it.


  \begin{thebibliography}{99}
  \bibitem{huang2006}
    Wei Huang, Shougata Ghosh, Siva Velusamy, Karthik Sankaranarayanan, Kevin Skadron, and Mircea R. Stan,
    ``HotSpot: A Compact Thermal Modeling Methodology for Early-Stage VLSI Design,''
    \emph{IEEE Transactions on Very Large Scale Integration (VLSI) Systems},
    vol.~14, No.~5, pages 501--513, May 2006.

  \bibitem{rao2008}
    Ravishankar Rao,
    ``Fast and Accurate Techniques for Early Design Space Exploration and Dynamic Thermal Management of Multi-Core Processors,''
    \emph{Arizona State University},
    2008.

  \bibitem{stoer2002}
    Josef Stoer and Roland Bulirsch,
    ``Introduction to Numerical Analysis,''
    \emph{Springer-Verlag, New York},
    third edition, 2002.

  \bibitem{xiang2010}
    Yun Xiang, Thidapat Chantem, Robert P. Dick, X. Sharon Hu, Li Shang,
    ``System-Level Reliability Modeling for MPSoCs,''
    \emph{CODES+ISSS'10},
    October 2010.

  \bibitem{liu2007}
    Yongpan Liu, Robert P. Dick, Li Shang, Huazhong Yang,
    ``Accurate Temperature-Dependent Integrated Circuit Leakage Power Estimation is Easy,''
    \emph{DATE'07},
    2007.

  \bibitem{schmitz2004}
    Marcus T. Schmitz, Bashir M. Al-Hashimi, Petru Eles,
    ``System-Level Design Techniques for Energy-Efficient Embedded Systems,''
    \emph{Kluwer Academic Publishers},
    2004.

  \bibitem{coskun2006}
    Ayse Kivilcim Coskun, Tajana Simunic Rosing, Kresimir Mihic, Giovanni De Micheli, and Yusuf Leblebici,
    ``Analysis and Optimization of MPSoC Reliability,''
    \emph{Journal of Low Power Electronics},
    vol.~2, pages 56–-69, 2006.

  \bibitem{hieu2004}
    Nguyen Van Hieu,
    ``Multilevel Interconnect Reliability,''
    \emph{Universiteit of Twente},
    2004.

  \bibitem{jedec2010}
    JEDEC Solid State Technology Association,
    ``Failure Mechanisms and Models for Semiconductor Devices,''
    \emph{JEDEC Publication},
    November 2010.

  \bibitem{mihic2004}
  Kresimir Mihic, Tajana Simunic and Giovanni De Micheli,
  ``Reliability and Power Management of Integrated Systems,''
  \emph{IEEE Transactions on Very Large Scale Integration (VLSI) Systems},
  vol.~15, No.4, pages 391--403, April 2004.

  \bibitem{simunic2005}
  T. Simunic, K. Mihic, G. De Micheli,
  ``Optimization of Reliability and Power Consumption in Systems on a Chip,''
  2005.

  \bibitem{lu2004}
  Zhijian Lu, Wei Huang, Shougata Ghosh, John Lach, Mircea Stan, Kevin Skadron,
  ``Analysis of Temporal and Spatial Temperature Gradients for IC Reliability,''
  March 2004.

\end{thebibliography}

\end{document}
