The proposed SSDTA framework is based on the well-known RC thermal model that employs the analogy between electrical and thermal circuits. According to this model, the temperature behaviour can be predicted by constracting equivalent RC thermal circuits where electrical current, voltage, resistance, and capacitance are replaced by heat flow, temperature, thermal resistance, and thermal capacitance, correspondingly. The heat equation in this model takes the following matrix form:
\begin{equation} \label{eq:thermal-ode}
  C \frac{dT}{dt} + G T = P
\end{equation}
where $T$ is a vector of temperature, $C$ is the thermal capacitance matrix, $G$ is the thermal conductance matrix (equal to $R^{-1}$, where $R$ is the resistance matrix), and $P$ is a vector of the power dissipation (the source of heat). $C$ and $G$ are $N_n \times N_n$ matrices, $T$ and $P$ are vectors of the length $N_n$, where $N_n$ is the number of thermal nodes. For convenience, in the rest of the paper we use the following denotation:
\begin{equation} \label{eq:initial}
  C \frac{dT}{dt} = A T + B
\end{equation}

An equivalent circuit of a multiprocessor system with a thermal package can be built in a wide range of different ways. Therefore, the number of thermal nodes $N_n$, as well as the conductance and capacitance matrices, depends on a particular realization of the model, i.e., its granularity. For instance, the wide spread thermal simulator HotSpot \cite{huang2006} includes two different implementations, so-called block and grid models. In the block model each core of a multiprocessor system is given one thermal node, while with the grid model the whole die is covered is an adjustable mesh of thermal nodes. Both models have a number of extra nodes for the thermal package of the die which includes the thermal interface material, heat spreader, and heat sink. For example, in case of the block model with $N_p$ processing elements, the total number of thermal nodes can be computed according to the following equation \cite{rao2008}:
\begin{equation} \label{eq:nodes}
  N_n = 4 \times N_p + 12
\end{equation}

The choice of a particular model depends on a desired accuracy. Without loss of generality, we shall use the block model of HotSpot in our experiments.
