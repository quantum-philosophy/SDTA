\subsection{Architecture Model} \label{sec:architecture-model}
We consider a heterogeneous multicore architecture with a set of processing elements $\Pi$ defined as the following:
\[
  \Pi = \{ \pi_i = (V_i, \: f_i, \: N_{gate \: i}): \; i = \range{0}{N_p - 1} \}
\]
where $V_i$, $f_i$, and $N_{gate \: i}$ are the supply voltage, frequency, and number of gates \cite{liao2005} of the $i$th core, respectively. The first two parameters are used for the dynamic power estimation while the last one for the leakage power as described in the following subsection.

\subsection{Power Model} \label{sec:power-model}
The total power dissipation of a processing element is defined as the sum of the dynamic and leakage power:
\[
  P = P_{dyn} + P_{leak} \hs P_{dyn} = C_{eff} \cdot f \cdot V^2
\]
where $C_{eff}$ is the effective switched capacitance, $V$ and $f$ are the supply voltage and frequency, respectively. The leakage part of the power dissipation is defined as \cite{liao2005}:
\begin{equation} \label{eq:total-power}
  P_{leak}(T) = N_{gate} \: V \: I_0 \left[ A \: T^2 e^{\frac{\alpha \: V + \beta}{T}} + B e^{(\gamma \: V + \delta)} \right]
\end{equation}
where $T$ and $V$ are the current temperature and supply voltage, respectively, $N_{gate}$ is the number of gates in the circuit, $I_0$ is the average leakage current at the reference temperature and supply voltage. $A$, $B$, $\alpha$, $\beta$, $\gamma$, and $\delta$ are the technology dependent constants found in \cite{liao2005}.

\subsection{Thermal Model} \label{sec:thermal-model}
Our proposed technique is based on the RC thermal model that employs the analogy between electrical and thermal circuits \cite{kreith2000}. Heat transfer is modeled with the following system of differential equations:
\begin{equation} \label{eq:fourier-model}
  \m{C} \: \frac{d\v{T}(t)}{dt} + \m{G} \: (\v{T}(t) - \v{T}_{amb})= \v{P}(t)
\end{equation}
where $\v{T}$ is the temperature vector, $\v{T}_{amb}$ is the ambient temperature vector, $\m{C}$ is the thermal capacitance matrix, $\m{G}$ is the thermal conductance matrix, and $\v{P}$ is the power dissipation vector (the source of heat). The dimensions of the system are $N_n \times N_n$, where $N_n$ is the number of thermal nodes.

The equivalent circuit of a multiprocessor system with a thermal package can be built in different ways depending on the intended level of details. Consequently, the number of nodes $N_n$ and structure of matrices $\m{C}$ and $\v{G}$ depend on a particular implementation of the model. Thermal nodes that belong to the package are called inactive in the sense that their power dissipation is assumed to be zero.

\iimage{circuit}{-50 0 -50 0}{A simplified equivalent RC thermal circuit of a dual-core die with a thermal package.}
Without loss of generality, in the paper we use thermal circuits constructed according to the block model used in HotSpot 5.0 \cite{huang2003} where each of the $N_p$ processing elements is captured by one thermal node. In the model, three cooling layers are present, namely the thermal interface material, heat spreader, and heat sink captured by $N_p$, $N_p + 4$, and $N_p + 8$ inactive thermal nodes, respectively. Therefore, the total number of thermal nodes $N_n$ is $4 \times N_p + 12$. A simplified example of such a thermal circuit for a dual-core architecture is depicted in \figref{fig:circuit}. It can be seen that the inter-core dependency is taken into account by modeling the heat flux between the cores (the top two thermal nodes) with the corresponding thermal resistance.

Some or all cores can be also modeled at a higher level of granularity where caches, ALUs, or registers will be captured as individual thermal nodes.
