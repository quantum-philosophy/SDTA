\subsection{The RC Thermal Model}
In our analysis we use the well-known thermal model called HotSpot. HotSpot is a modeling methodology for developing compact thermal models based on the popular stacked-layer packaging scheme in modern very large-scale integration systems \cite{huang2006}. The model is based on the following differential equation \cite{rao2008}:
\[
  C \frac{dT}{dt} + G T = P
\]
where $C$ is the thermal capacitance matrix, $G$ is the thermal conductance matrix (equal to $\frac{1}{R}$, where $R$ is the resistance matrix), $P$ is the power dissipation, and $T$ is the temperature vector. The matrices are of the dimensions $n \times n$, the vector is of the length $n$, where $n$ is the number of thermal nodes.

For convenience, in the rest of the paper we use the following denotation:
\begin{equation} \label{eq:initial}
  C \frac{dT}{dt} = A T + B
\end{equation}

HotSpot comes with a tool of the same name to perform all the calculations. If we are interested in the transient temperature, it solves \equref{eq:initial} using the Runge-Kutta numerical method. When we need to get only the steady-state temperature, $\frac{dT}{dt}$ becomes zero leaving a linear system, which is HotSpot solves the linear system with help of the regular LU decomposition.
