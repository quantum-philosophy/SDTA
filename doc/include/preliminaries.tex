In our analysis we use the well-known thermal model called HotSpot \cite{huang2006}. The model is based on the analogy between electrical and thermal circuits and can be described with the following differential equation \cite{rao2008}:
\begin{equation} \label{eq:thermal-ode}
  C \frac{dT}{dt} + G T = P
\end{equation}
where $C$ is the thermal capacitance matrix, $G$ is the thermal conductance matrix (equal to $\frac{1}{R}$, where $R$ is the resistance matrix), $P$ is a vector of the power dissipation, and $T$ is the temperature vector. The matrices are of the dimensions $N_n \times N_n$, the vector is of the length $N_n$, where $N_n$ is the number of thermal nodes.

For convenience, in the rest of the paper we use the following denotation:
\begin{equation} \label{eq:initial}
  C \frac{dT}{dt} = A T + B
\end{equation}

HotSpot comes with a tool of the same name to perform all the calculations. If we are interested in the transient temperature, it solves \equref{eq:initial} using the Runge-Kutta numerical method. When we need to get only the steady-state temperature, $\frac{dT}{dt}$ becomes zero, and \equref{eq:initial} turns into a linear system, which HotSpot solves using the LU decomposition.

Note the number of thermal nodes $N_n$ depends on a particular realization of the model, i.e., its granularity. The HotSpot simulator includes two different implementations: the block and grid models. In the block model each core of a multiprocessor system is given one thermal node, while with the grid model the whole die is covered is an adjustable mesh of thermal nodes. Both models have a number of extra nodes for the package of the die\footnote{The HotSpot thermal model includes the thermal interface material, heat spreader, and heat sink.}. For instance, in case of the block model with $N_p$ processing elements, the total number of thermal nodes can be computed as the following \cite{rao2008}:
\begin{equation} \label{eq:nodes}
  N_n = 4 \times N_p + 12
\end{equation}

The choice of the model depends on a desired accuracy. Without loss of generality, we shall use the block model.
