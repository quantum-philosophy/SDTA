\subsection{Architecture Model}
We consider a heterogeneous multicore architecture with a set of processing elements $\Pi$ defined as the following \cite{liao2005}:
\begin{align*}
  & \Pi = \{ \pi_k: \; k = 0 \dots N_p - 1 \} \\
  & \pi_k \rightarrow (V_k, \: f_k, \: N_{gate \: k})
\end{align*}
where $V_k$, $f_k$, and $N_{gate \: k}$ are the supply voltage, frequency, and number of gates of the $k$th core, correspondingly. The first two parameters are used for the dynamic power estimation while the last one for the power leakage as described in the following subsection.

\subsection{Power Model}
The total power dissipation of a processing element is defined as a sum of the dynamic power and power leakage:
\begin{align*}
  & P = P_{dyn} + P_{leak} \\
  & P_{dyn} = C_{eff} \; f \; V^2
\end{align*}
where $C_{eff}$ is the effective switched capacitance, $V$ and $f$ are the supplied voltage and frequency, correspondingly. The model for the leakage part of the power dissipation is presented in \cite{liao2005} and defined as:
\begin{align*}
  & P_{leak}(T) = N_{gate} \: I_{avg}(T) \: V \\
  & I_{avg}(T) = I_0 \: f_{avg}(T) \\
  & f_{avg}(T) = A \: T^2 e^{((\alpha \: V + \beta)/T)} + B e^{(\gamma \: V + \delta)}
\end{align*}
where $N_{gate}$ is the number of gates in the circuit\footnote{The number of gates in a circuit can be estimated using the technique described in \cite{li2004}.}, $I_0$ is the average leakage current at the reference temperature $T_0$ and supply voltage $V_0$, $f_{avg}$ is the scaling function. $I_{avg}$ is the average leakage current at the current temperature $T$ and voltage $V$. $A$, $B$, $\alpha$, $\beta$, $\gamma$, and $\delta$ are the technology dependent constants found in \cite{liao2005}.

\subsection{RC Thermal Model}
The proposed SSDTA technique is based on the well-known RC thermal model that employs the analogy between electrical and thermal circuits \cite{kreith2000}. Heat transfer is modeled with the following system of differential equations:
\begin{equation} \label{eq:fourier-model}
  C \: \frac{dT(t)}{dt} + G \: (T(t) - T_{amb})= P(t)
\end{equation}
where $T$ is the temperature vector, $T_{amb}$ is the ambient temperature vector, $C$ is the thermal capacitance matrix, $G$ is the thermal conductance matrix, and $P$ is the power dissipation vector (the source of heat). The dimensions of the system are $N_n \times N_n$, where $N_n$ is the number of thermal nodes.

An equivalent circuit of a multiprocessor system with a thermal package can be built in a wide range of different ways, therefore, the number of thermal nodes $N_n$ and structure of matrices $C$ and $G$ depend on a particular implementation of the model.
