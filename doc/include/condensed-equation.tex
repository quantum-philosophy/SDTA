Now we shall propose a much faster solution. Let us return back to the system of linear equations that we are to solve. It is described with the following recurrence:
\begin{equation} \label{eq:ce-recurrent}
  Y_{i + 1} = K_i \: Y_i + Q_i, \; i = 0 \dots N_s - 1
\end{equation}

A common technique to solve such systems is to form a so-called \emph{condensed equation} (CE), or \emph{condensed system} \cite{stoer2002}. Let us undertake this procedure step by step.

The iterative repetition of this equation leads us to:
\begin{equation} \label{eq:y-recurrent}
  Y_i = \prod_{j = 0}^{i - 1} K_j \: Y_0 + P_{i - 1}, \; i = 1 \dots N_s
\end{equation}
where $P_i$ are defined as the following:
\begin{align}
  P_0 & = Q_0 \nonumber \\
  P_i & = \sum_{l = 1}^i \prod_{j = l}^i K_j \: Q_{l - 1} + Q_i, \: i = 1 \dots N_s - 1 \nonumber \\
  P_i & = K_i \: P_{i - 1} + Q_i, \; i = 1 \dots N_s - 1 \label{eq:p-recurrent}
\end{align}

Therefore, we can calculate the final value $Y_{N_s}$ from \equref{eq:y-recurrent}:
\[
  Y_{N_s} = \prod_{j = 0}^{N_s - 1} K_j \: Y_0 + P_{N_s - 1}
\]

Taking into account the boundary condition given by \equref{eq:boundary-condition}, we obtain the following system of linear equations:
\begin{equation} \label{eq:core-system}
  (I - \prod_{j = 0}^{N_s - 1} K_j) \: Y_0 = P_{N_s - 1}
\end{equation}

Now we recall that $K_i$ is the matrix exponential, therefore, the following simplification holds:
\begin{align*}
  \prod_{j = i}^l K_j = \prod_{j = i}^l e^{D \triangle t_j} & = e^{D \sum_{j = i}^l \triangle t_j} \\
  & = U e^{\left( \sum_{j = i}^l \triangle t_j \: \Lambda \right)} U^T
\end{align*}
since the product of each pair $D \: \triangle t_j$ and $D \: \triangle t_k$ is commutative. Therefore:
\begin{align*}
  \prod_{j = 0}^{N_s - 1} K_j & = e^{D \mathcal{T}} = U \: e^{\mathcal{T} \Lambda} \: U^T \\
    & = U \left[
      \begin{array}{ccc}
        e^{\mathcal{T} \lambda_0} & \cdots & 0 \\
        \vdots & \ddots & \vdots \\
        0 & \cdots & e^{\mathcal{T} \lambda_{N_n - 1}}
      \end{array}
    \right] U^T
\end{align*}
where $U$ is a square matrix of the eigenvectors of $D$, $\Lambda$ is a diagonal matrix of the eigenvalues, and $\mathcal{T}$ is the period of the application. Substituting this product into \equref{eq:core-system}, we obtain the following system:
\[
  (I - U \: e^{\mathcal{T} \Lambda} \: U^T) Y_0 = P_{N_s - 1}
\]

The identity matrix $I$ can be represented as $U U^T$, consequently:
\begin{align*}
  & U (I - e^{\mathcal{T} \Lambda}) U^T \: Y_0 = P_{N_s - 1} \\
  & Y_0 = U (I - e^{\mathcal{T} \Lambda})^{-1} U^T P_{N_s - 1} \\
  & Y_0 = U M U^T P_{N_s - 1}
\end{align*}
where $M$ is a diagonal matrix with the following structure:
\[
  M = \left[
    \begin{array}{ccc}
      \frac{1}{1 - e^{\mathcal{T} \lambda_0}} & \cdots & 0 \\
      \vdots & \ddots & \vdots \\
      0 & \cdots & \frac{1}{1 - e^{\mathcal{T} \lambda_{N_n - 1}}}
    \end{array}
  \right]
\]

The solution of this system, $Y_0$, is the first component of the vector $\mathbb{Y}$. $P_{N_s - 1}$ can be calculated using \equref{eq:p-recurrent}. All other vectors $Y_i$ for $i = 1 \dots N_s - 1$ are successively found with help of \equref{eq:ce-recurrent}.

As we see, in this approach there is no need to inverse any matrix, the solution of the system is obtained by scalar divisions and a similarity transformation with $U$. The experimental results given in \secref{sec:results-ssdtp} show that this solution has the highest performance without any loss of accuracy among other methods that we have considered in our analysis.

Let us now consider one specific case, we make one assumption concerning the time intervals $\triangle t_i$ that allows us to perform all the calculations in a more efficient manner. We assume that \emph{the time intervals are equal}, $\triangle t_i = \triangle t$ for $i = 0 \dots N_s - 1$, i.e., the distance in time between two successive power measurements stays constant. We refer to this distance as the \emph{sampling interval}. The preferable size of this sampling interval depends on a particular application and the level of accuracy that we want to achieve. Having this assumption, the recurrent process (\equref{eq:recurrent-equation}) turns into:
\[
  Y_{i+1} = K \: Y_i + G \: B_i
\]
where:
\begin{align*}
  & K = e^{D \: \triangle t} \\
  & G = D^{-1} \left( e^{D \: \triangle t} - I \right) C^{-\frac{1}{2}}
\end{align*}

It should be noted that now $K$ and $G$ are constants, since they depend only on the matrices $D$, $C$, and the sampling interval $\triangle t$, which is fixed. In this case, the block diagonal of the matrix $\mathbb{A}$ in \equref{eq:system} is composed of the same repeating block $K_i = K$ for $i = 0 \dots N_s - 1$, and the recurrent expressions take the following form:
\begin{align}
  & Y_{i + 1} = K \: Y_i + Q_i, \; i = 0 \dots N_s - 1 \nonumber \\
  & P_i = K \: P_{i - 1} + Q_i, \; i = 1 \dots N_s - 1 \nonumber
\end{align}
