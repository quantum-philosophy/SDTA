Before going to the proposed solution itself, we first perform one important adjustment to \equref{eq:fourier-model} in order to be more efficient in our future calculations. We notice that the analytical solution in \equref{eq:solution} includes the matrix exponential and inverse of the product $C^{-1} \: G$, which is an arbitrary square matrix. It is preferable to have a symmetrix matrix to do such computations, since for a real symmetric matrix $M$ the following eigenvalue decomposition holds:
\begin{equation} \label{eq:eigenvalue-decomposition}
  M = U \Lambda U^T
\end{equation}
where $U$ is a square matrix of the eigenvectors of $M$ and $\Lambda$ is a diagonal matrix composed of the eigenvalues $\lambda_i$. Once we have obtained such a decomposition, calculation of the matrix exponential and inverse becomes a trivial task:
\begin{align*}
  & e^M = U \; e^{\Lambda} \; U^T = U \: \left[
      \begin{array}{ccc}
        e^{\lambda_0} & \cdots & 0 \\
        \vdots & \ddots & \vdots \\
        0 & \cdots & e^{\lambda_{n - 1}}
      \end{array}
    \right] \; U^T \\
  & M^{-1} = U \: \Lambda^{-1} \: U^T = U \left[
      \begin{array}{ccc}
        \frac{1}{\lambda_0} & \cdots & 0 \\
        \vdots & \ddots & \vdots \\
        0 & \cdots & \frac{1}{\lambda_{n - 1}}
      \end{array}
    \right] U^T \\
\end{align*}

Since the conductance matrix $G$ is a symmetric matrix (and the capacitance $C$ a diagonal matrix) \cite{rao2007}, we can apply the following substitution:
\begin{align*}
  & \tilde{T}(t) = C^{\frac{1}{2}} T(t) \\
  & \tilde{A} = -C^{-\frac{1}{2}} G \: C^{-\frac{1}{2}}
\end{align*}
with the result:
\begin{align*}
  & \frac{d\tilde{T}(t)}{dt} = \tilde{A} \: Y(t) + C^{-\frac{1}{2}} P \\
  & \tilde{T}(t) = e^{\tilde{A} t} \tilde{T}_0 + \tilde{A}^{-1} (e^{\tilde{A} t} - I) C^{-\frac{1}{2}} P
\end{align*}
where $\tilde{A}$ is a symmetric matrix. Therefore, in case of the matrix exponential we have:
\[
  e^{\tilde{A} t} = U \: e^{\Lambda t} \: U^T = U \left[
      \begin{array}{ccc}
        e^{t \lambda_0} & \cdots & 0 \\
        \vdots & \ddots & \vdots \\
        0 & \cdots & e^{t \lambda_{N_n - 1}}
      \end{array}
    \right] U^T
\]
A similar equation can be obtained for the matrix inverse. It should be noted that the enginvalue decomposition is performed only once for a particular RC thermal circuit and can be used for all future calculations.

The next step is to update the SSDTP system given in \equref{eq:recurrent-system}:
\begin{align}
  & \tilde{T}_{i+1} = \tilde{K}_i \: \tilde{T}_i + \tilde{B}_i \: P_i \label{eq:recurrent-equation} \\
  & \tilde{K}_i = e^{\tilde{A} \: \Delta t_i} \nonumber \\
  & \tilde{B}_i = \tilde{A}^{-1} \left( e^{\tilde{A} \Delta t_i} - I \right) C^{-\frac{1}{2}} \nonumber
\end{align}
Using the eigenvalue decomposition, the last equation can be computed in the following way:
\begin{align*}
  \tilde{B}_i & = U \: \Lambda^{-1} \: U^T \left(U \: e^{\Lambda \Delta t_i} \: U^T - U \: U^T \right) C^{-\frac{1}{2}} = \\
      & = U \left[
        \begin{array}{ccc}
          \frac{e^{\Delta t_i \: \lambda_0} - 1}{\lambda_0} & \cdots & 0 \\
          \vdots & \ddots & \vdots \\
          0 & \cdots & \frac{e^{\Delta t_i \: \lambda_{N_n - 1}} - 1}{\lambda_{N_n - 1}}
        \end{array}
      \right] U^T \: C^{-\frac{1}{2}}
\end{align*}

Now we shall describe the proposed solution. Let us return back to the recurrent system and denote $Q_i = \tilde{B}_i \: P_i$:
\begin{align}
  & \tilde{T}_{i + 1} = \tilde{K}_i \: \tilde{T}_i + Q_i, \; i = 0 \dots N_s - 1 \label{eq:ce-recurrent} \\
  & \tilde{T}_0 = \tilde{T}_{N_s + 1} \nonumber
\end{align}
A common technique to solve such systems is to form a so-called \emph{condensed equation} (CE), or \emph{condensed system} \cite{stoer2002}. Let us undertake this procedure step by step.

The iterative repetition of \equref{eq:ce-recurrent} results in:
\begin{equation} \label{eq:y-recurrent}
  \tilde{T}_i = \prod_{j = 0}^{i - 1} \tilde{K}_j \: \tilde{T}_0 + W_{i - 1}, \; i = 1 \dots N_s
\end{equation}
where $W_i$ are defined as the following:
\begin{align}
  W_0 & = Q_0 \nonumber \\
  W_i & = \sum_{l = 1}^i \prod_{j = l}^i \tilde{K}_j \: Q_{l - 1} + Q_i, \: i = 1 \dots N_s - 1 \nonumber \\
  W_i & = \tilde{K}_i \: W_{i - 1} + Q_i, \; i = 1 \dots N_s - 1 \label{eq:p-recurrent}
\end{align}
Therefore, we can calculate the final value $\tilde{T}_{N_s}$ from \equref{eq:y-recurrent}:
\[
  \tilde{T}_{N_s} = \prod_{j = 0}^{N_s - 1} \tilde{K}_j \: \tilde{T}_0 + W_{N_s - 1}
\]
Taking into account the boundary condition given by \equref{eq:boundary-condition}, we obtain the following system of linear equations:
\begin{equation} \label{eq:core-system}
  (I - \prod_{j = 0}^{N_s - 1} \tilde{K}_j) \: \tilde{T}_0 = W_{N_s - 1}
\end{equation}
We recall that $\tilde{K}_i$ is the matrix exponential, therefore, the following simplification holds:
\begin{align*}
  \prod_{j = i}^l \tilde{K}_j = \prod_{j = i}^l e^{\tilde{A} \Delta t_j} & = e^{\tilde{A} \sum_{j = i}^l \Delta t_j} \\
  & = U e^{\left( \sum_{j = i}^l \Delta t_j \: \Lambda \right)} U^T
\end{align*}
Consequently:
\begin{align*}
  \prod_{j = 0}^{N_s - 1} \tilde{K}_j & = e^{\tilde{A} \mathcal{T}} = U \: e^{\mathcal{T} \Lambda} \: U^T \\
    & = U \left[
      \begin{array}{ccc}
        e^{\mathcal{T} \lambda_0} & \cdots & 0 \\
        \vdots & \ddots & \vdots \\
        0 & \cdots & e^{\mathcal{T} \lambda_{N_n - 1}}
      \end{array}
    \right] U^T
\end{align*}
where $\mathcal{T}$ is the overall period. Substituting this product into \equref{eq:core-system}, we obtain the following system:
\[
  (I - U \: e^{\mathcal{T} \Lambda} \: U^T) \: \tilde{T}_0 = W_{N_s - 1}
\]
Finally, the identity matrix $I$ can be splitted into $U U^T$, hence:
\begin{align*}
  & \tilde{T}_0 = U \: (I - e^{\mathcal{T} \Lambda})^{-1} \: U^T \: W_{N_s - 1} \\
  & \tilde{T}_0 = U \: \left[
      \begin{array}{ccc}
        \frac{1}{1 - e^{\mathcal{T} \lambda_0}} & \cdots & 0 \\
        \vdots & \ddots & \vdots \\
        0 & \cdots & \frac{1}{1 - e^{\mathcal{T} \lambda_{N_n - 1}}}
      \end{array}
    \right] \: U^T \: W_{N_s - 1}
\end{align*}
Here $W_{N_s - 1}$ can be calculated using \equref{eq:p-recurrent}. As we see, there is no need to inverse any matrix, the solution of the system is obtained by scalar divisions and a similarity transformation with $U$. All other vectors $\tilde{T}_i$ for $i = 1 \dots N_s - 1$ are successively found from \equref{eq:ce-recurrent}.

Let us now consider one specific case where all the computation can be performed in a much more effecient manner. Assume that the power profile $\mathbb{P}$ is evenly sampled with the \emph{sampling interval} equal to $\Delta t$, i.e., $\Delta t_i = \Delta t$ for $i = 0 \dots N_s - 1$. Having this assumption, the recurrent process (\equref{eq:recurrent-equation}) turns into:
\[
  \tilde{T}_{i+1} = \tilde{K} \: \tilde{T}_i + \tilde{B} \: P_i
\]
where:
\begin{align*}
  & \tilde{K} = e^{\tilde{A} \: \Delta t} \\
  & \tilde{B} = \tilde{A}^{-1} \left( e^{\tilde{A} \: \Delta t} - I \right) C^{-\frac{1}{2}}
\end{align*}
Here $\tilde{K}$ and $\tilde{B}$ are constants, since they depend only on the matrices $\tilde{A}$, $C$, and sampling interval $\Delta t$, which is fixed. In this case, the block diagonal of the matrix $\tilde{\mathbb{K}}$, similar to \equref{eq:system}, is composed of the same repeating block $\tilde{K}$ and the recurrent expressions take the following form:
\begin{align}
  & \tilde{T}_{i + 1} = \tilde{K} \: \tilde{T}_i + Q_i, \; i = 0 \dots N_s - 1 \nonumber \\
  & W_i = \tilde{K} \: W_{i - 1} + Q_i, \; i = 1 \dots N_s - 1 \nonumber
\end{align}
