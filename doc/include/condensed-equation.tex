In this section we propose a fast approach to solve the system given by \equref{eq:system}. The approach consists of an auxiliary transformation (\secref{sec:ce-auxiliary}) and the solution itself (\secref{sec:ce-solution}).

The major problem with the earlier-discussed techniques is that (1) the sparseness of the matrix is not taken into account, leading to an enormous memory consumption and prohibitive solutions, and/or (2) its specific structure, which is discussed later, is totally ignored, resulting in an inefficiency of the computations and inaccuracy. Our proposed technique considers both features and delivers solutions in time proportional to $N_s N_n^3$ while operating only on a few $N_n \times N_n$ matrices. The dependency on the number of steps in the power profile $N_s$, which it typically by far dominats, i.e., $N_s \gg N_n$, is linear. The preparatory work, needed by the technique, is done only once for a particular RC thermal circuit and can be considered as a given for various power profiles, which is not always the case with the other solutions.

\iimage{sparseness-of-system}{-60 210 -60 210}{Sparseness of the system of linear equations for the SSDTP calculation.}

Let us start with from the discussion on the above-mentioned specific structure, which importance is immense, of the system given by \equref{eq:system} and depicted in \figref{fig:sparseness-of-system}. It can be seen that non-zero elements, marked with blue points in the figure, are located only on the block diagonal, one subdiagonal just above the block diagonal, and one subdiagonal in the left bottom corner. The block diagonal is composed of $N_n \times N_n$ matrices while all elements of the subdiagonals are equal \mbox{to $-1$}. Linear systems with the same structure arise in boundary value problems for ODEs where a common technique to solve them is to form a so-called condensed equation (CE), or condensed system \cite{stoer2002}. Before undertaking this procedure, we perform an adjustment to the initial analytical solution described in the following subsection.

\subsection{Auxiliary Transformation} \label{sec:ce-auxiliary}
It can be noticed that the analytical solution in \equref{eq:solution} includes two computationally expensive operations, namely the matrix exponential and inverse involving \mbox{$\m{A} = - \m{C}^{-1} \: \m{G}$}, which is an arbitrary square matrix. It is preferable to have a symmetric matrix to perform these computations, since for a real symmetric matrix $\m{M}$ the following eigenvalue decomposition with independent eigenvectors holds \cite{press2007}:
\begin{equation} \label{eq:eigenvalue-decomposition}
  \m{M} = \m{U} \m{\Lambda} \m{U}^\v{T}
\end{equation}
where $\m{U}$ is a square matrix of the eigenvectors, $\m{U}^T$ is the transpose of $\m{U}$, and $\m{\Lambda}$ is a diagonal matrix of the eigenvalues of $\m{M}$. Once we have obtained such a decomposition, the calculation of the matrix exponential and inverse becomes a trivial task:
\begin{align*}
  & e^\m{M} = \m{U} \; e^{\m{\Lambda}} \; \m{U}^\v{T} = \m{U} \: \diag{e^{\lambda_0}}{e^{\lambda_{n - 1}}} \: \m{U}^\v{T} \\
  & \m{M}^{-1} = \m{U} \: \m{\Lambda}^{-1} \: \m{U}^\v{T} = \m{U} \: \diag{\frac{1}{\lambda_0}}{\frac{1}{\lambda_{n - 1}}} \: \m{U}^\v{T} \\
\end{align*}
where $diag$ denotes a diagonal matrix and $\lambda_i$ are eigenvalues of $\m{M}$.

The conductance matrix $\m{G}$ is a symmetric matrix, since if a node A is connected to B, then B is also connected to A with the same conductance \cite{huang2003}. However, as it is mentioned previously, the product of $\m{G}$ with the inverse of the capacitance matrix $\m{C}$ does not have this property. We propose to use the following transformation in order to keep the desired symmetry:
\begin{equation} \label{eq:substitution}
  \tilde{\v{T}}(t) = \m{C}^{\frac{1}{2}} \v{T}(t) \hs \tilde{\m{A}} = -\m{C}^{-\frac{1}{2}} \m{G} \: \m{C}^{-\frac{1}{2}}
\end{equation}
where $\tm{A}$ is symmetric, since $\tm{A}^T = -(\m{C}^{-\half} \m{G} \m{C}^{-\half})^T = -(\m{C}^{-\half})^T \m{G}^T (\m{C}^{-\half})^T = \tilde{\m{A}}$. Note that the capacitance matrix $\m{C}$ is a diagonal matrix \cite{huang2003}, hence, $\m{C}^\half$ takes linear time to compute. Consequently, the system of ODEs (\equref{eq:fourier-model}) and its solutions (\equref{eq:solution}) can be rewritten as the following:
\begin{align*}
  & \frac{d\tilde{\v{T}}(t)}{dt} = \tilde{\m{A}} \: \m{Y}(t) + \m{C}^{-\frac{1}{2}} \v{P} \\
  & \tilde{\v{T}}(t) = e^{\tilde{\m{A}} t} \tilde{\v{T}}_0 + \tilde{\m{A}}^{-1} (e^{\tilde{\m{A}} t} - \m{I}) \m{C}^{-\frac{1}{2}} \v{P}
\end{align*}
where $\tilde{\m{A}}$ is a symmetric matrix. Therefore, in the case of, e.g., the matrix exponential we have:
\begin{equation} \label{eq:matrix-exponential}
  e^{\tm{A} t} = \m{U} \: e^{\m{\Lambda} t} \: \m{U}^T = \m{U} \: \diag{e^{t \lambda_0}}{e^{t \lambda_{N_n - 1}}} \: \m{U}^T
\end{equation}
where $\lambda_i$ are eigenvalues of $\tilde{\m{A}}$. A similar equation can be obtained for the matrix inverse.

The next step is to update the SSDTP system given in \equref{eq:recurrent-system}:
\begin{align}
  & \tv{T}_{i+1} = \tm{K}_i \: \tv{T}_i + \tm{B}_i \: \v{P}_i \label{eq:recurrent-equation} \\
  & \tm{K}_i = e^{\tm{A} \: \Delta t_i} \hs \tm{B}_i = \tm{A}^{-1} \left( e^{\tilde{\m{A}} \Delta t_i} - \m{I} \right) \m{C}^{-\frac{1}{2}} \nonumber
\end{align}
Using the eigenvalue decomposition, the last equation can be computed in the following way:
\begin{align*}
  \tilde{\m{B}}_i & = \m{U} \: \m{\Lambda}^{-1} \: \m{U}^T \left(\m{U} \: e^{\m{\Lambda} \Delta t_i} \: \m{U}^T - \m{U} \: \m{U}^T \right) \m{C}^{-\frac{1}{2}} = \\
  & = \m{U} \: \diag{\frac{e^{\Delta t_i \: \lambda_0} - 1}{\lambda_0}}{\frac{e^{\Delta t_i \: \lambda_{N_n - 1}} - 1}{\lambda_{N_n - 1}}} \: \m{U}^T \: \m{C}^{-\frac{1}{2}}
\end{align*}

\subsection{Solution with Condensed Equation (CE)} \label{sec:ce-solution}
Let us return back to the recurrent system given by \equref{eq:recurrent-equation} and denote \mbox{$\m{Q}_i = \tilde{\m{B}}_i \: \v{P}_i$}:
\begin{align}
  & \tilde{\v{T}}_{i + 1} = \tilde{\m{K}}_i \: \tilde{\v{T}}_i + \m{Q}_i, \; i = \range{0}{N_s - 1} \label{eq:ce-recurrent} \\
  & \tilde{\v{T}}_0 = \tilde{\v{T}}_{N_s + 1} \nonumber
\end{align}
In order to form the condensed equation mentioned in the beginning of this section, we perform the iterative repetition of \equref{eq:ce-recurrent} that leads us to:
\begin{equation} \label{eq:y-recurrent}
  \tv{T}_i = \prod_{j = 0}^{i - 1} \tm{K}_j \: \tv{T}_0 + \m{W}_{i - 1}, \; i = \range{1}{N_s}
\end{equation}
where $\m{W}_i$ are defined as the following:
\begin{align}
  \m{W}_0 & = \m{Q}_0 \nonumber \\
  %\m{W}_i & = \sum_{l = 1}^i \prod_{j = l}^i \tilde{\m{K}}_j \: \m{Q}_{l - 1} + \m{Q}_i, \: i = \range{1}{N_s - 1} \nonumber \\
  \m{W}_i & = \tilde{\m{K}}_i \: \m{W}_{i - 1} + \m{Q}_i, \; i = \range{1}{N_s - 1} \label{eq:p-recurrent}
\end{align}
Therefore, we can calculate the final vector $\tilde{\v{T}}_{N_s}$ using \equref{eq:y-recurrent} and \equref{eq:p-recurrent}:
\[
  \tilde{\v{T}}_{N_s} = \prod_{j = 0}^{N_s - 1} \tilde{\m{K}}_j \: \tilde{\v{T}}_0 + \m{W}_{N_s - 1}
\]
Taking into account the boundary condition given by \equref{eq:boundary-condition}, we obtain the following system of linear equations:
\begin{equation} \label{eq:core-system}
  (\m{I} - \prod_{j = 0}^{N_s - 1} \tilde{\m{K}}_j) \: \tilde{\v{T}}_0 = \m{W}_{N_s - 1}
\end{equation}
We recall that $\tilde{\m{K}}_i$ is the matrix exponential given by \equref{eq:matrix-exponential}, therefore, the following simplification holds:
\[
  \prod_{j = i}^l \tilde{\m{K}}_j = \prod_{j = i}^l e^{\tilde{\m{A}} \Delta t_j} = e^{\tilde{\m{A}} \sum_{j = i}^l \Delta t_j} = \m{U} e^{\left( \sum_{j = i}^l \Delta t_j \: \m{\Lambda} \right)} \m{U}^T
\]
Consequently:
\[
  \prod_{j = 0}^{N_s - 1} \tilde{\m{K}}_j = \m{U} \: \diag{e^{\period \lambda_0}}{e^{\period \lambda_{N_n - 1}}} \: \m{U}^T
\]
where $\period$ is the application period. Substituting this product into \equref{eq:core-system}, we obtain the following system:
\[
  (\m{I} - \m{U} \: e^{\period \m{\Lambda}} \: \m{U}^T) \: \tilde{\v{T}}_0 = \m{W}_{N_s - 1}
\]
The identity matrix $\m{I}$ can be splitted into $\m{U} \m{U}^T$, hence:
\[
  \tilde{\v{T}}_0 = \m{U} \: (\m{I} - e^{\period \m{\Lambda}})^{-1} \: \m{U}^T \: \m{W}_{N_s - 1} = \m{Z} \: \m{W}_{N_s - 1}
\]
where:
\begin{equation} \label{eq:m-matrix}
  \m{Z} = \m{U} \: \diag{\frac{1}{1 - e^{\period \lambda_0}}}{\frac{1}{1 - e^{\period \lambda_{N_n - 1}}}} \: \m{U}^T
\end{equation}
The equation gives the initial solution vector $\tv{T}_0$, the rest of vectors $\tv{T}_i$ for $i = \range{1}{N_s - 1}$ are successively found from \equref{eq:ce-recurrent}.

Since the power profile is evenly sampled with the sampling interval $\Delta t$, i.e., $\Delta t_i = \Delta t$ for $i = \range{0}{N_s - 1}$, the recurrent process in \equref{eq:recurrent-equation} turns into:
\[
  \tilde{\v{T}}_{i+1} = \tilde{\m{K}} \: \tilde{\v{T}}_i + \tilde{\m{B}} \: \v{P}_i
\]
where:
\[
  \tm{K} = e^{\tm{A} \: \Delta t} \hs \tm{B} = \tm{A}^{-1} ( e^{\tm{A} \: \Delta t} - \m{I} ) \m{C}^{-\frac{1}{2}}
\]
Here $\tm{K}$ and $\tm{B}$ are constants, since they depend only on the matrices $\tm{A}$, $\m{C}$, and sampling interval $\Delta t$, which is fixed. In this case, the block diagonal of the matrix $\tilde{\mathbb{A}}$, similar to \equref{eq:system}, is composed of the same repeating block $\tilde{\m{K}}$ and the recurrent expressions take the following form:
\begin{align}
  & \tv{T}_{i + 1} = \tm{K} \: \tv{T}_i + \m{Q}_i, \; i = \range{0}{N_s - 1} \nonumber \\
  & \m{W}_i = \tm{K} \: \m{W}_{i - 1} + \m{Q}_i, \; i = \range{1}{N_s - 1} \nonumber
\end{align}
The last step of the solution is to return back to temperature by performing the backward substitution opposite to \equref{eq:substitution}:
\[
  \v{T}_i = \m{C}^{-\frac{1}{2}} \: \tv{T}_i, \: i = \range{0}{N_s - 1}
\]

As we see, the auxiliary substitution from \secref{sec:ce-auxiliary} allows us to perform the single-time eigenvalue decomposition with orthogonal eigenvectors (\equref{eq:eigenvalue-decomposition}) that later eases the computational process at several stages. First, the decomposition is employed to compute the matrix exponential, matrix inverse, and, consequently, matrices $\tm{K}$ and $\tm{B}$. Then, the linear system given by \equref{eq:core-system} is solved without any explicit inversion of any matrix, the solution $\tv{T}_0$ is obtained by scalar divisions and a similarity transformation with $\m{U}$ (\equref{eq:m-matrix}).

As it was mentioned previously, the eigenvalue decomposition along with $\tm{K}$ and $\tm{B}$ are computed only once for a particular RC thermal circuit and can be considered as a given for optimization purposes, significantly decreasing the computational time. Moreover, if the subject of optimization is the same time interval, \equref{eq:m-matrix} is a constant as well.
