\image{sparseness-of-system}{0 210 0 210}{Sparseness of the system of linear equations for the SSDTP calculation. Each blue point corresponds to a non-zero element of the matrix of the system.}
The matrix of the system given by \equref{eq:system} has a very specific structure that can be observed in \figref{fig:sparseness-of-system}. Non-zero elements are located only on the block diagonal (composed of $N_n \times N_n$ matrices), one subdiagonal just above the block diagonal, and one subdiagonal in the left bottom corner.

In this section we propose a fast approach to solve the system given by \equref{eq:system}. The approach consists of one preparatorary step aimed to be more efficient in the future calculations and the solution itself.

\subsection{Auxiliary Transformation}
It can be noticed that the analytical solution in \equref{eq:solution} includes the matrix exponential and inverse of the product $\m{\m{C}}^{-1} \: \m{\m{G}}$, which is an arbitrary square matrix. It is preferable to have a symmetric matrix to perform these computations, since for a real symmetric matrix $\m{M}$ the following eigenvalue decomposition holds:
\begin{equation} \label{eq:eigenvalue-decomposition}
  \m{M} = \m{\m{U}} \m{\Lambda} \m{\m{U}}^\v{T}
\end{equation}
where $\m{\m{U}}$ is a square matrix of the eigenvectors of $\m{M}$ and $\m{\Lambda}$ is a diagonal matrix composed of the eigenvalues $\lambda_i$. Once we have obtained such a decomposition, calculation of the matrix exponential and inverse becomes a trivial task:
\begin{align*}
  & e^\m{M} = \m{\m{U}} \; e^{\m{\Lambda}} \; \m{\m{U}}^\v{T} = \m{\m{U}}\: \left[
      \begin{array}{ccc}
        e^{\lambda_0} & \cdots & 0 \\
        \vdots & \ddots & \vdots \\
        0 & \cdots & e^{\lambda_{n - 1}}
      \end{array}
    \right] \; \m{\m{U}}^\v{T} \\
  & \m{M}^{-1} = \m{\m{U}} \: \m{\Lambda}^{-1} \: \m{\m{U}}^\v{T} = \m{\m{U}} \left[
      \begin{array}{ccc}
        \frac{1}{\lambda_0} & \cdots & 0 \\
        \vdots & \ddots & \vdots \\
        0 & \cdots & \frac{1}{\lambda_{n - 1}}
      \end{array}
    \right] \m{\m{U}}^\v{T} \\
\end{align*}
Since the conductance matrix $\m{\m{G}}$ is a symmetric matrix (and the capacitance $\m{\m{C}}$ a diagonal matrix) \cite{rao2007}, we can apply the following substitution:
\begin{align*}
  & \tilde{\v{T}}(t) = \m{C}^{\frac{1}{2}} \v{T}(t) \\
  & \tilde{\m{A}} = -\m{C}^{-\frac{1}{2}} \m{G} \: \m{C}^{-\frac{1}{2}}
\end{align*}
with the result:
\begin{align*}
  & \frac{d\tilde{\v{T}}(t)}{dt} = \tilde{\m{A}} \: \m{Y}(t) + \m{C}^{-\frac{1}{2}} \v{P} \\
  & \tilde{\v{T}}(t) = e^{\tilde{\m{A}} t} \tilde{\v{T}}_0 + \tilde{\m{A}}^{-1} (e^{\tilde{\m{A}} t} - \m{I}) \m{C}^{-\frac{1}{2}} \v{P}
\end{align*}
where $\tilde{\m{A}}$ is a symmetric matrix. Therefore, in case of the matrix exponential we have:
\[
  e^{\tilde{\m{A}} t} = \m{U} \: e^{\m{\Lambda} t} \: \m{U}^T = \m{U} \left[
      \begin{array}{ccc}
        e^{t \lambda_0} & \cdots & 0 \\
        \vdots & \ddots & \vdots \\
        0 & \cdots & e^{t \lambda_{N_n - 1}}
      \end{array}
    \right] \m{U}^T
\]
A similar equation can be obtained for the matrix inverse. It should be noted that the eigenvalue decomposition is performed only once for a particular RC thermal circuit and can be used for all future calculations.

The next step is to update the SSDTP system given in \equref{eq:recurrent-system}:
\begin{align}
  & \tilde{\v{T}}_{i+1} = \tilde{\m{K}}_i \: \tilde{\v{T}}_i + \tilde{\m{B}}_i \: \v{P}_i \label{eq:recurrent-equation} \\
  & \tilde{\m{K}}_i = e^{\tilde{\m{A}} \: \Delta t_i} \nonumber \\
  & \tilde{\m{B}}_i = \tilde{\m{A}}^{-1} \left( e^{\tilde{\m{A}} \Delta t_i} - \m{I} \right) \m{C}^{-\frac{1}{2}} \nonumber
\end{align}
Using the eigenvalue decomposition, the last equation can be computed in the following way:
\begin{align*}
  \tilde{\m{B}}_i & = \m{U} \: \m{\Lambda}^{-1} \: \m{U}^T \left(\m{U} \: e^{\m{\Lambda} \Delta t_i} \: \m{U}^T - \m{U} \: \m{U}^T \right) \m{C}^{-\frac{1}{2}} = \\
      & = \m{U} \left[
        \begin{array}{ccc}
          \frac{e^{\Delta t_i \: \lambda_0} - 1}{\lambda_0} & \cdots & 0 \\
          \vdots & \ddots & \vdots \\
          0 & \cdots & \frac{e^{\Delta t_i \: \lambda_{N_n - 1}} - 1}{\lambda_{N_n - 1}}
        \end{array}
      \right] \m{U}^T \: \m{C}^{-\frac{1}{2}}
\end{align*}

\subsection{Solution with Condensed Equation}
Let us return back to the recurrent system and denote \mbox{$\m{Q}_i = \tilde{\m{B}}_i \: \v{P}_i$}:
\begin{align}
  & \tilde{\v{T}}_{i + 1} = \tilde{\m{K}}_i \: \tilde{\v{T}}_i + \m{Q}_i, \; i = 0 \dots N_s - 1 \label{eq:ce-recurrent} \\
  & \tilde{\v{T}}_0 = \tilde{\v{T}}_{N_s + 1} \nonumber
\end{align}
A common technique to solve such systems is to form a so-called condensed equation (CE), or condensed system \cite{stoer2002}. Let us undertake this procedure step by step.

The iterative repetition of \equref{eq:ce-recurrent} results in:
\begin{equation} \label{eq:y-recurrent}
  \tilde{\v{T}}_i = \prod_{j = 0}^{i - 1} \tilde{\m{K}}_j \: \tilde{\v{T}}_0 + \m{W}_{i - 1}, \; i = 1 \dots N_s
\end{equation}
where $\m{W}_i$ are defined as the following:
\begin{align}
  \m{W}_0 & = \m{Q}_0 \nonumber \\
  \m{W}_i & = \sum_{l = 1}^i \prod_{j = l}^i \tilde{\m{K}}_j \: \m{Q}_{l - 1} + \m{Q}_i, \: i = 1 \dots N_s - 1 \nonumber \\
  \m{W}_i & = \tilde{\m{K}}_i \: \m{W}_{i - 1} + \m{Q}_i, \; i = 1 \dots N_s - 1 \label{eq:p-recurrent}
\end{align}
Therefore, we can calculate the final value $\tilde{\v{T}}_{N_s}$ from \equref{eq:y-recurrent}:
\[
  \tilde{\v{T}}_{N_s} = \prod_{j = 0}^{N_s - 1} \tilde{\m{K}}_j \: \tilde{\v{T}}_0 + \m{W}_{N_s - 1}
\]
Taking into account the boundary condition given by \equref{eq:boundary-condition}, we obtain the following system of linear equations:
\begin{equation} \label{eq:core-system}
  (\m{I} - \prod_{j = 0}^{N_s - 1} \tilde{\m{K}}_j) \: \tilde{\v{T}}_0 = \m{W}_{N_s - 1}
\end{equation}
We recall that $\tilde{\m{K}}_i$ is the matrix exponential, therefore, the following simplification holds:
\begin{align*}
  \prod_{j = i}^l \tilde{\m{K}}_j = \prod_{j = i}^l e^{\tilde{\m{A}} \Delta t_j} & = e^{\tilde{\m{A}} \sum_{j = i}^l \Delta t_j} \\
  & = \m{U} e^{\left( \sum_{j = i}^l \Delta t_j \: \m{\Lambda} \right)} \m{U}^T
\end{align*}
Consequently:
\begin{align*}
  \prod_{j = 0}^{N_s - 1} \tilde{\m{K}}_j & = e^{\tilde{\m{A}} \mathcal{T}} = \m{U} \: e^{\mathcal{T} \m{\Lambda}} \: \m{U}^T \\
    & = \m{U} \left[
      \begin{array}{ccc}
        e^{\mathcal{T} \lambda_0} & \cdots & 0 \\
        \vdots & \ddots & \vdots \\
        0 & \cdots & e^{\mathcal{T} \lambda_{N_n - 1}}
      \end{array}
    \right] \m{U}^T
\end{align*}
where $\mathcal{T}$ is the overall period. Substituting this product into \equref{eq:core-system}, we obtain the following system:
\[
  (\m{I} - \m{U} \: e^{\mathcal{T} \m{\Lambda}} \: \m{U}^T) \: \tilde{\v{T}}_0 = \m{W}_{N_s - 1}
\]
Finally, the identity matrix $\m{I}$ can be splitted into $\m{U} \m{U}^T$, hence:
\begin{align*}
  & \tilde{\v{T}}_0 = \m{U} \: (\m{I} - e^{\mathcal{T} \m{\Lambda}})^{-1} \: \m{U}^T \: \m{W}_{N_s - 1} \\
  & \tilde{\v{T}}_0 = \m{U} \: \left[
      \begin{array}{ccc}
        \frac{1}{1 - e^{\mathcal{T} \lambda_0}} & \cdots & 0 \\
        \vdots & \ddots & \vdots \\
        0 & \cdots & \frac{1}{1 - e^{\mathcal{T} \lambda_{N_n - 1}}}
      \end{array}
    \right] \: \m{U}^T \: \m{W}_{N_s - 1}
\end{align*}
Here $\m{W}_{N_s - 1}$ can be calculated using \equref{eq:p-recurrent}. As we see, there is no need to inverse any matrix, the solution of the system is obtained by scalar divisions and a similarity transformation with $\m{U}$. All other vectors $\tilde{\v{T}}_i$ for \mbox{$i = 1 \dots N_s - 1$} are successively found from \equref{eq:ce-recurrent}.

Now we assume that the power profile $\mathbb{\v{P}}$ is evenly sampled with the sampling interval equal to $\Delta t$, i.e., $\Delta t_i = \Delta t$ for $i = 0 \dots N_s - 1$. Having this assumption, the recurrent process in \equref{eq:recurrent-equation} turns into:
\[
  \tilde{\v{T}}_{i+1} = \tilde{\m{K}} \: \tilde{\v{T}}_i + \tilde{\m{B}} \: \v{P}_i
\]
where:
\begin{align*}
  & \tilde{\m{K}} = e^{\tilde{\m{A}} \: \Delta t} \\
  & \tilde{\m{B}} = \tilde{\m{A}}^{-1} ( e^{\tilde{\m{A}} \: \Delta t} - \m{I} ) \m{C}^{-\frac{1}{2}}
\end{align*}
Here $\tilde{\m{K}}$ and $\tilde{\m{B}}$ are constants, since they depend only on the matrices $\tilde{\m{A}}$, $\m{C}$, and sampling interval $\Delta t$, which is fixed. In this case, the block diagonal of the matrix $\tilde{\mathbb{\m{K}}}$, similar to \equref{eq:system}, is composed of the same repeating block $\tilde{\m{K}}$ and the recurrent expressions take the following form:
\begin{align}
  & \tilde{\v{T}}_{i + 1} = \tilde{\m{K}} \: \tilde{\v{T}}_i + \m{Q}_i, \; i = 0 \dots N_s - 1 \nonumber \\
  & \m{W}_i = \tilde{\m{K}} \: \m{W}_{i - 1} + \m{Q}_i, \; i = 1 \dots N_s - 1 \nonumber
\end{align}
