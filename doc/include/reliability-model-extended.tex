\balance
In our analysis, we use the reliability model presented in \cite{huang2009, xiang2010}. The model is based on the assumption that the time to failure $\mathcal{T}$ has a Weibull distribution, i.e., $\mathcal{T} \sim Weibull(\eta, \beta)$ where $\eta$ and $\beta$ are the scaling and shape parameters, respectively. The expectation of the distribution is the following:
\begin{equation} \label{eq:general-mttf}
  \expectation{\mathcal{T}} = \eta \; \Gamma(1 + \frac{1}{\beta})
\end{equation}
where $\Gamma$ is the gamma function. $\expectation{\mathcal{T}}$ is the mean time to failure (MTTF) that we denote by $\theta$.

The shape parameter $\beta$ is independent on the temperature variation \cite{chang2006}, which, however, is not the case with the scaling parameter $\eta$. Therefore, the distribution varies with the temperature. We can split the overall period of the application $\period$ into $N_m$ time intervals $\Delta t_i$, so that during each time interval $\Delta t_i$ the corresponding $\eta_i$ is a constant:
\begin{equation} \label{eq:eta-one}
  \eta_i = \frac{\theta_i}{\Gamma(1 + \frac{1}{\beta})}
\end{equation}
where $\theta_i$ is the MTTF in the $i$th time interval as if we had the failure distribution of this interval all the time. For now the values $\theta_i$ are unknown and depend on the particular failure mechanism. As it is shown in \cite{xiang2010}, the reliability function $R(t)$, i.e., the probability of survival until an arbitrary time $t \geq 0$, can be approximated as the following:
\[
  R(t) = e^{-(\frac{t}{\period} \sum_{i=0}^{N_m - 1} \frac{\Delta t_i}{\eta_i})^\beta}
\]
The formula keeps the form of the Weibull distribution with the scaling parameter equal to:
\begin{equation} \label{eq:eta-many}
  \eta = \frac{\period}{\sum_{i=0}^{N_m - 1} \frac{\Delta t_i}{\eta_i}}
\end{equation}
The MTTF with respect to the whole application period can be obtained by combining \equref{eq:general-mttf}, \equref{eq:eta-one}, and \equref{eq:eta-many}.

As mentioned previously, in order to compute the MTTF, we need to consider the particular failure mechanism and determine the values $\theta_i$ needed in \equref{eq:eta-one}. We focus on the thermal cycling fatigue (\secref{sec:reliability-model}). Assuming this concrete failure model the duration $\Delta t_i$, during which the corresponding scaling parameter $\eta_i$ is constant \equref{eq:eta-one}, is exactly a thermal cycle.

When the system is exposed to identical thermal cycles, the number of such cycles to failure can be estimated using a modified version of the well-known Coffin-Manson equation with the Arrhenius term \cite{xiang2010, jedec2010}:
\[
  N_c = A (\Delta T - \Delta T_0)^{-b} e^{\frac{E_a}{k T_{max}}}
\]
where $A$ is an empirically determined constant, $\Delta T$ is the thermal cycle excursion, $\Delta T_0$ is the portion of the temperature range in the elastic region which does not cause damage, $b$ is the Coffin-Manson exponent, which is also empirically determined, $E_{a}$ is the activation energy, $k$ is the Boltzmann constant, and $T_{max}$ is the maximal temperature during the thermal cycle. The system undergoes a number of different thermal cycles each with its own duration $\Delta t_i$ over the application period and each cycle causes its own damage. Therefore, having $N_m$ thermal cycles characterized by the number of cycles to failure $N_{c\:i}$ and duration $\Delta t_i$, we can compute $\theta_i$:
\begin{equation} \label{eq:mttf-cycle}
  \theta_i = N_{c \: i} \; \Delta t_i
\end{equation}
Taking equations \eqref{eq:general-mttf}, \eqref{eq:eta-one}, \eqref{eq:eta-many}, and \eqref{eq:mttf-cycle} together, we obtain the following expression to estimate the MTTF of one component in the system:
\begin{align}
  \theta = \frac{\period}{\sum_{i=0}^{N_m - 1} \frac{1}{N_{c \: i}}}
\end{align}
In order to identify thermal cycles in the temperature curve, we follow the approach given in \cite{xiang2010} where the rainflow counting method is employed.
