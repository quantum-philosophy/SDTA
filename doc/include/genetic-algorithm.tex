The optimization procedure is based on a genetic algorithm \cite{schmitz2004} with the fitness function $\mttf$ given by \equref{eq:fitness-function}. Each chromosome is a vector of $2 \times N_t$ elements, where the first half encodes priorities of the tasks and the second represents a mapping. The population contains $4 \times N_t$ individuals that are initialized partially randomly and partially based on the initial temperature-aware solution \cite{xie2006}. In each generation, a number of individuals, called parents, are chosen for breeding by the tournament selection with the number of competitors proportional to the population size. The parents undergo the 2-point crossover with $0.8$ probability and uniform mutation with $0.01$ probability. The evolution mechanism follows the elitism model where the best individual always survives. The stopping condition is an absence of improvement within 200 successive generations.

The fitness of a chromosome, \equref{eq:fitness-function}, is evaluated in a number of steps. First, the decoded priorities and mapping are given to a list scheduler that produces schedules for each of the cores. If the application schedule does not satisfy the deadline, the solution is penalized proportionally to the delay and is not further evaluated; otherwise, based on the parameters of the architecture and tasks, a power profile is obtained and the corresponding SSDTP is computed by our proposed method. If the SSDTP violates the temperature constraint given by \equref{eq:t-max}, the solution is penalized proportionally to the amount of violation and not further processed; otherwise, the MTTF of each core is estimated according to \equref{eq:one-mttf} and the fitness function $\mttf$ is computed.
