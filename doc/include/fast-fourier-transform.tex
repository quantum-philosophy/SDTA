One can notice that the overall matrix of the system under the assumption of the equal time intervals becomes a block Toeplitz matrix, because inner blocks $\mathbb{A}(i, \: j)$ satisfy the following criterion:
\[
  \mathbb{A}(i, j) = \mathbb{A}(i+1, \: j+1), \; i, j = 0 \dots N_s - 2
\]

To be more specific, the matrix is a block-circulant matrix where each block row vector is rotated one block element to the right relative to the preceding block row vector. This leads us to a wide range of possible techniques to solve \mbox{$\mathbb{A} \: \mathbb{Y} = \mathbb{B}$}, for example, the Fast Fourier Transform (FFT) \cite{mazancourt1983}, \cite{vescovo1997}.

In spite of the fact that the FFT approach is \emph{much faster} then the solution obtained with the UMF, our experiments have shown that the condensed equation method is even faster (see \secref{sec:results-ssdtp}), therefore, we concentrate on it and discuss the FFT in brief as a possible alternative.

$\mathbb{A}$ has $N_s \times N_s$ blocks, each block is a $N_n \times N_n$ submatrix. Since the matrix is a block-circulant matrix, it can be represented with only $N_s$ blocks that form the top block row:
\[
  \mathbb{A}(j), \; j = 0 \dots N_s - 1
\]
and all other rows can be easily found shifting this one. To solve the system, we need to apply the Discrete Fourier Transform to these $N_s$ blocks:
\[
  \mathbb{A}(k)^f = \sum_{j = 0}^{N_s - 1} \mathbb{A}(j) \; \omega_{N_s}^{jk}, \; k = 0 \dots N_s - 1
\]
where $\omega_{N_s} = e^{\frac{-2 \pi i}{N_s}}$. Here we perform a bulk transform of all $N_n \times N_n$ vectors at once, whereas the vector-by-vector version is the following for each $n$ and $m = 0 \dots N_n - 1$:
\[
  \mathbb{A}(k)^f_{nm} = \sum_{j = 0}^{N_s - 1} \mathbb{A}(j)_{nm} \; \omega_{N_s}^{jk}, \; k = 0 \dots N_s - 1
\]
Note that in our case only two matrices are non-zero, therefore, this procedure can be shrunk. By applying the transformation, we come from the time domain to the frequency domain. Also we need to perform the same operation on the right-hand vector $\mathbb{B}$ splitted into $N_s$ chunks, denoted $\mathbb{B}(j)$, of $N_n$ successive elements:
\[
  \mathbb{B}(k)^f = \sum_{j = 0}^{N_s - 1} \mathbb{B}(j) \; \omega_{N_s}^{jk}, \; k = 0 \dots N_s - 1
\]

The next step is to solve $N_s$ systems with matrices $(\mathbb{A}(k)^f)^{\ast}$ and corresponding vectors $\mathbb{B}(k)^f$, the asterisk here denotes the complex conjugate:
\[
  (\mathbb{A}(k)^f)^{\ast} \; \mathbb{Y}(k)^f = \mathbb{B}(k)^f, \; k = 0 \dots N_s - 1
\]
where all matrices $\mathbb{A}(k)^f$ ($N_n \times N_n$ matrices) are symmetric, since $\mathbb{A}(k)$ are so, therefore, the eigenvalue decomposition can significantly simplify the solution process.

The last step is to return back to the time domain with the Inverse Discrete Fourier Transform:
\[
  \mathbb{Y}(k) = \frac{1}{N_s} \sum_{j = 0}^{N_s - 1} \mathbb{Y}(j)^f \; \omega_{N_s}^{-jk}, \; k = 0 \dots N_s - 1
\]
