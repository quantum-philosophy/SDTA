As it was shown in \secref{sec:hotspot-solution}, the state of the art solutions either produce inaccurate and, in extreme cases, completely useless results, or they are unacceplably slow. In this section we eliminate the first problem by obtaining an analytical expression of the SSDTP and tackle the second one in \secref{sec:condensed-equation} where a fast technique to perform the computations is proposed.

In the onwards explanation, without loss of generality, we assume $\v{T}(t) \equiv \v{T}(t) - \v{T}_{amb}$. Let the power consumption vector $\v{P}(t)$ be constant and equal to $\v{P}$, then the system given by \equref{eq:fourier-model} is a system of ordinary differential equations (ODE) with the following solution:
\begin{equation} \label{eq:solution}
  \v{T}(t) = e^{\m{A} t} \; \v{T}_0 + \m{A}^{-1}(e^{\m{A} t} - \m{I})\m{C}^{-1} \v{P}
\end{equation}
where $\m{A} = -\m{C}^{-1} \: \m{G}$ and $\v{T}_0$ is the initial temperature. Therefore, given a discrete power profile, the corresponding temperature profile can be found using the following recurrence:
\begin{equation} \label{eq:recurrent-system}
  \v{T}_{i+1} = \m{K}_i \: \v{T}_i + \m{B}_i \: \v{P}_i
\end{equation}
where:
\begin{align*}
  & \m{K}_i = e^{\m{A} \Delta t_i} \\
  & \m{B}_i = \m{A}^{-1}(e^{\m{A} \Delta t_i} - \m{I})\m{C}^{-1}
\end{align*}

\subsection{Transient Temperature Analysis}
Given the initial temperature $\v{T}_0$, the recurrence in \equref{eq:recurrent-system} can be applied to perform the TTA. Our experiments show that in the case of equal time intervals, where $\m{K}_i$ and $\m{B}_i$ are constants, this approach demonstrates a significant performance improvement relative to iterative solutions of ODEs, e.g., the fourth-order Runge-Kutta method implemented in HotSpot. The same observation is made in \cite{thiele2011}.

The TTA using the analytical technique given in \equref{eq:recurrent-system} can be employed to approximate the SSDTP as it is described in \secref{sec:hotspot-iterative-solution} dramatically speeding up the computational process. However, the number of required simulations is unchanged since it is basically an alternative way to perform the same type of analysis. The technique that we shall propose later on equals to only two simulations using the above-mentioned analytical approach while the produced solutions are exact.

\subsection{Steady-State Dynamic Temperature Analysis}
For the steady-state case the following system of linear equations can be derived from \equref{eq:recurrent-system}:
\[
  \begin{cases}
    \m{K}_0 \: \v{T}_0 - \v{T}_1 & = -\m{B}_0 \: \v{P}_0 \\
    ... \\
    -\v{T}_0 + \m{K}_{N_s - 1} \: \v{T}_{N_s - 1} & = -\m{B}_{N_s - 1} \: \v{P}_{N_s - 1}
  \end{cases}
\]
where the last equation respects the boundary condition and ensures the equality of temperature values on both ends of the overall period:
\begin{equation} \label{eq:boundary-condition}
  \v{T}_0 = \v{T}_{N_s}
\end{equation}
To get the whole picture, the system can be written as:
\begin{align}
  & \mathbb{K} \: \mathbb{X} = \mathbb{B} \label{eq:system} \\
  & \mathbb{K} = \left[
    \begin{array}{ccccc}
      \m{K}_0 & -\m{I} & 0 & \cdots & 0 \\
      0 & \m{K}_1 & -\m{I} &  & \vdots \\
      \vdots &  & \ddots & -\m{I} & 0 \\
      0 &  &  & \m{K}_{N_s - 2} & -\m{I} \\
      -\m{I} & 0 & \cdots & 0 & \m{K}_{N_s - 1}
    \end{array}
  \right] \nonumber \\
  & \mathbb{X} = \left[
    \begin{array}{c}
      \v{T}_0 \\
      \vdots \\
      \v{T}_{N_s - 1}
    \end{array}
  \right] \nonumber \\
  & \mathbb{B} = \left[
    \begin{array}{c}
      -\m{B}_0 \: \v{P}_0 \\
      \vdots \\
      -\m{B}_{N_s - 1} \: \v{P}_{N_s - 1}
    \end{array}
  \right] \nonumber
\end{align}
where $\mathbb{K}$ is a $N_n N_s \times N_n N_s$ matrix, $\mathbb{X}$ and $\mathbb{B}$ are vectors with $N_n N_s$ elements.

It can be seen that we have obtained a regular system of liner equations with a specific structure. Now we shall briefly discuss possible alternatives to solve it.

\subsubsection{Direct Dense Solutions}
The first straight-forward way to solve the system is to use dense solvers such as the LU decomposition. The problem here is that such systems could be extremely large, especially when we want to achieve a higher level of accuracy and, therefore, the power profile contains a lot of steps $N_s$. Each new step produces $N_n$ new equations in the system given by \equref{eq:system}. The complexity grows very rapidly with the number of processing elements $N_p$. For instanse, the simpliest model implemented in HotSpot uses the following relation:
\[
  N_n = 4 \times N_p + 12
\]
Therefore, each new processing element increases each matrix $K_i$ by 4 rows and 4 columns, and each vector $Y_i$ and $Q_i$ by 4 elements. As an example, if the power profile for a single-processor system is composed of 1000 steps, then having the same discretization but with one additional core results in a linear system with 4000 additional equations. All in all, a fast and accurate approach to solve \equref{eq:system} is required.


\subsubsection{Direct Sparse Solutions}
\image{sparseness-of-system}{80 210 80 210}{Sparseness of the system of linear equations for the SSDTP calculation. Each blue point corresponds to a non-zero element of the matrix of the system.}
One may notice that the matrix $\mathbb{A}$ is an extremely sparse matrix with a very specific structure that can be observed in \figref{fig:sparseness-of-system}. The matrix has non-zero elements only on its block diagonal (composed of $N_n \times N_n$ matrices), one subdiagonal just above the block diagonal, and one subdiagonal in the left bottom corner. Therefore, instead of the dense LU decomposition we can apply algorithms that are specially designed for such cases.
In our experiments we use the UMFPACK library, a set of routines for solving unsymmetric sparse linear systems based on the Unsymmetric MultiFrontal method (UMF) \cite{umfpack2004}.


\subsubsection{Solutions for Toeplitz and Block-Circulant Systems}
One can notice that the overall matrix of the system under the assumption of the equal time intervals becomes a block Toeplitz matrix, because inner blocks $\mathbb{A}(i, \: j)$ satisfy the following criterion:
\[
  \mathbb{A}(i, j) = \mathbb{A}(i+1, \: j+1), \; i, j = 0 \dots N_s - 2
\]

To be more specific, the matrix is a block-circulant matrix where each block row vector is rotated one block element to the right relative to the preceding block row vector. This leads us to a wide range of possible techniques to solve \mbox{$\mathbb{A} \: \mathbb{Y} = \mathbb{B}$}, for example, the Fast Fourier Transform (FFT) \cite{mazancourt1983}, \cite{vescovo1997}.

In spite of the fact that the FFT approach is \emph{much faster} then the solution obtained with the UMF, our experiments have shown that the condensed equation method is even faster (see \secref{sec:results-ssdtp}), therefore, we concentrate on it and discuss the FFT in brief as a possible alternative.

$\mathbb{A}$ has $N_s \times N_s$ blocks, each block is a $N_n \times N_n$ submatrix. Since the matrix is a block-circulant matrix, it can be represented with only $N_s$ blocks that form the top block row:
\[
  \mathbb{A}(j), \; j = 0 \dots N_s - 1
\]
and all other rows can be easily found shifting this one. To solve the system, we need to apply the Discrete Fourier Transform to these $N_s$ blocks:
\[
  \mathbb{A}(k)^f = \sum_{j = 0}^{N_s - 1} \mathbb{A}(j) \; \omega_{N_s}^{jk}, \; k = 0 \dots N_s - 1
\]
where $\omega_{N_s} = e^{\frac{-2 \pi i}{N_s}}$. Here we perform a bulk transform of all $N_n \times N_n$ vectors at once, whereas the vector-by-vector version is the following for each $n$ and $m = 0 \dots N_n - 1$:
\[
  \mathbb{A}(k)^f_{nm} = \sum_{j = 0}^{N_s - 1} \mathbb{A}(j)_{nm} \; \omega_{N_s}^{jk}, \; k = 0 \dots N_s - 1
\]
Note that in our case only two matrices are non-zero, therefore, this procedure can be shrunk. By applying the transformation, we come from the time domain to the frequency domain. Also we need to perform the same operation on the right-hand vector $\mathbb{B}$ splitted into $N_s$ chunks, denoted $\mathbb{B}(j)$, of $N_n$ successive elements:
\[
  \mathbb{B}(k)^f = \sum_{j = 0}^{N_s - 1} \mathbb{B}(j) \; \omega_{N_s}^{jk}, \; k = 0 \dots N_s - 1
\]

The next step is to solve $N_s$ systems with matrices $(\mathbb{A}(k)^f)^{\ast}$ and corresponding vectors $\mathbb{B}(k)^f$, the asterisk here denotes the complex conjugate:
\[
  (\mathbb{A}(k)^f)^{\ast} \; \mathbb{Y}(k)^f = \mathbb{B}(k)^f, \; k = 0 \dots N_s - 1
\]
where all matrices $\mathbb{A}(k)^f$ ($N_n \times N_n$ matrices) are symmetric, since $\mathbb{A}(k)$ are so, therefore, the eigenvalue decomposition can significantly simplify the solution process.

The last step is to return back to the time domain with the Inverse Discrete Fourier Transform:
\[
  \mathbb{Y}(k) = \frac{1}{N_s} \sum_{j = 0}^{N_s - 1} \mathbb{Y}(j)^f \; \omega_{N_s}^{-jk}, \; k = 0 \dots N_s - 1
\]


\subsubsection{Iterative Methods for Systems of Linear Equations}
Another possible technique is iterative methods for solving systems of linear equations (e.g., the Jacobi, Gauss–Seidel, Successive over-relaxation methods). These methods are designed to overcome problems of direct solvers, since they do not operate on full matrices and, therefore, consume less memory. Consequently, they can be applied for extremely large systems, but the most important issues with these methods are their convergence and accuracy. In our analysis we did not observe any advantages of using this methods for this particular problem, they demonstrated slow convergence and poor accuracy. Hence, we excude them from the paper.

