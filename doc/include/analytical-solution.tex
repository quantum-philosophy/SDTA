As shown in \secref{sec:hotspot-solution}, the state of the art solutions either produce inaccurate and, in many cases, completely useless results, or they are unacceptably slow. In this section we eliminate the first problem by obtaining an analytical solution for the SSDTP and tackle the second one in \secref{sec:condensed-equation} where a fast solution technique is proposed.

In the following explanation, without loss of generality, we assume $\v{T}(t) \equiv \v{T}(t) - \v{T}_{amb}$. Let the power consumption vector $\v{P}(t)$ be constant and equal to $\v{P}$; then the system given by \equref{eq:fourier-model} is a system of ordinary differential equations (ODE) with the following solution:
\begin{equation} \label{eq:solution}
  \v{T}(t) = e^{\m{A} t} \; \v{T}_0 + \m{A}^{-1} (e^{\m{A} t} - \m{I}) \; \m{C}^{-1} \v{P}
\end{equation}
where $\m{A} = -\m{C}^{-1} \: \m{G}$, $\v{T}_0$ is the initial temperature, and $\m{I}$ is the identity matrix. Therefore, given a discrete power profile, the corresponding temperature profile can be found using the following recurrence:
\begin{equation} \label{eq:recurrent-system}
  \v{T}_{i+1} = \m{K}_i \: \v{T}_i + \m{B}_i \: \v{P}_i
\end{equation}
where $\m{K}_i = e^{\m{A} \Delta t_i}$ and $\m{B}_i = \m{A}^{-1}(e^{\m{A} \Delta t_i} - \m{I})\m{C}^{-1}$.

The approach can be used to perform the TTA as it is discussed in \appref{ap:tta-analytical}. For the SSDTA case the following system of linear equations can be derived from \equref{eq:recurrent-system}:
\[
  \begin{cases}
    \m{K}_0 \: \v{T}_0 - \v{T}_1 & = -\m{B}_0 \: \v{P}_0 \\
    ... \\
    -\v{T}_0 + \m{K}_{N_s - 1} \: \v{T}_{N_s - 1} & = -\m{B}_{N_s - 1} \: \v{P}_{N_s - 1}
  \end{cases}
\]
where the last equation enforces the boundary condition and ensures the equality of temperature values on both ends of the period:
\begin{equation} \label{eq:boundary-condition}
  \v{T}_0 = \v{T}_{N_s}
\end{equation}
To get the whole picture, the system can be written as:
\begin{equation} \label{eq:system}
\resizebox{0.9\linewidth}{!}{
  $\underbrace{\left[
    \begin{array}{ccccc}
      \m{K}_0 & -\m{I} & 0 & \cdots & 0 \\
      0 & \m{K}_1 & -\m{I} &  & \vdots \\
      \vdots &  & \ddots & -\m{I} & 0 \\
      0 &  &  & \m{K}_{N_s - 2} & -\m{I} \\
      -\m{I} & 0 & \cdots & 0 & \m{K}_{N_s - 1}
    \end{array}
  \right]}_{\displaystyle \mathbb{A}} \underbrace{\left[
    \begin{array}{c}
      \v{T}_0 \\
      \\
      \vdots \\
      \\
      \v{T}_{N_s - 1}
    \end{array}
  \right]}_{\displaystyle \mathbb{X}} = \underbrace{\left[
    \begin{array}{c}
      -\m{B}_0 \: \v{P}_0 \\
      \\
      \vdots \\
      \\
      -\m{B}_{N_s - 1} \: \v{P}_{N_s - 1}
    \end{array}
  \right]}_{\displaystyle \mathbb{B}}$
}
\end{equation}
where $\mathbb{A}$ is a $N_n N_s \times N_n N_s$ matrix, $\mathbb{X}$ and $\mathbb{B}$ are vectors with $N_n N_s$ elements. It can be seen that we have obtained a regular system of linear equations. Straight-forward techniques to solve it and their disadvantages are further discussed in \appref{ap:straight-forward}.
