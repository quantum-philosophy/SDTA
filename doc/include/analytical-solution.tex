As shown in \secref{sec:hotspot-solution}, the state of the art solutions either produce inaccurate and, in many cases, completely useless results, or they are unacceplably slow. In this section we eliminate the first problem by obtaining an analytical solution for the SSDTP and tackle the second one in \secref{sec:condensed-equation} where a fast technique to perform the computations is proposed.

In the following explanation, without loss of generality, we assume $\v{T}(t) \equiv \v{T}(t) - \v{T}_{amb}$. Let the power consumption vector $\v{P}(t)$ be constant and equal to $\v{P}$, then the system given by \equref{eq:fourier-model} is a system of ordinary differential equations (ODE) with the following solution:
\begin{equation} \label{eq:solution}
  \v{T}(t) = e^{\m{A} t} \; \v{T}_0 + \m{A}^{-1}(e^{\m{A} t} - \m{I})\m{C}^{-1} \v{P}
\end{equation}
where $\m{A} = -\m{C}^{-1} \: \m{G}$ and $\v{T}_0$ is the initial temperature. Therefore, given a discrete power profile, the corresponding temperature profile can be found using the following recurrence:
\begin{equation} \label{eq:recurrent-system}
  \v{T}_{i+1} = \m{K}_i \: \v{T}_i + \m{B}_i \: \v{P}_i
\end{equation}
where:
\begin{align*}
  & \m{K}_i = e^{\m{A} \Delta t_i} \\
  & \m{B}_i = \m{A}^{-1}(e^{\m{A} \Delta t_i} - \m{I})\m{C}^{-1}
\end{align*}

\subsection{Transient Temperature Analysis (TTA)} \label{sec:tta-analytical}
Given the initial temperature $\v{T}_0$, the recurrence in \equref{eq:recurrent-system} can be applied to perform the TTA. Our experiments show that when time intervals $\Delta t_i$ have the same length, i.e., $\Delta t_i = \Delta t, \forall i$, and matrices $\m{K}_i$ and $\m{B}_i$ become constant, this approach demonstrates a significant performance improvement relative to iterative solutions of ODEs, e.g., the fourth-order Runge-Kutta method implemented in HotSpot. The same observation is made in \cite{thiele2011}.

The TTA using the analytical technique given in \equref{eq:recurrent-system} can be employed to approximate the SSDTP by applying it over successive application periods, as shown in \secref{sec:hotspot-iterative-solution}. Since each iteration of the TTA with this approach is much faster than with HotSpot, it will significantly speed up the SSDTP calculation. However, the number of required iterations is similar to HotSpot (see \figref{fig:hotspot-error}) keeping the computational process relatively slow.

\subsection{Steady-State Dynamic Temperature Analysis (SSDTA)}
For the steady-state case the following system of linear equations can be derived from \equref{eq:recurrent-system}:
\[
  \begin{cases}
    \m{K}_0 \: \v{T}_0 - \v{T}_1 & = -\m{B}_0 \: \v{P}_0 \\
    ... \\
    -\v{T}_0 + \m{K}_{N_s - 1} \: \v{T}_{N_s - 1} & = -\m{B}_{N_s - 1} \: \v{P}_{N_s - 1}
  \end{cases}
\]
where the last equation enforces the boundary condition and ensures the equality of temperature values on both ends of the period:
\begin{equation} \label{eq:boundary-condition}
  \v{T}_0 = \v{T}_{N_s}
\end{equation}
To get the whole picture, the system can be written as:
\begin{align}
  & \mathbb{K} \: \mathbb{X} = \mathbb{B} \label{eq:system} \\
  & \mathbb{K} = \left[
    \begin{array}{ccccc}
      \m{K}_0 & -\m{I} & 0 & \cdots & 0 \\
      0 & \m{K}_1 & -\m{I} &  & \vdots \\
      \vdots &  & \ddots & -\m{I} & 0 \\
      0 &  &  & \m{K}_{N_s - 2} & -\m{I} \\
      -\m{I} & 0 & \cdots & 0 & \m{K}_{N_s - 1}
    \end{array}
  \right] \nonumber \\
  & \mathbb{X} = \left[
    \begin{array}{c}
      \v{T}_0 \\
      \vdots \\
      \v{T}_{N_s - 1}
    \end{array}
  \right] \nonumber \\
  & \mathbb{B} = \left[
    \begin{array}{c}
      -\m{B}_0 \: \v{P}_0 \\
      \vdots \\
      -\m{B}_{N_s - 1} \: \v{P}_{N_s - 1}
    \end{array}
  \right] \nonumber
\end{align}
where $\mathbb{K}$ is a $N_n N_s \times N_n N_s$ matrix, $\mathbb{X}$ and $\mathbb{B}$ are vectors with $N_n N_s$ elements.

It can be seen that we have obtained a regular system of linear equations with a specific structure. Now we shall briefly discuss possible techniques to solve it.

\subsubsection{Direct Dense and Sparse Solvers}
The first straight-forward way to solve the system is to use dense solvers such as the LU decomposition \cite{press2007}. However, a more advanced approach is to employ sparse solvers since the matrix of the system given in \equref{eq:system} is a sparse matrix. Therefore, algorithms specially designed for such cases are preferable, e.g., the Unsymmetric MultiFrontal method \cite{umfpack2004}. The problem here is that the systems to solve could be extremely large, especially when a high accuracy is required. In this case, a fine-graned power profile is obtained with an increased number of steps and/or a large number of thermal nodes corresponding to a high level of details in the equivalent RC thermal circuit. Our experiments have shown that direct solvers are extremely slow and consume a large amount of memory. Therefore, we will not further consider them in the paper.


\subsubsection{Solutions for Toeplitz and Block-Circulant Systems} \label{sec:fast-fourier-transform}
One can notice that the overall matrix of the system under the assumption of the equal time intervals becomes a block Toeplitz matrix, because inner blocks $\mathbb{A}(i, \: j)$ satisfy the following criterion:
\[
  \mathbb{A}(i, j) = \mathbb{A}(i+1, \: j+1), \; i, j = 0 \dots N_s - 2
\]

To be more specific, the matrix is a block-circulant matrix where each block row vector is rotated one block element to the right relative to the preceding block row vector. This leads us to a wide range of possible techniques to solve \mbox{$\mathbb{A} \: \mathbb{Y} = \mathbb{B}$}, for example, the Fast Fourier Transform (FFT) \cite{mazancourt1983}, \cite{vescovo1997}.

In spite of the fact that the FFT approach is \emph{much faster} then the solution obtained with the UMF, our experiments have shown that the condensed equation method is even faster (see \secref{sec:results-ssdtp}), therefore, we concentrate on it and discuss the FFT in brief as a possible alternative.

$\mathbb{A}$ has $N_s \times N_s$ blocks, each block is a $N_n \times N_n$ submatrix. Since the matrix is a block-circulant matrix, it can be represented with only $N_s$ blocks that form the top block row:
\[
  \mathbb{A}(j), \; j = 0 \dots N_s - 1
\]
and all other rows can be easily found shifting this one. To solve the system, we need to apply the Discrete Fourier Transform to these $N_s$ blocks:
\[
  \mathbb{A}(k)^f = \sum_{j = 0}^{N_s - 1} \mathbb{A}(j) \; \omega_{N_s}^{jk}, \; k = 0 \dots N_s - 1
\]
where $\omega_{N_s} = e^{\frac{-2 \pi i}{N_s}}$. Here we perform a bulk transform of all $N_n \times N_n$ vectors at once, whereas the vector-by-vector version is the following for each $n$ and $m = 0 \dots N_n - 1$:
\[
  \mathbb{A}(k)^f_{nm} = \sum_{j = 0}^{N_s - 1} \mathbb{A}(j)_{nm} \; \omega_{N_s}^{jk}, \; k = 0 \dots N_s - 1
\]
Note that in our case only two matrices are non-zero, therefore, this procedure can be shrunk. By applying the transformation, we come from the time domain to the frequency domain. Also we need to perform the same operation on the right-hand vector $\mathbb{B}$ splitted into $N_s$ chunks, denoted $\mathbb{B}(j)$, of $N_n$ successive elements:
\[
  \mathbb{B}(k)^f = \sum_{j = 0}^{N_s - 1} \mathbb{B}(j) \; \omega_{N_s}^{jk}, \; k = 0 \dots N_s - 1
\]

The next step is to solve $N_s$ systems with matrices $(\mathbb{A}(k)^f)^{\ast}$ and corresponding vectors $\mathbb{B}(k)^f$, the asterisk here denotes the complex conjugate:
\[
  (\mathbb{A}(k)^f)^{\ast} \; \mathbb{Y}(k)^f = \mathbb{B}(k)^f, \; k = 0 \dots N_s - 1
\]
where all matrices $\mathbb{A}(k)^f$ ($N_n \times N_n$ matrices) are symmetric, since $\mathbb{A}(k)$ are so, therefore, the eigenvalue decomposition can significantly simplify the solution process.

The last step is to return back to the time domain with the Inverse Discrete Fourier Transform:
\[
  \mathbb{Y}(k) = \frac{1}{N_s} \sum_{j = 0}^{N_s - 1} \mathbb{Y}(j)^f \; \omega_{N_s}^{-jk}, \; k = 0 \dots N_s - 1
\]


\subsubsection{Iterative Methods for Systems of Linear Equations}
Another possible technique is iterative methods for solving systems of linear equations (e.g., the Jacobi, Gauss–Seidel, Successive over-relaxation methods). These methods are designed to overcome problems of direct solvers, since they do not operate on full matrices and, therefore, consume less memory. Consequently, they can be applied for extremely large systems, but the most important issues with these methods are their convergence and accuracy. In our analysis we did not observe any advantages of using this methods for this particular problem, they demonstrated slow convergence and poor accuracy. Hence, we excude them from the paper.

