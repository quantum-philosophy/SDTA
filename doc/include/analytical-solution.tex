In order to solve the problem, we use an analytical approach. The direct solution of \equref{eq:initial} is given by the following equation:
\begin{equation} \label{eq:solution}
  T(t) = e^{C^{-1}A t} \; T_0 + (C^{-1} A)^{-1}(e^{C^{-1}A t} - I)C^{-1} B
\end{equation}

The solution provides us with the transient temperature and holds only when the power vector $B$ is constant. If it is not the case, we need to simulate shorter time intervals where this assumption can take place. Before going to the steady-state case, we perform one important adjustment to the system in order to be more efficient in our future calculations. According to \equref{eq:solution}, we need to compute the matrix exponential of the matrix $C^{-1} A t$. It would be much easier to accomplish if the matrix were symmetric, because a real symmetric matrix is \emph{diagonalizable} and has \emph{independent} (orthogonal) real eigenvectors:
\begin{equation} \label{eq:eigenvalue-decomposition}
  M = U \Lambda U^T
\end{equation}
where $M$ is a real symmetric matrix, $U$ is a square matrix of the eigenvectors of $M$, $\Lambda$ is a diagonal matrix composed of the eigenvalues of $M$ ($\lambda_i$), the equation itself is called the eigenvalue decomposition. Once we have such a decomposition, calculating the matrix exponential becomes a trivial task:
\begin{align}
  & e^M = e^{U \Lambda U^T} = U \: e^{\Lambda} \: U^T \nonumber \\
  & e^{\Lambda} = \left[
      \begin{array}{ccc}
        e^{\lambda_0} & \cdots & 0 \\
        \vdots & \ddots & \vdots \\
        0 & \cdots & e^{\lambda_{n - 1}}
      \end{array}
    \right] \nonumber
\end{align}

Hence, instead of $C^{-1} A$ in front of the variable vector we want to have a symmetry matrix. In order to achieve this, we perform the following substitution:
\begin{align*}
  Y & = C^{\frac{1}{2}} T \\
  D & = C^{-\frac{1}{2}} A \: C^{-\frac{1}{2}} \\
  E & = C^{-\frac{1}{2}} B
\end{align*}
with the result:
\begin{align}
  \frac{dY}{dt} & = D \: Y + E \nonumber \\
  Y(t) & = e^{D t} Y_0 + D^{-1} (e^{D t} - I) E \label{eq:modified-solution} \\
  T(t) & = C^{-\frac{1}{2}} Y(t) \label{eq:finalization}
\end{align}

In this case, $D$ is a symmetric matrix, therefore, it will be easier to find the matrix exponential of $D \: t$ using the above-mentioned eigenvalue decomposition (\equref{eq:eigenvalue-decomposition}):
\[
  e^{D t} = U \: e^{\Lambda t} \: U^T = U \left[
      \begin{array}{ccc}
        e^{t \lambda_0} & \cdots & 0 \\
        \vdots & \ddots & \vdots \\
        0 & \cdots & e^{t \lambda_{N_n - 1}}
      \end{array}
    \right] U^T
\]

Now we shift our focus at the power profile $B$ and come closer to the SSDTP. Each row of $B$ corresponds to a particular time interval $\triangle t_i$ and represents the power consumption $B_i$ during this interval of all processing elements. Each step $i = 0 \dots N_s - 1$ of the iterative process we have a pair $(\triangle t_i, B_i)$ which gives us a temperature vector $T_i$ according to \equref{eq:modified-solution} where $t = \triangle t_i$. The iterative process can be described as the following:
\begin{align}
  & Y_{i+1} = K_i \: Y_i + G_i \: B_i \label{eq:recurrent-equation} \\
  & K_i = e^{D \: \triangle t_i} \nonumber \\
  & G_i = D^{-1} \left( e^{D \triangle t_i} - I \right) C^{-\frac{1}{2}} \nonumber
\end{align}

Since we perform the eigenvalue decomposition of D (\equref{eq:eigenvalue-decomposition}), $D^{-1}$ can be efficiently computed in the following way:
\[
  D^{-1} = U \: \Lambda^{-1} \: U^T = U \left[
      \begin{array}{ccc}
        \frac{1}{\lambda_0} & \cdots & 0 \\
        \vdots & \ddots & \vdots \\
        0 & \cdots & \frac{1}{\lambda_{N_n - 1}}
      \end{array}
    \right] U^T \\
\]
therefore:
\begin{align*}
  G_i & = U \: \Lambda^{-1} \: U^T \left(U \: e^{\Lambda \triangle t_i} \: U^T - U \: U^T \right) C^{-\frac{1}{2}} = \\
      & = U \left[
        \begin{array}{ccc}
          \frac{e^{\triangle t_i \: \lambda_0} - 1}{\lambda_0} & \cdots & 0 \\
          \vdots & \ddots & \vdots \\
          0 & \cdots & \frac{e^{\triangle t_i \: \lambda_{N_n - 1}} - 1}{\lambda_{N_n - 1}}
        \end{array}
      \right] U^T \: C^{-\frac{1}{2}}
\end{align*}

Consequently, in order to find SSDTC, we need to solve the following system of linear equations:
\[
  \begin{cases}
    K_0 \: Y_0 - Y_1 & = -Q_0 \\
    ... \\
    K_{N_s - 1} \: Y_{N_s - 1} - Y_{N_s} & = -Q_{N_s - 1}
  \end{cases}
\]
where $Q_i = G_i \: B_i$. Also we should take into account the boundary condition which ensures that the temperature has the same values on both ends of the curve:
\begin{equation} \label{eq:boundary-condition}
  Y_0 = Y_{N_s}
\end{equation}

Hence, the system of linear equations takes the following form:
\[
  \begin{cases}
    K_0 \: Y_0 - Y_1 & = -Q_0 \\
    ... \\
    -Y_0 + K_{N_s - 1} \: Y_{N_s - 1} & = -Q_{N_s - 1}
  \end{cases}
\]

To get the whole picture, the system can be written as:
\begin{align}
  & \mathbb{A} \: \mathbb{Y} = \mathbb{B} \label{eq:system} \\
  & \mathbb{A} = \left[
    \begin{array}{ccccc}
      K_0 & -I & 0 & \cdots & 0 \\
      0 & K_1 & -I &  & \vdots \\
      \vdots &  & \ddots & -I & 0 \\
      0 &  &  & K_{N_s - 2} & -I \\
      -I & 0 & \cdots & 0 & K_{N_s - 1}
    \end{array}
  \right] \nonumber \\
  & \mathbb{Y} = \left[
    \begin{array}{c}
      Y_0 \\
      \vdots \\
      Y_{N_s - 1}
    \end{array}
  \right] \nonumber \\
  & \mathbb{B} = \left[
    \begin{array}{c}
      -Q_0 \\
      \vdots \\
      -Q_{N_s - 1}
    \end{array}
  \right] \nonumber
\end{align}

where $\mathbb{A}$ is a square matrix of the dimensions $N_n N_s \times N_n N_s$. $\mathbb{Y}$ and $\mathbb{B}$ are vectors of the length $N_n N_s$.

We have obtained a regular system of liner equations with the SSDTP as its solution ($Y$ should also be processed with \equref{eq:finalization} in order to return back to $T$).

\subsection{Direct Dense Solution}
The first straight-forward way to solve the system is to use dense solvers such as the LU decomposition \cite{press2007}. The problem here is that the systems to solve could be extremely large, especially when we want to achieve a higher accuracy and, therefore, the power profile contains a lot of steps $N_s$. Each new step produces $N_n$ new equations in the system given by \equref{eq:system}. The complexity grows very rapidly with the number of processing elements $N_p$. For instanse, one of the models implemented in HotSpot is based on the relation $N_n = 4 \times N_p + 12$ where each new core increases the system by $4 \times N_s$ equations and $4 \times N_s$ unknowns. All in all, a fast and accurate approach to solve \equref{eq:system} is required.


\subsection{Direct Sparse Solution}
\image{sparseness-of-system}{80 210 80 210}{The sparseness of the system that we need to solve in order to obtain the SSDTP. Each blue point corresponds to a non-zero element of the matrix. All non-zero elements are located on the block diagonal of the matrix, one subdiagonal, and one subdiagonal in the left bottom corner.}
One may notice that the matrix $\mathbb{A}$ is an extremely sparse matrix with a very specific structure that can be observed in \figref{fig:sparseness-of-system}. The matrix has non-zero elements only on its block diagonal (composed of $N_n \times N_n$ matrices), one subdiagonal just above the block diagonal, and one subdiagonal in the left bottom corner. Therefore, instead of the dense LU decomposition we can apply algorithms that are specially designed for such cases.
In our experiments we use the UMFPACK library, a set of routines for solving unsymmetric sparse linear systems based on the Unsymmetric MultiFrontal method (UMF) \cite{umfpack2004}.


\subsection{Condensed Equation}
In this section we propose a fast approach to solve the system given by \equref{eq:system}. The approach consists of an auxiliary transformation aimed to be more efficient in the future calculations, \secref{sec:ce-auxiliary}, and the solution itself, \secref{sec:ce-solution}.

\subsection{Observation}
\image{sparseness-of-system}{-60 210 -60 210}{Sparseness of the system of linear equations for the SSDTP calculation. Each blue point corresponds to a non-zero element of the matrix of the system.}
We make an important observation that the system given by \equref{eq:system} has a very specific structure depicted in \figref{fig:sparseness-of-system}. Non-zero elements, marked with blue points in the figure, are located only on the block diagonal, one subdiagonal just above the block diagonal, and one subdiagonal in the left bottom corner. The block diagonal is composed of $N_n \times N_n$ matrices while all elements of the subdiagonals are equal \mbox{to $-1$}. Linear systems with the same structure arise in boundary value problems for ODEs where a common technique to solve them is to form a so-called condensed equation (CE), or condensed system \cite{stoer2002}. Before undertaking this procedure, we perform an adjustment to the initial analytical solution described in the following subsection.

\subsection{Auxiliary Transformation} \label{sec:ce-auxiliary}
It can be noticed that the analytical solution in \equref{eq:solution} includes the matrix exponential and inverse of the product \mbox{$\m{A} = - \m{C}^{-1} \: \m{G}$}, which is an arbitrary square matrix. It is preferable to have a symmetric matrix to perform these computations, since for a real symmetric matrix $\m{M}$ the following eigenvalue decomposition with independent (orthogonal) eigenvectors holds \cite{press2007}:
\begin{equation} \label{eq:eigenvalue-decomposition}
  \m{M} = \m{U} \m{\Lambda} \m{U}^\v{T}
\end{equation}
where $\m{U}$ is a square matrix of the eigenvectors and $\m{\Lambda}$ is a diagonal matrix of the eigenvalues of $\m{M}$. Once we have obtained such a decomposition, the calculation of the matrix exponential and inverse becomes a trivial task:
\begin{align*}
  & e^\m{M} = \m{U} \; e^{\m{\Lambda}} \; \m{U}^\v{T} = \m{U}\: \left[
      \begin{array}{ccc}
        e^{\lambda_0} & \cdots & 0 \\
        \vdots & \ddots & \vdots \\
        0 & \cdots & e^{\lambda_{n - 1}}
      \end{array}
    \right] \; \m{U}^\v{T} \\
  & \m{M}^{-1} = \m{U} \: \m{\Lambda}^{-1} \: \m{U}^\v{T} = \m{U} \left[
      \begin{array}{ccc}
        \frac{1}{\lambda_0} & \cdots & 0 \\
        \vdots & \ddots & \vdots \\
        0 & \cdots & \frac{1}{\lambda_{n - 1}}
      \end{array}
    \right] \m{U}^\v{T} \\
\end{align*}
where $\lambda_i$ are eigenvalues of $\m{M}$.

Since the conductance matrix $\m{G}$ is a symmetric matrix (and the capacitance $\m{C}$ a diagonal matrix) \cite{rao2007}, we can apply the following substitution aimed to keep the symmetry:
\begin{align}
  & \tilde{\v{T}}(t) = \m{C}^{\frac{1}{2}} \v{T}(t) \label{eq:substitution} \\
  & \tilde{\m{A}} = -\m{C}^{-\frac{1}{2}} \m{G} \: \m{C}^{-\frac{1}{2}} \nonumber
\end{align}
with the result:
\begin{align*}
  & \frac{d\tilde{\v{T}}(t)}{dt} = \tilde{\m{A}} \: \m{Y}(t) + \m{C}^{-\frac{1}{2}} \v{P} \\
  & \tilde{\v{T}}(t) = e^{\tilde{\m{A}} t} \tilde{\v{T}}_0 + \tilde{\m{A}}^{-1} (e^{\tilde{\m{A}} t} - \m{I}) \m{C}^{-\frac{1}{2}} \v{P}
\end{align*}
where $\tilde{\m{A}}$ is a symmetric matrix. Therefore, in the case of the matrix exponential we have:
\begin{equation} \label{eq:matrix-exponential}
  e^{\tilde{\m{A}} t} = \m{U} \: e^{\m{\Lambda} t} \: \m{U}^T = \m{U} \left[
      \begin{array}{ccc}
        e^{t \lambda_0} & \cdots & 0 \\
        \vdots & \ddots & \vdots \\
        0 & \cdots & e^{t \lambda_{N_n - 1}}
      \end{array}
    \right] \m{U}^T
\end{equation}
where $\lambda_i$ are eigenvalues of $\tilde{\m{A}}$. A similar equation can be obtained for the matrix inverse.

The next step is to update the SSDTP system given in \equref{eq:recurrent-system}:
\begin{align}
  & \tilde{\v{T}}_{i+1} = \tilde{\m{K}}_i \: \tilde{\v{T}}_i + \tilde{\m{B}}_i \: \v{P}_i \label{eq:recurrent-equation} \\
  & \tilde{\m{K}}_i = e^{\tilde{\m{A}} \: \Delta t_i} \nonumber \\
  & \tilde{\m{B}}_i = \tilde{\m{A}}^{-1} \left( e^{\tilde{\m{A}} \Delta t_i} - \m{I} \right) \m{C}^{-\frac{1}{2}} \nonumber
\end{align}
Using the eigenvalue decomposition, the last equation can be computed in the following way:
\begin{align*}
  \tilde{\m{B}}_i & = \m{U} \: \m{\Lambda}^{-1} \: \m{U}^T \left(\m{U} \: e^{\m{\Lambda} \Delta t_i} \: \m{U}^T - \m{U} \: \m{U}^T \right) \m{C}^{-\frac{1}{2}} = \\
      & = \m{U} \left[
        \begin{array}{ccc}
          \frac{e^{\Delta t_i \: \lambda_0} - 1}{\lambda_0} & \cdots & 0 \\
          \vdots & \ddots & \vdots \\
          0 & \cdots & \frac{e^{\Delta t_i \: \lambda_{N_n - 1}} - 1}{\lambda_{N_n - 1}}
        \end{array}
      \right] \m{U}^T \: \m{C}^{-\frac{1}{2}}
\end{align*}

\subsection{Solution with Condensed Equation (CE)} \label{sec:ce-solution}
Let us return back to the recurrent system given by \equref{eq:recurrent-equation} and denote \mbox{$\m{Q}_i = \tilde{\m{B}}_i \: \v{P}_i$}:
\begin{align}
  & \tilde{\v{T}}_{i + 1} = \tilde{\m{K}}_i \: \tilde{\v{T}}_i + \m{Q}_i, \; i = 0 \dots N_s - 1 \label{eq:ce-recurrent} \\
  & \tilde{\v{T}}_0 = \tilde{\v{T}}_{N_s + 1} \nonumber
\end{align}
In order to form the above-mentioned condensed equation, we perform the iterative repetition of \equref{eq:ce-recurrent} that leads us to:
\begin{equation} \label{eq:y-recurrent}
  \tilde{\v{T}}_i = \prod_{j = 0}^{i - 1} \tilde{\m{K}}_j \: \tilde{\v{T}}_0 + \m{W}_{i - 1}, \; i = 1 \dots N_s
\end{equation}
where $\m{W}_i$ are defined as the following:
\begin{align}
  \m{W}_0 & = \m{Q}_0 \nonumber \\
  \m{W}_i & = \sum_{l = 1}^i \prod_{j = l}^i \tilde{\m{K}}_j \: \m{Q}_{l - 1} + \m{Q}_i, \: i = 1 \dots N_s - 1 \nonumber \\
  \m{W}_i & = \tilde{\m{K}}_i \: \m{W}_{i - 1} + \m{Q}_i, \; i = 1 \dots N_s - 1 \label{eq:p-recurrent}
\end{align}
Therefore, we can calculate the final vector $\tilde{\v{T}}_{N_s}$ using \equref{eq:y-recurrent} and \equref{eq:p-recurrent}:
\[
  \tilde{\v{T}}_{N_s} = \prod_{j = 0}^{N_s - 1} \tilde{\m{K}}_j \: \tilde{\v{T}}_0 + \m{W}_{N_s - 1}
\]
Taking into account the boundary condition given by \equref{eq:boundary-condition}, we obtain the following system of linear equations:
\begin{equation} \label{eq:core-system}
  (\m{I} - \prod_{j = 0}^{N_s - 1} \tilde{\m{K}}_j) \: \tilde{\v{T}}_0 = \m{W}_{N_s - 1}
\end{equation}
We recall that $\tilde{\m{K}}_i$ is the matrix exponential given by \equref{eq:matrix-exponential}, therefore, the following simplification holds:
\begin{align*}
  \prod_{j = i}^l \tilde{\m{K}}_j = \prod_{j = i}^l e^{\tilde{\m{A}} \Delta t_j} & = e^{\tilde{\m{A}} \sum_{j = i}^l \Delta t_j} \\
  & = \m{U} e^{\left( \sum_{j = i}^l \Delta t_j \: \m{\Lambda} \right)} \m{U}^T
\end{align*}
Consequently:
\begin{align*}
  \prod_{j = 0}^{N_s - 1} \tilde{\m{K}}_j & = e^{\tilde{\m{A}} \period} = \m{U} \: e^{\period \m{\Lambda}} \: \m{U}^T \\
    & = \m{U} \left[
      \begin{array}{ccc}
        e^{\period \lambda_0} & \cdots & 0 \\
        \vdots & \ddots & \vdots \\
        0 & \cdots & e^{\period \lambda_{N_n - 1}}
      \end{array}
    \right] \m{U}^T
\end{align*}
where $\period$ is the application period. Substituting this product into \equref{eq:core-system}, we obtain the following system:
\[
  (\m{I} - \m{U} \: e^{\period \m{\Lambda}} \: \m{U}^T) \: \tilde{\v{T}}_0 = \m{W}_{N_s - 1}
\]
The identity matrix $\m{I}$ can be splitted into $\m{U} \m{U}^T$, hence:
\begin{align*}
  & \tilde{\v{T}}_0 = \m{U} \: (\m{I} - e^{\period \m{\Lambda}})^{-1} \: \m{U}^T \: \m{W}_{N_s - 1} \\
  & \tilde{\v{T}}_0 = \m{U} \: \left[
      \begin{array}{ccc}
        \frac{1}{1 - e^{\period \lambda_0}} & \cdots & 0 \\
        \vdots & \ddots & \vdots \\
        0 & \cdots & \frac{1}{1 - e^{\period \lambda_{N_n - 1}}}
      \end{array}
    \right] \: \m{U}^T \: \m{W}_{N_s - 1}
\end{align*}
The equation gives the initial solution vector $\tilde{\v{T}}_0$, the rest of vectors $\tilde{\v{T}}_i$ for \mbox{$i = 1 \dots N_s - 1$} are successively found from \equref{eq:ce-recurrent}.

Now we assume that the power profile is evenly sampled with the sampling interval equal to $\Delta t$, i.e., $\Delta t_i = \Delta t$ for $i = 0 \dots N_s - 1$. Having this assumption, the recurrent process in \equref{eq:recurrent-equation} turns into:
\[
  \tilde{\v{T}}_{i+1} = \tilde{\m{K}} \: \tilde{\v{T}}_i + \tilde{\m{B}} \: \v{P}_i
\]
where:
\begin{align*}
  & \tilde{\m{K}} = e^{\tilde{\m{A}} \: \Delta t} \\
  & \tilde{\m{B}} = \tilde{\m{A}}^{-1} ( e^{\tilde{\m{A}} \: \Delta t} - \m{I} ) \m{C}^{-\frac{1}{2}}
\end{align*}
Here $\tilde{\m{K}}$ and $\tilde{\m{B}}$ are constants, since they depend only on the matrices $\tilde{\m{A}}$, $\m{C}$, and sampling interval $\Delta t$, which is fixed. In this case, the block diagonal of the matrix $\tilde{\mathbb{K}}$, similar to \equref{eq:system}, is composed of the same repeating block $\tilde{\m{K}}$ and the recurrent expressions take the following form:
\begin{align}
  & \tilde{\v{T}}_{i + 1} = \tilde{\m{K}} \: \tilde{\v{T}}_i + \m{Q}_i, \; i = 0 \dots N_s - 1 \nonumber \\
  & \m{W}_i = \tilde{\m{K}} \: \m{W}_{i - 1} + \m{Q}_i, \; i = 1 \dots N_s - 1 \nonumber
\end{align}
The last step of the solution is to return back to temperature by performing the backward substitution opposite to \equref{eq:substitution}:
\[
  \v{T}_i = \m{C}^{-\frac{1}{2}} \: \tilde{\v{T}}_i, \: i = 0 \dots N_s - 1
\]

As we see, the auxiliary substitution from \secref{sec:ce-auxiliary} allows us to perform the eigenvalue decomposition with orthogonal eigenvectors (\equref{eq:eigenvalue-decomposition}) that later eases the computational process at several stages. First, the decomposition is employed to compute the matrix exponential, matrix inverse, and, consequently, matrices $\tilde{\m{K}}$ and $\tilde{\m{B}}_i$. Then, the linear system given by \equref{eq:core-system} is solved without any explicit inversion of any matrix, the solution $\tilde{\v{T}}_0$ is obtained by scalar divisions and a similarity transformation with $\m{U}$.

It should be noted that the eigenvalue decomposition is performed only once for a particular RC thermal circuit and can be thought as a given for optimization purposes significantly decreasing the computational time.


\subsection{Fast Fourier Transform}
The overall matrix of the system under the assumption of the equal time intervals becomes a block Toeplitz matrix. To be more specific, the matrix is a block-circulant matrix where each block row vector is rotated one block element to the right relative to the preceding block row vector. This leads us to a wide range of possible techniques to solve the system, e.g., the Fast Fourier Transform (FFT) \cite{mazancourt1983, vescovo1997} that we include in our experiments in \secref{sec:results-ssdtp}.


\subsection{Other solutions}
Another possible technique, that we considered but do not discuss here in details due to the shortage of space, is iterative methods for solving systems of linear systems (e.g. the Jacobi, Gauss–Seidel, Successive over-relaxation methods). These methods are designed to overcome problems of direct solvers. For instance, they do not operate on full matrices and, therefore, they consume much less memory and can be applied for large systems. The most important issues with these methods are their convergence and accuracy. In our analysis we did not observe any advantages of using this methods for this particular problem, they demonstrated slow convergence and poor accuracy.
