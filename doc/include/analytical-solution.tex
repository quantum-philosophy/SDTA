As shown in \secref{sec:hotspot-solution}, the state of the art solutions either produce inaccurate and, in many cases, completely useless results, or they are unacceplably slow. In this section we eliminate the first problem by obtaining an analytical solution for the SSDTP and tackle the second one in \secref{sec:condensed-equation} where a fast technique to perform the computations is proposed.

In the following explanation, without loss of generality, we assume $\v{T}(t) \equiv \v{T}(t) - \v{T}_{amb}$. Let the power consumption vector $\v{P}(t)$ be constant and equal to $\v{P}$; then the system given by \equref{eq:fourier-model} is a system of ordinary differential equations (ODE) with the following solution:
\begin{equation} \label{eq:solution}
  \v{T}(t) = e^{\m{A} t} \; \v{T}_0 + \m{A}^{-1} (e^{\m{A} t} - \m{I}) \; \m{C}^{-1} \v{P}
\end{equation}
where $\m{A} = -\m{C}^{-1} \: \m{G}$, $\v{T}_0$ is the initial temperature, and $\m{I}$ is the identity matrix. Therefore, given a discrete power profile, the corresponding temperature profile can be found using the following recurrence:
\begin{equation} \label{eq:recurrent-system}
  \v{T}_{i+1} = \m{K}_i \: \v{T}_i + \m{B}_i \: \v{P}_i
\end{equation}
where:
\[
  \m{K}_i = e^{\m{A} \Delta t_i} \hs \m{B}_i = \m{A}^{-1}(e^{\m{A} \Delta t_i} - \m{I})\m{C}^{-1}
\]

\subsection{Transient Temperature Analysis (TTA)} \label{sec:tta-analytical}
Given the initial temperature $\v{T}_0$, the recurrence in \equref{eq:recurrent-system} can be applied to perform the TTA. Our experiments show that, since intervals $\Delta t_i$ have the same length and matrices $\m{K}_i$ and $\m{B}_i$ become constant, this approach demonstrates a significant performance improvement relative to iterative solutions of ODEs, e.g., the fourth-order Runge-Kutta method implemented in HotSpot. The same observation is made in \cite{thiele2011}.

The TTA using the analytical technique given in \equref{eq:recurrent-system} can be employed to approximate the SSDTP by applying it over successive application periods, as shown in \secref{sec:hotspot-iterative-solution}. Since each iteration of the TTA, with this approach, is much faster than with HotSpot, it will significantly speed up the SSDTP calculation. However, the number of required iterations is similar to HotSpot (\figref{fig:hotspot-error}), still keeping the computational process slow (\secref{sec:results-ssdtp}).

\subsection{Steady-State Dynamic Temperature Analysis (SSDTA)} \label{sec:ssdta-analytical}
For the steady-state case the following system of linear equations can be derived from \equref{eq:recurrent-system}:
\[
  \begin{cases}
    \m{K}_0 \: \v{T}_0 - \v{T}_1 & = -\m{B}_0 \: \v{P}_0 \\
    ... \\
    -\v{T}_0 + \m{K}_{N_s - 1} \: \v{T}_{N_s - 1} & = -\m{B}_{N_s - 1} \: \v{P}_{N_s - 1}
  \end{cases}
\]
where the last equation enforces the boundary condition and ensures the equality of temperature values on both ends of the period:
\begin{equation} \label{eq:boundary-condition}
  \v{T}_0 = \v{T}_{N_s}
\end{equation}
To get the whole picture, the system can be written as:
\begin{align}
  & \mathbb{K} \: \mathbb{X} = \mathbb{B} \label{eq:system} \\
  & \mathbb{K} = \left[
    \begin{array}{ccccc}
      \m{K}_0 & -\m{I} & 0 & \cdots & 0 \\
      0 & \m{K}_1 & -\m{I} &  & \vdots \\
      \vdots &  & \ddots & -\m{I} & 0 \\
      0 &  &  & \m{K}_{N_s - 2} & -\m{I} \\
      -\m{I} & 0 & \cdots & 0 & \m{K}_{N_s - 1}
    \end{array}
  \right] \nonumber \\
  & \mathbb{X} = \left[
    \begin{array}{c}
      \v{T}_0 \\
      \vdots \\
      \v{T}_{N_s - 1}
    \end{array}
  \right] \; \; \; \; \mathbb{B} = \left[
    \begin{array}{c}
      -\m{B}_0 \: \v{P}_0 \\
      \vdots \\
      -\m{B}_{N_s - 1} \: \v{P}_{N_s - 1}
    \end{array}
  \right] \nonumber
\end{align}
where $\mathbb{K}$ is a $N_n N_s \times N_n N_s$ matrix, $\mathbb{X}$ and $\mathbb{B}$ are vectors with $N_n N_s$ elements. It can be seen that we have obtained a regular system of linear equations. Now we shall briefly discuss possible techniques to solve it.

The first straight-forward way to solve the system is to use dense solvers such as the LU decomposition \cite{press2007}. However, a more advanced approach is to employ sparse solvers since the matrix of the system given in \equref{eq:system} is a sparse matrix. Therefore, algorithms specially designed for such cases are preferable, e.g., the Unsymmetric MultiFrontal method \cite{umfpack2004}. The computational complexity of the solution is proportionally to $N_s^3 N_n^3$ \cite{press2007} where $N_n$ is the number of nodes and $N_s$ is the number of steps in the power profile. The problem here is that the systems can be extremely large and not feasible to solve, and, if solved, the significance of solutions can be totally lost \cite{press2007}. This is especially the case when a high accuracy is required and a fine-graned power profile is obtained with an increased number of steps and/or a large number of thermal nodes, corresponding to a high level of details in the equivalent RC thermal circuit. Our experiments have shown that direct solvers are extremely slow and consume a large amount of memory. Therefore, we will not further consider them in the paper.


The overall matrix of the system under the assumption of the equal time intervals becomes a block Toeplitz matrix. To be more specific, the matrix is a block-circulant matrix where each block row vector is rotated one block element to the right relative to the preceding block row vector. This leads us to a wide range of possible techniques to solve the system, e.g., the Fast Fourier Transform (FFT) \cite{mazancourt1983, vescovo1997} that we include in our experiments in \secref{sec:results-ssdtp}.


Another possible technique, that we considered but do not discuss here in details due to the shortage of space, is iterative methods for solving systems of linear systems (e.g. the Jacobi, Gauss–Seidel, Successive over-relaxation methods). These methods are designed to overcome problems of direct solvers, since they do not operate on full matrices and, therefore, consume less memory. Consequently, they can be applied for extremely large systems, but the most important issues with these methods are their convergence and accuracy. In our analysis we did not observe any advantages of using this methods for this particular problem, they demonstrated slow convergence and poor accuracy.

