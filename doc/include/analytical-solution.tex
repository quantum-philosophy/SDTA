As it was shown in \secref{sec:hotspot-solution}, the state of the art solutions either produce inaccurate and, in extreme cases, completely useless results, or they are unacceplably slow. In this section we eliminate the first problem by obtaining an analytical expression of the SSDTP and tackle the second one in \secref{sec:condensed-equation} where a fast technique to perform the computations is proposed.

In the onwards explanation, without loss of generality, we assume $\v{T}(t) \equiv \v{T}(t) - \v{T}_{amb}$. Let the power consumption vector $\v{P}(t)$ be constant and equal to $\v{P}$, then the system given by \equref{eq:fourier-model} is a system of ordinary differential equations (ODE) with the following solution:
\begin{equation} \label{eq:solution}
  \v{T}(t) = e^{\m{A} t} \; \v{T}_0 + \m{A}^{-1}(e^{\m{A} t} - \m{I})\m{C}^{-1} \v{P}
\end{equation}
where $\m{A} = -\m{C}^{-1} \: \m{G}$ and $\v{T}_0$ is the initial temperature. Therefore, given a discrete power profile, the corresponding temperature profile can be found using the following recurrence:
\begin{equation} \label{eq:recurrent-system}
  \v{T}_{i+1} = \m{K}_i \: \v{T}_i + \m{B}_i \: \v{P}_i
\end{equation}
where:
\begin{align*}
  & \m{K}_i = e^{\m{A} \Delta t_i} \\
  & \m{B}_i = \m{A}^{-1}(e^{\m{A} \Delta t_i} - \m{I})\m{C}^{-1}
\end{align*}

\subsection{Transient Temperature Analysis}
Given the initial temperature $\v{T}_0$, the recurrence in \equref{eq:recurrent-system} can be applied to perform the TTA. Our experiments show that in the case of equal time intervals, where $\m{K}_i$ and $\m{B}_i$ are constants, this approach demonstrates a significant performance improvement relative to iterative solutions of ODEs, e.g., the fourth-order Runge-Kutta method implemented in HotSpot. The same observation is made in \cite{thiele2011}.

The TTA using the analytical technique given in \equref{eq:recurrent-system} can be employed to approximate the SSDTP as it is described in \secref{sec:hotspot-iterative-solution} dramatically speeding up the computational process. However, the number of required simulations is unchanged since it is basically an alternative way to perform the same type of analysis. The technique that we shall propose later on equals to only two simulations using the above-mentioned analytical approach while the produced solutions are exact.

\subsection{Steady-State Dynamic Temperature Analysis}
For the steady-state case the following system of linear equations can be derived from \equref{eq:recurrent-system}:
\[
  \begin{cases}
    \m{K}_0 \: \v{T}_0 - \v{T}_1 & = -\m{B}_0 \: \v{P}_0 \\
    ... \\
    -\v{T}_0 + \m{K}_{N_s - 1} \: \v{T}_{N_s - 1} & = -\m{B}_{N_s - 1} \: \v{P}_{N_s - 1}
  \end{cases}
\]
where the last equation respects the boundary condition and ensures the equality of temperature values on both ends of the overall period:
\begin{equation} \label{eq:boundary-condition}
  \v{T}_0 = \v{T}_{N_s}
\end{equation}
To get the whole picture, the system can be written as:
\begin{align}
  & \mathbb{K} \: \mathbb{X} = \mathbb{B} \label{eq:system} \\
  & \mathbb{K} = \left[
    \begin{array}{ccccc}
      \m{K}_0 & -\m{I} & 0 & \cdots & 0 \\
      0 & \m{K}_1 & -\m{I} &  & \vdots \\
      \vdots &  & \ddots & -\m{I} & 0 \\
      0 &  &  & \m{K}_{N_s - 2} & -\m{I} \\
      -\m{I} & 0 & \cdots & 0 & \m{K}_{N_s - 1}
    \end{array}
  \right] \nonumber \\
  & \mathbb{X} = \left[
    \begin{array}{c}
      \v{T}_0 \\
      \vdots \\
      \v{T}_{N_s - 1}
    \end{array}
  \right] \nonumber \\
  & \mathbb{B} = \left[
    \begin{array}{c}
      -\m{B}_0 \: \v{P}_0 \\
      \vdots \\
      -\m{B}_{N_s - 1} \: \v{P}_{N_s - 1}
    \end{array}
  \right] \nonumber
\end{align}
where $\mathbb{K}$ is a $N_n N_s \times N_n N_s$ matrix, $\mathbb{X}$ and $\mathbb{B}$ are vectors with $N_n N_s$ elements.

It can be seen that we have obtained a regular system of liner equations with a specific structure. Now we shall briefly discuss possible alternatives to solve it.

\subsubsection{Direct Dense Solutions}
The first straight-forward way to solve the system is to use dense solvers such as the LU decomposition \cite{press2007}. The problem here is that the systems to solve could be extremely large, especially when we want to achieve a higher accuracy and, therefore, the power profile contains a lot of steps $N_s$. Each new step produces $N_n$ new equations in the system given by \equref{eq:system}. The complexity grows very rapidly with the number of processing elements $N_p$. For instanse, one of the models implemented in HotSpot is based on the relation $N_n = 4 \times N_p + 12$ where each new core increases the system by $4 \times N_s$ equations and $4 \times N_s$ unknowns. All in all, a fast and accurate approach to solve \equref{eq:system} is required.


\subsubsection{Direct Sparse Solutions}
\image{sparseness-of-system}{80 210 80 210}{The sparseness of the system that we need to solve in order to obtain the SSDTP. Each blue point corresponds to a non-zero element of the matrix. All non-zero elements are located on the block diagonal of the matrix, one subdiagonal, and one subdiagonal in the left bottom corner.}
One may notice that the matrix $\mathbb{A}$ is an extremely sparse matrix with a very specific structure that can be observed in \figref{fig:sparseness-of-system}. The matrix has non-zero elements only on its block diagonal (composed of $N_n \times N_n$ matrices), one subdiagonal just above the block diagonal, and one subdiagonal in the left bottom corner. Therefore, instead of the dense LU decomposition we can apply algorithms that are specially designed for such cases.
In our experiments we use the UMFPACK library, a set of routines for solving unsymmetric sparse linear systems based on the Unsymmetric MultiFrontal method (UMF) \cite{umfpack2004}.


\subsubsection{Solutions for Toeplitz and Block-Circulant Systems}
The overall matrix of the system under the assumption of the equal time intervals becomes a block Toeplitz matrix. To be more specific, the matrix is a block-circulant matrix where each block row vector is rotated one block element to the right relative to the preceding block row vector. This leads us to a wide range of possible techniques to solve the system, e.g., the Fast Fourier Transform (FFT) \cite{mazancourt1983, vescovo1997} that we include in our experiments in \secref{sec:results-ssdtp}.


\subsubsection{Iterative Methods for Systems of Linear Equations}
Another possible technique, that we considered but do not discuss here in details due to the shortage of space, is iterative methods for solving systems of linear systems (e.g. the Jacobi, Gauss–Seidel, Successive over-relaxation methods). These methods are designed to overcome problems of direct solvers, since they do not operate on full matrices and, therefore, consume less memory. Consequently, they can be applied for extremely large systems, but the most important issues with these methods are their convergence and accuracy. In our analysis we did not observe any advantages of using this methods for this particular problem, they demonstrated slow convergence and poor accuracy.

