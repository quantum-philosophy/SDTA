Temperature analysis today is one of the important components of an embedded system design framework. There are three common types of the analysis: the Transient Temperature Analysis (TTA), Steady-State Temperature Analysis (SSTA), and Steady-State Dynamic Temperature Analysis (SSDTA). The TTA is used to simulate a particular time interval where the detailed knowledge of the temperature behaviour is required, e.g., the maximal temperature and its oscillation. In the second case, one is interested in the steady state of the system when the thermal balance is assumed to be reached and temperature does not change. The SSDTA combines properties of both of the formers and analises the time period where the stabilization process is also assumed to be finished, but instead of having a constant temperature the thermal system exhibits a periodic pattern, called the Steady-State Dynamic Temperature Profile (SSDTP). The later is the subject of this paper.

The de facto approach to perform temperature analysis is to construct an equivalent RC thermal circuit \cite{kreith2000} of the system as it is outlined in \secref{sec:thermal-model}. Concequently, a large number of techniques based on this approach has appeared where the major difference is in the level of granularity with which they represent the system. HotSpot \cite{huang2006}, an architecture and system-level model and simulator, is usually the default choice, e.g., \cite{lu2004, srinivasan2004, liao2005, coskun2006, liu2007, huang2009, xiang2010, thiele2011}. Since temperature analysis is a computationaly entensive task, several simplifications have been proposed in order to speed up the process \cite{rao2007, hanumaiah2009}; nevetheless, temperature analysis remains one of the main bottlenecks of the temperature-aware optimization techniques.

The SSDTA can be frequently found in the literature when the subject of the discussion is systems that undergo periodic workloads. For instance, in \cite{bao2010} an equivalent 1-RC thermal circuit of a single processor platform is constructed and an energy optimization of a set of periodic tasks is performed. Another important and widely spread application of the SSDTA is the estimation of the reliability charachteristics of the system. The failure mechanisms commonly considered are electromigraion, time dependent dielectric breakdown, and thermal cycling \cite{jedec2010}. The importance of temperature gradients on the interconnect reliability is discussed in \cite{lu2004}. In \cite{hieu2004} the fast thermal cycling-induced thin film cracking is studied. The effect of the temperature management techniques is considered in \cite{srinivasan2004}.

Temperature-aware reliability optimization today is becomming an essential part of the design process of embedded systems.  In \cite{coskun2006} a power management policy is formulated as a linear program under reliability constrains induced by the above-mentioned temperature-dependent failure mechanisms. A reliability model for multiprocessor systems with periodic tasks is proposed in \cite{huang2009}. A similar model is used in \cite{xiang2010} where a combination of two failure rate distributions is introduced as a part of the system-level reliability modeling.

The need in appropriate techniques for the temperature analysis is constantly increasing. In this paper we consider multiprocessor systems with periodic power profiles and propose an accurate and fast approach for the SSDTA based on the RC thermal model.

The rest of the paper is organized as the following. In \secref{sec:preliminaries} we introduce the architecture, power, and thermal models. The problem formulation is given in \secref{sec:problem}. Possible solutions with existing thermal simulators are discussed in \secref{sec:hotspot-solution}. In \secref{sec:analytical-solution} we obtain an analytical expression of the SSDTP and overview several methods to compute it. In \secref{sec:condensed-equation} we propose a fast and elegant technique to solve the problem. The leakage power is taken into account in \secref{sec:leakage}. \secref{sec:reliability} describes the reliability model that our experiments are based on. The performance comparison for the SSDTP calculation as well as the results of the reliability optimization are given in \secref{sec:results}. Finally, \secref{sec:conclusion} concludes the paper.
