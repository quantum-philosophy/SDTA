Temperature analysis today is one of the important components of an embedded system design framework. There are three common types of the analysis: the Transient Temperature Analysis (TTA), Steady-State Temperature Analysis (SSTA), and Steady-State Dynamic Temperature Analysis (SSDTA). The TTA is used to simulate a particular time interval where the detailed knowledge of the temperature behaviour is required, e.g., the maximal temperature and its oscillation. In the second case, one is interested in the steady state of the system when the thermal balance is assumed to be reached and temperature does not change. The SSDTA combines properties of both of the formers and analises the time period where the stabilization process is also assumed to be finished, but instead of having a constant temperature the thermal system exhibits a periodic pattern, called the Steady-State Dynamic Temperature Profile (SSDTP). The later is the subject of this paper.

The de facto approach to perform temperature analysis is to construct the equivalent RC thermal circuit \cite{kreith2000} of the system as it is outlined in \secref{sec:thermal-model}. Concequently, a large number of techniques based on this approach has appeared where the major difference is in the level of details with which they represent the system. HotSpot \cite{huang2008}, an architecture and system-level model and simulator, is usually the default choice, e.g., \cite{lu2004, srinivasan2004, liao2005, coskun2006, liu2007, huang2009, xiang2010, thiele2011}. However, the computational performance of HotSpot is often not sufficient for intensive optimization purposes that leaded to a number of proposed simplifications aimed to speed up the process \cite{rao2007, rao2009}. Nevetheless, temperature analysis remains one of the main bottlenecks of the temperature-aware optimization techniques.

The SSDTA can be frequently found in the literature where the subject of the discussion is systems that undergo periodic workloads. For instance, in \cite{bao2010} a set of periodic tasks is considered and an energy optimization is performed, however, only for a single processor platform.

The important and widely spread application of the SSDTA is the estimation of the reliability charachteristics of multiprocessor systems. The concern about temperature-aware reliability is dictated by the presence of a significant influence of temperature on integrated circuits, which can dramatically affect their lifetime \cite{lu2004, srinivasan2004, jedec2010, hieu2004}. The failure mechanisms commonly considered are electromigraion, time dependent dielectric breakdown, and thermal cycling, which are directly driven by the temperature \cite{jedec2010}. In \cite{huang2009} a reliability model for multiprocessor systems with periodic tasks is proposed. A similar model is derived in \cite{xiang2010} where a combination of two failure rate distributions is introduced as a part of the system-level reliability modeling. In \cite{huang2009} the knowledge of the SSDTP is essential and a rough approximation of it is proposed, however, as it is shown in \secref{sec:steady-state-approximation}, the approximation is inaccurate for a wide range of configurations. The analysis in \cite{xiang2010} is based on the simulation of long traces with approximately 500 successive periods of applications over which the TTA is performed by HotSpot. Such a technique takes a significant computational time, as demonstrated in \secref{sec:results-ssdtp}, and is problematic to be applied inside an optimization loop. In \cite{coskun2006} a power management policy is formulated as a linear program under reliability constrains induced by the above-mentioned temperature-dependent failure mechanisms. In this case, two opposite approaches are proposed, a coarse approximation and cycle-accurate TTA simulation, with the same discussion about accuracy and speed as above. In \cite{srinivasan2004} the SSA followed by one TTA iteration is employed in order to approximate the SSDTA. The first one is used to obtain the initial temperature for the second with the assumption that one TTA iteration can hypotheticaly lead to a satisfactory SSDTP estimation.

As we can see, the demand in accurate and fast SSDTA techniques is constantly increasing. In this paper we consider multiprocessor systems with periodic power profiles and propose an accurate and fast approach for the SSDTP calculation based on the RC thermal model. We demonstrate the proposed solution by performing a termperature-aware reliability optimization based on the thermal cycling failure mechanism and show that the SSDTA is a must for an embedded system design framework.

The rest of the paper is organized as the following. In \secref{sec:preliminaries} we introduce the architecture, power, and thermal models. The problem formulation is given in \secref{sec:problem}. Possible solutions with existing thermal simulators are discussed in \secref{sec:hotspot-solution}. In \secref{sec:analytical-solution} we obtain an analytical expression of the SSDTP and overview several solution methods. In \secref{sec:condensed-equation} we propose a fast and accurate technique to compute the SSDTP. The leakage power is taken into account in \secref{sec:leakage}. \secref{sec:reliability} describes the reliability model that our experiments are based on. The performance comparison for the SSDTP calculation as well as the results of the reliability optimization are given in \secref{sec:results}. Finally, \secref{sec:conclusion} concludes the paper.
