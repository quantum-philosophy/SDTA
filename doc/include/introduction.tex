Temperature analysis today is one of the important components of an embedded system design framework, since the temperature oscilation and distribution over integreted circuits cannot be neglected and should be properly taken into account during the design stage of an embedded system.

About temperature analysis in general.

About two types of this analysis.

Referencies to Min.

Problems with existing solutions.

The influence of temperature on the reliability characteristics on the mictorachitecture level is widely studied in the literature. The models of the common failure machanisms are annually published in \cite{jedec2010}. The importance of the temporal and spatial temperature gradients on the interconnect reliability is discussed in \cite{lu2004}. In \cite{hieu2004} the fast thermal cycling-induced thin film cracking is studied. The effect of the temperature management techniques is given in \cite{srinivasan2003}.

Consequently, the temperature-aware reliability optimization today is an essential part of the design process of embedded systems where the highly discussed failure mechanisms are electromigraion, time dependent dielectric breakdown, and thermal cycling. In \cite{coskun2006} a power management policy is formulated as a linear program under reliability constrains induced by the above-mentioned temperature-dependent failure mechanisms. A reliability model for multiprocessor systems with periodic tasks is proposed in \cite{huang2009}. A similar model is used in \cite{xiang2010} where a combination of two failure rate distributions is proposed as a part of the system-level reliability modeling.

All in all, the need in appropriate techniques for the temperature analysis is constantly increasing. In this paper we consider multiprocessor systems with periodic power profiles and propose an accurate and fast approach for the Steady-State Dynamic Temperature Analysis (SSDTA) based on the RC thermal model.

The rest of the paper is organized as the following. In \secref{sec:preliminaries} we introduce the architecture, power, and thermal models. The problem formulation is given in \secref{sec:problem} where we are aimed to find the Steady-State Dynamic Temperature Profile (SSDTP). Possible solutions with existing thermal simulators are given in \secref{sec:hotspot-solution}. In \secref{sec:analytical-solution} we obtain an analytical expression of the SSDTP, overview several techiques to tackle it, and propose our solution. The power leakage is taken into account in \secref{sec:leakage}. \secref{sec:reliability} is dedicated to the reliability model that our experiments are based on. The performance comparison between different approaches for the SSDTP calculation as well as the results of the reliability optimization are given in \secref{sec:results}. Finally, \secref{sec:conclusion} concludes the paper.
