Due to increasing power densities, temperature has evolved into a major concern for designers of modern embedded systems. Thus, temperature analysis has become an important component of current embedded system design frameworks.

Temperature-aware system-level design methods rely on the availability of temperature modeling and analysis tools. System-level temperature modeling approaches are mostly based on the duality between heat transfer and electrical phenomena \cite{kreith2000}. The basic idea is to build an equivalent circuit of thermal resistances and capacitances capturing both the architecture blocks and elements of the thermal package. HotSpot \cite{huang2003}, an architecture and system-level model and simulator, is the state of the art choice for system-level temperature analysis, as in \cite{srinivasan2004, liao2005, coskun2006, liu2007, huang2009, xiang2010, thiele2011}.

However, temperature analysis time with HotSpot, or other similar approaches, is too long to be used inside a tempera\-ture-aware system-level optimization loop. The long thermal simulation time can severely limit the efficiency of the design space exploration. There has been some work on establishing fast system-level temperature analysis techniques. They also build on the duality between heat transfer and electrical phenomena and are based on restrictive assumptions in order to simplify the model. The approaches proposed in \cite{rai2011, bao2010}, for example, are strictly restricted to monocore systems. The method described in \cite{rao2009} is restricted to homogeneous platforms and to applications in which the execution time of individual tasks is long, comparable with the thermal time constant of the package (in the order of 100 $s$).

Broadly speaking, there are two types of thermal analysis: (1) static temperature analysis, that produces a hypothetical temperature value (the steady-state temperature) at which the circuit is supposed to function if running for a long time under a certain constant (average) input power; (2) dynamic temperature analysis, that produces a transient temperature curve which describes the temperature behavior of the circuit as a function of time, when exposed to an arbitrary power profile.

The steady-state temperature, as produced by static analysis, is an approximation of the thermal behavior with limited applicability. It assumes that, eventually, the circuit will function at one constant temperature.  This, however, is very often not the case in reality. In the context of a variable power profile applied periodically, the circuit will not reach a constant steady-state temperature but a steady state in which temperature is varying according to a certain periodic pattern. This pattern is captured by the steady-state dynamic temperature profile (SSDTP).

In the case of applications which exhibit periodic or close to periodic behavior (or which are characterized by several operation modes, each of which exhibits a periodic behavior), the SSDTP is of particular importance for system design. Any design optimization has to be performed such that the efficiency and reliability of the system are maximized considering not a very short transient time interval at system start (or mode setup) but the context in which the system functions over a long period of time.

A typical design task, for which the SSDTP is of central importance, is temperature-aware reliability optimization. The impact of temperature on the lifetime of electronic circuits is well-known \cite{srinivasan2004, coskun2006, xiang2010, jedec2010}. The failure mechanisms commonly considered are electromigration, time-dependent dielectric breakdown, and thermal cycling, which are directly driven by the temperature \cite{jedec2010}. What is important in this context is that not only average and maximum temperature, but also the amplitude and frequency of temperature oscillations, have a huge impact on the overall lifetime of the chip. Thus, efficient reliability optimization depends on the availability of the actual SSDTP.

Two approaches have been applied in literature in order to obtain the SSDTP, as a prerequisite for reliability optimization. An approximate SSDTP can be produced by running a temperature simulator over one or more successive periods of the application until one can assume that a sufficient approximation of the thermal steady state has been reached \cite{srinivasan2004}. Such an approach is both time consuming and potentially inaccurate. A very rough but fast approximation of the SSDTP is proposed in \cite{huang2009}. It constructs a stepwise temperature curve where each step corresponds to the static steady-state temperature that would be reached if a certain constant power was applied for a sufficiently long time. In \secref{sec:hotspot-solution} we will further elaborate on these two state of the art solutions. As our experiments show, they are too slow and/or too inaccurate in order to efficiently be used inside a temperature-aware system-level optimization loop for, e.g., reliability optimization.

In this paper we consider multiprocessor systems running applications exhibiting a power profile that can be considered periodic. Our contribution is twofold. First, we propose an approach that is both accurate and fast, for SSDTP calculation. Second, we show how our approach makes it possible to efficiently perform reliability optimization, based on the thermal cycling (TC) failure mechanism. More exactly, we propose a temperature-aware task mapping and scheduling technique that addresses the TC ageing effect. Experiments demonstrate the superiority of the proposed techniques, compared to the state of the art.

The rest of the paper is organized as follows. In \secref{sec:preliminaries} we introduce the architecture, power, and thermal models. The problem formulation is given in \secref{sec:problem}. The state of the art solutions are discussed in \secref{sec:hotspot-solution}. In \secref{sec:analytical-solution} we obtain an analytical formulation for the SSDTP calculation. In \secref{sec:condensed-equation} and \secref{sec:leakage} we propose a fast and accurate technique to compute the SSDTP. \secref{sec:reliability} formulates the temperature-aware reliability optimization problem and proposes a solution based on our fast SSDTP calculation. Experimental results are given in \secref{sec:results} and \secref{sec:conclusion} concludes the paper. Supplementary materials are given in the appendix, Sec. \ref{app:thermal-circuits}--\ref{app:references}.
