The temperature analysis is one of the most important components of a well-established design of an embedded system. Whenever it is possible, system architects are trying to take it into account in order to produce results of a higher quality. The measurment of such quality depends on the purpose of a particular system. For instance, it can be the mean time to failure (MTTF) or energy consumption. High quality always comes by taking the right desicions during different stages of the production process of an embedded system. Good examples of the hard questions that have to be resolved are the floorplanning, task allocation, and task scheduling. Since the temperature-aware design is by far not a new topic to discuss, one may find a lot of examples in the literature where taking into account the operational temperature of the die yields significant improvemnts in the obtainted results. Unfortunately, this is not a staight-forward thing to do, since the operational temperature depends on a lot of different parameters that in their turn can be even more complex, for example, the dynamic power consumption and power leakage. Moreover, modern policies of the temperature runaway prevention and energy conservation (e.g., the dynamic power management and dynamic voltage/frequency scaling) make the situation even worse causing considerable temperature fluctuations within the die and, consequently, decreasing its lifetime \cite{simunic2005}.

All in all, the influence of the temperature oscilation and distribution over integreted circuits is an extremely urgent topic (\cite{hieu2004}, \cite{lu2004}, \cite{jedec2010}, \cite{xiang2010}, etc.) and the need in adequate techniques for the temperature analysis is constantly increasing over the years. In this paper we consider multiprocessor systems with periodic power and temperature profiles and propose a precise (withing the constrains of the thermal model) and fast approach for the steady-state dynamic temperature analysis (SSDTA). The proposed approach is based on the HotSpot thermal model \cite{huang2006}.

The rest of the paper is organized as the following. In \secref{sec:preliminaries} we briefly discuss the basics of the RC thermal model that we use in our analysis. We overview the prior work and state our motivation in \secref{sec:motivation}. The problem formulation is given in \secref{sec:problem}. After we define the problem, we investigate what can be done using the HotSpot simulator, \secref{sec:hotspot-solution}. In \secref{sec:analytical-solution} we obtain an analytical expression of the problem and propose several techiques to solve it. In \secref{sec:reliability} we overview the reliability model that our experiments are based on. The performance comparison between different approaches for the SSDTP calculation as well as the results of the reliability optimization are given in \secref{sec:results}. \secref{sec:conclusion} concludes our paper and outlines the future work.
