There is no need to say how important it is to take right decisions during all the stages of the production process of an embedded system. Right decisions lead to systems of a high quality from different perspectives, e.g., high reliability, lower energy consumption, or a long lifetime of the device itself. Good examples of the hard questions that have to be resolved during the design stage are the floorplanning, task allocation, and task scheduling. It has been shown that taking into account the operational temperature of the die can produce considerably better results concerning all these decisions. But this is not an easy thing to do, since the operational temperature depends on a lot of parameters, such as the dynamic power consumption and power leakage. Moreover, modern policies of the temperature runaway prevention and energy conservation (for example, the dynamic power management and dynamic voltage/frequency scaling) make the situation even worse causing considerable temperature fluctuations within the die and, consequently, decreasing its lifetime \cite{mihic2004}, \cite{simunic2005}.

The importance of the temperature distribution over ICs and its oscilation have been widely studied in the literature (for instance, \cite{hieu2004}, \cite{lu2004}, \cite{jedec2010}, \cite{xiang2010}, etc.) which proves the fact that it is an extremely urgent topic and the need in adequate temperature analysis techniques is constantly increasing over years. In this paper we consider the HotSpot thermal model \cite{huang2006} and propose a precise (withing the constrains of the model) and fast way to calculate the steady-state dynamic temperature profile (SSDTP) of a multiprocessor system that executes a set of periodic tasks. We assume that the power profile of the system is periodic and, hence, the corresponding temperature profile is periodic as well. The SSDTP consists of a number of temperature curves for each of the processing units, we shall refer to these curves as the steady-state dynamic temperature curves (SSDTC).

In order to demonstrate our approach to the steady-state dynamic temperature analysis (SSDTA), we perform the temperature-aware reliability optimization of multiprocessor systems with periodic applications. The reliability model, that we have chosen as an example, is based on the thermal cycling failure mechanism \cite{xiang2010}, \cite{huang2009}. We use a genetic algorithm \cite{schmitz2004} for the task allocation and scheduling in order to optimize the mean time to failure (MTTF) of the system. We also conduct a multi-objective optimization with the MTTF as the first goal and the energy consumption as the second. In this case, the solution is a Pareto front.
