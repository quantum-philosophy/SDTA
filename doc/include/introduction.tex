Temperature analysis today is one of the important components of an embedded system design framework. There are three common approaches to perform this analysis: the Transient Temperature Analysis (TTA), Steady-State Temperature Analysis (SSTA), and Steady-State Dynamic Temperature Analysis (SSDTA). The TTA is used to simulate the temperature behaviour during a particular time interval where the details of this behaviour can be obtained, e.g., the maximal temperature or its oscillation. In the second case, one is interested in the steady state of the system when the thermal balance is assumed to be reached and temperature does not change. The SSDTA combines properties of both of the formers and analises the time period where the stabilization process is also assumed to be finished, but instead of having a constant temperature the thermal system exhibits a periodic pattern, called the Steady-State Dynamic Temperature Profile (SSDTP).

\todo{References to Min.}

\todo{Problems with existing solutions.}

The influence of temperature on the reliability characteristics of integrated circuits is widely studied in the literature. The models of the common failure machanisms are annually published in \cite{jedec2010}. The importance of temperature gradients on the interconnect reliability is discussed in \cite{lu2004}. In \cite{hieu2004} the fast thermal cycling-induced thin film cracking is studied. The effect of the temperature management techniques is considered in \cite{srinivasan2003}.

Temperature-aware reliability optimization today is becomming an essential part of the design process of embedded systems. The failure mechanisms considered are electromigraion, time dependent dielectric breakdown, and thermal cycling. In \cite{coskun2006} a power management policy is formulated as a linear program under reliability constrains induced by the above-mentioned temperature-dependent failure mechanisms. A reliability model for multiprocessor systems with periodic tasks is proposed in \cite{huang2009}. A similar model is used in \cite{xiang2010} where a combination of two failure rate distributions is proposed as a part of the system-level reliability modeling.

All in all, the need in appropriate techniques for the temperature analysis is constantly increasing. In this paper we consider multiprocessor systems with periodic power profiles and propose an accurate and fast approach for the SSDTA based on the RC thermal model.

The rest of the paper is organized as the following. In \secref{sec:preliminaries} we introduce the architecture, power, and thermal models. The problem formulation is given in \secref{sec:problem}. Possible solutions with existing thermal simulators are discussed in \secref{sec:hotspot-solution}. In \secref{sec:analytical-solution} we obtain an analytical expression of the SSDTP, overview several techiques to compute it, and propose our solution. The leakage power is taken into account in \secref{sec:leakage}. \secref{sec:reliability} describes the reliability model that our experiments are based on. The performance comparison for the SSDTP calculation as well as the results of the reliability optimization are given in \secref{sec:results}. Finally, \secref{sec:conclusion} concludes the paper.
