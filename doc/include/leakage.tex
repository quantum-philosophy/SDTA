  Nowadays, power leakage is an important issue that should be carefully taken into account. We propose to employ two techniques to consider leakage in the analytical solution using the CE method \cite{liu2007}:
  \begin{itemize}
    \item A linear approximation of the leakage current.
    \item An alternate computation of the temperature and power profiles until a convergence of temperature.
  \end{itemize}

\subsection{Linear Approximation}
In can be seen in \equref{eq:thermal-ode} that a linear approximation of the leakage power:
\[
  P_{leak}(t) = A \: T(t) + B
\]
where $A$ is a diagonal matrix keeps all the equations of the proposed analytical solution untouched:
\begin{align*}
  & C \: \frac{dT(t)}{dt} + \bar{G} \: T(t) = \bar{P} \\
  & \bar{G} = G - A \\
  & \bar{P} = P_{dyn} + B
\end{align*}

\subsection{Alternate Computation}
In this case we have an iterative process where we calculate the temperature and power profiles in turns. With each new temperature profile we update the power profile by computing the power leakage and adding it to the dynamic power:
\[
  P_i = P_{dyn} + P_{leak}(T_i)
\]
The process continues until the temperature converges, i.e., the different between two successive temperature profile is below a redefined bound.
