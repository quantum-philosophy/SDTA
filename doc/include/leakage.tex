So far in our computations, we have assumed that power is independent on temperature. In reality, due to the leakage component, the power dissipation is a strong function of temperatue that cannot be neglected (\secref{sec:power-model}). Two techniques can be applied to enhance our proposed solution with temperature dependent leakage modeling.

\subsection{Alternate Computation}
\image{leakage}{0 0 0 0}{The closed loop between temperature and power.}
In this case, we have an iterative process depicted in \figref{fig:leakage} where the temperature and power profiles are calculated in turns. With each new temperature profile we update the power profile by computing the leakage power and adding it to the dynamic power:
\[
  \mathbb{P}_i = \mathbb{P}_{dyn} + \mathbb{P}_{leak}(\mathbb{T}_i)
\]
The process continues until the temperature converges, i.e., the difference between two successive temperature profiles is below a predefined bound.

\subsection{Linear Approximation}
A linear approximation of the leakage power has the following matrix form:
\[
  \v{P}_{leak}(\v{T}) = \m{A} \: \v{T}(t) + \v{B}
\]
where $\m{A}$ is a $N_n \times N_n$ diagonal matrix of the proportionality and $\v{B}$ is a vector with $N_n$ elements of the intercept. Both characterize the leakage power for each of the $N_n$ thermal nodes in the system. It can be seen that the approximation keeps \equref{eq:fourier-model} untouched:
\[
  \m{C} \: \frac{d\v{T}(t)}{dt} + \bar{\m{G}} \: (\v{T}(t) - \v{T}_{amb}) = \bar{\v{P}}
\]
where:
\begin{align*}
  & \bar{\m{G}} = \m{G} - \m{A} \\
  & \bar{\v{P}} = \v{P}_{dyn} + \m{A} \: \v{T}_{amb} + \v{B}
\end{align*}
Therefore, all solutions based on this equation can include the linearized leakage model with no additional costs. Moreover, in spite of its simplicity, the approximation provides a good estimation as it is shown in \cite{liu2007}.
