In order to demonstrate the speed and accuracy of the proposed solution, we apply it in the context of temperature-aware mapping and scheduling in order to address the agin effect and optimize the energy consumption. Both mapping and scheduling are based on genetic algorithms described in \cite{schmitz2004}. The genetic scheduling internally relies on the list scheduler. Let us start from the description of the system that we use.

\subsection{Architecture Model}
We consider a multiprocessor system with heterogeneous architecture that is composed of a number of processing elements (PE). Each processing element $PE_i$ is characterized by its supply voltage $V_i$, frequency $f_i$, and the number of gates:
\begin{align*}
  & PE = \{ PE_i \} \\
  & PE_i \rightarrow (V_i, f_i, N_{gate i})
\end{align*}

\subsection{Task Model}
The system runs a periodic application $A$ which contains a number of tasks. The period of the application $T$ can be thought as a shared deadline for all the tasks. The tasks themselves and the data dependencies between them are described with a task graph:
\begin{align*}
  & A \rightarrow (G, T) \\
  & G = (\Pi, \Gamma) \\
  & \Pi = \{\tau_i\} \\
  & \Gamma = \{\gamma_i\}
\end{align*}

Each pair of a processing element $PE_i$ and a task $\tau_j$ is described by the effective switched capacitance $C_{eff \; ij}$ and the number of clock cycles $NC_{ij}$ that are requited for the given task to perform on the given core:
\begin{equation*}
  (PE_i, \tau_j) \rightarrow (C_{eff \; ij}, NC_{ij})
\end{equation*}

\subsection{Mapping and Scheduling}
We use genetic algorithms for mapping and scheduling presented in \cite{schmitz2004}.

\subsection{Power Model}
The total power consumption of a task is a sum of the dynamic power and the leakage power:
\begin{align*}
  & P = P_{dyn} + P_{leak} \\
  & P_{dyn} = C_{eff} \cdot f \cdot V_{dd}^2
\end{align*}
where $C_{eff}$ is the effective switched capacitance of a task, $V_{dd}$ and $f$ are the supplied voltage and frequency of a processing element correspondingly. For the leakage part, the model presented in \cite{liao2005} is used\footnote{Although, the leakage model can be computational intensive, the linear approximation presented in \cite{liu2007} can be used.}:
\begin{align*}
  & P_{leak} = N_{gate} \cdot I_{avg} \cdot V_{dd} \\
  & I_{avg} = I_s(T_0, V_0) \cdot f_{avg}(T, V_{dd}) \\
  & f(T, V_{dd}) = A \cdot T^2 \cdot e^{((\alpha \cdot V_{dd} + \beta)/T)} + B \cdot e^{(\gamma \cdot V_{dd} + \delta)}
\end{align*}
where $N_{gate}$ is the number of gates in the circuit,\footnote{The number of gates in a circuit can be estimated using the technique described in \cite{li2004}.} $I_s (T_0, V_0)$ is the average leakage current at the given temperature $T_0$ and supply voltage $V_0$, $f_{avg}$ is the scaling function. The last two form $I_{avg}$, the average leakage current at the current temperature $T$ and voltage $V_{dd}$. $A$, $B$, $\alpha$, $\beta$, $\gamma$, and $\delta$ are the technology dependent constants given in \cite{liao2005}.

In should be noted that the dynamic power consumption does not depend on temperature, it is a constant for each combination of a processing element with its voltage/frequency pair and a task with its effective switched capacitance. That is not the case with the leakage power.

\subsection{Temperature-Aware Reliability Optimization}
We address the thermal cycling (TC) component of the common packaging/interfacial failure mechanisms \cite{jedec2010}. The reliability model, presented in \cite{xiang2010}, is used. The number of cycles to failure ($N_{TC}$) can be estimated using a modified version of the well-known Coffin-Manson equation with the Arrhenius term \cite{jedec2010}, \cite{xiang2010}, \cite{ciappa2003}:
\begin{equation} \label{eq:number_of_cycles_to_failure}
  N = A_{TC} (\triangle T - \triangle T_0)^{-b} e^{\frac{E_{a_{TC}}}{k T_{max}}}
\end{equation}
where $A_{TC}$ is an empirically determined constant, $\triangle T$ is the thermal cycle amplitude, $\triangle T_0$ is the portion of the temperature range in the elastic region which does not cause damage, $b$ is the Coffin-Manson exponent which also empirically determined, $E_{a_{TC}}$ is the activation energy, $k$ is the Boltzmann constant, and $T_{max}$ is the maximal temperature during the thermal cycle.

\equref{eq:number_of_cycles_to_failure} is suitable for one repeating thermal cycle with constant parameters. If we have a range of $m$ different thermal cycles in a given time interval, the following equation should be used instead to estimate the number of such sets of thermal cycles to failure \cite{xiang2010}:
\begin{equation} \label{eq:number_of_periods_to_failure}
  N_{TC} = \frac{m}{\sum_{i=0}^m \frac{1}{N_i}}
\end{equation}

The mean time to failure (MTTF) is calculated using the following equation:
\begin{equation} \label{eq:mttf}
  MTTF_{TC} = \frac{N_{TC} \sum_{i=1}^m t_i}{m}
\end{equation}
where $m$ is the number of cycles in a given time interval, and $\sum_{k=1}^m t_k$ is the total duration of the given time period.

Taking \equref{eq:number_of_cycles_to_failure}, \equref{eq:number_of_periods_to_failure}, and \equref{eq:mttf} together yields:
\begin{equation}
  MTTF_{TC} = \frac{A_{TC} \cdot T}{\sum_{i=1}^m (\triangle T_i - \triangle T_0)^b e^{- \frac{E_{a_{TC}}}{k T_{max i}}}}
\end{equation}

\subsection{Experimental Setup}
Each processing element is described with a supply voltage in the diapason 0.6--1.0 V, a frequency in 2--4 GHz, and a number of gates in $5^5$--$15^5$. The application, that is being run on this multiprocessor system, consists of a number of tasks in the range 5--100. The effective switched capacitance and the number of clock cycles of each task are in the ranges $2^{-8}$--$4^{-8}$ and $5^9$--$15^9$ respectively. In \equref{eq:number_of_cycles_to_failure} the constant $b$ for brittle fracture (Si and its dielectrics) is found to be 6--9 \cite{jedec2010}, $E_{a_{TC}}$ for TC lies between 0.5--0.7 eV \cite{vigrass}.

\subsection{Performance Comparison} \label{sec:comparison}
TODO.

\subsection{Obtained Results}
TODO.
