In order to demonstrate the speed and accuracy of the proposed solution, we apply it in the context of temperature-aware mapping and scheduling in order to optimize the energy consumption. Both mapping and scheduling are based on genetic algorithms described in \cite{schmitz2004}. The genetic scheduling internally relies on the list scheduler.

Let us start from the description of the system model that we use.

\subsection{System Model}
We consider a multiprocessor system with heterogeneous architecture that is composed of a number of processing elements (PE):
\[
  PE = \{ PE_i \}
\]

\note{If we can to investigate the dependency between the reliability and the thermal cycling, then we should choose the most exposed configuration to this thermal cycling. Therefore, DVFS seems to be a good example, because it is what exactly it does --- causes voltage/frequency and, consequently, temperature variations. Or we do not need such complexity and can leave only ACTIVE/IDLE modes without any DVFS policy?}

Each processing element $PE_i$ is DVFS-aware, therefore, it has a number of active states $S_i$ characterized by corresponding supply voltage $v_{ij}$ and frequency $f_{ij}$:
\begin{align*}
  & PE_i \rightarrow S_i = \{ s_{ij} \} \\
  & s_{ij} \rightarrow (v_{ij}, f_{ij})
\end{align*}

The system runs a number of periodic tasks. The tasks themselves and the data dependencies between them are represented with a task graph:
\begin{align*}
  & G = (\Pi, \Gamma) \\
  & \Pi = \{\tau_i\} \\
  & \Gamma = \{\gamma_i\}
\end{align*}

Each pair of a task $\tau_j$ and a processing element $PE_i$ is described by effective switched capacitance $C_{eff \; ij}$ and the worse case number of clock cycles $WNC_{ij}$:
\[
  (PE_i, \tau_j) \rightarrow (C_{eff \; ij}, WNC_{ij})
\]

In addition, each task has a deadline $dl_j$.

\subsection{Mapping and Scheduling}
We use genetic algorithms for mapping and scheduling from \cite{schmitz2004}.

\subsection{Temperature-Aware Energy Optimization}
The energy dissipation depends on a particular processing element, its active state, and a particular task:
\[
  E = P t = P \frac{NC}{f}
\]
where $NC$ is the number of clock cycles that the task requires.

The total power consumption of a task is usually treated as a sum of the dynamic power and the leakage power:
\begin{align*}
  & P = P_{dyn} + P_{leak} \\
  & P_{dyn} = C_{eff} V^2 f \\
  & P_{leak} = V I_{leak} = V (MT + N)
\end{align*}

In should be noted that the dynamic power consumption does not depend on temperature, it is a constant for each combination of a processing element with its active state and a task with its worse case number of clock cycles.

For the leakage part, a linear approximation, presented in \cite{liu2007}, is used. $M$ and $N$ are the leakage coefficients that are calculated at the reference temperature $T_{ref}$ using Taylor series expansion of the initial equation for the leakage current $I_{leak}$.

However, the concern about the leak of charge across the channel of the transistor seems to be decreasing with taking into the manufacturing process new types of transistors, e.g. the FinFET and the UTB.

\todo{Put some strong references here.}

\note{Therefore, we should not put too much stress on the leakage part. We should say that we can also take it into account, but it is not that necessary, it is not our main concern or contribution. Alexandru Andrei thinks so as well. Intel goes beyond 22nm with those transistors. May be we should even skip this leakage stuff?}

\todo{The energy part of the optimization problem formulation.}

\subsection{Temperature-Aware Reliability Optimization}
We address the thermal cycling (TC) component of common failure mechanisms, which also contain electromigration, time dependent dielectric breakdown, and stress migration \cite{xiang2010}.

\note{The reliability estimation is based on \cite{coskun2006}, which in its turn is based on \cite{hieu2004}. It should be mentioned, the \emph{maximal} frequency that \cite{hieu2004} considers is just $f = 80 Hz$ with the \emph{minimal} temperature variation around $\Delta T = 200^{\circ}C$, which is not by far the case here.}

\todo{The reliability part of the optimization problem formulation.}

\subsection{Performance Comparison} \label{sec:comparison}
\todo{Compare the performance with the dummy HotSpot approach.}
