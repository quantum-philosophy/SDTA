In order to demonstrate the speed and accuracy of the proposed solution, we apply it in the context of temperature-aware energy and reliability optimization.

\todo{Describe what we are trying to achieve with this example.}

\note{Do not dig too deep into details. We are trying to stay on the system level, give a big picture of the usage and benefits of SSDTC.}

\subsection{System Model}
There is a number of processing elements (PE) that could have different types of architecture:
\[
  PE = \{ PE_i \}
\]

\note{If we can to investigate the dependency between the reliability and the thermal cycling, then we should choose the most exposed configuration to this thermal cycling. Therefore, DVFS seems to be a good example, because it is what exactly it does --- causes voltage/frequency and, consequently, temperature variations. Or we do not need such complexity and can leave only ACTIVE/IDLE modes without any DVFS policy?}

Each processing element $PE_i$ is DVFS-aware, therefore, it has a number of active states $S_i$ characterized by corresponding supply voltage $v_{ij}$ and frequency $f_{ij}$:
\begin{align*}
  & PE_i \rightarrow S_i = \{ s_{ij} \} \\
  & s_{ij} \rightarrow (v_{ij}, f_{ij})
\end{align*}

Also there is a number of applications and data dependencies between them represented by a task graph:
\begin{align*}
  & G = (\Pi, \Gamma) \\
  & \Pi = \{\tau_i\} \\
  & \Gamma = \{\gamma_i\}
\end{align*}

Each pair of task $\tau_j$ and processing element $PE_i$ is described by effective switched capacitance $C_{eff \; ij}$ and the worse case number of clock cycles $WNC_{ij}$:
\[
  (PE_i, \tau_j) \rightarrow (C_{eff \; ij}, WNC_{ij})
\]

At the same time, each task has a deadline $dl_j$.

\subsection{Mapping and Scheduling}
\todo{Describe a bit the algorithms that we use, all this genetic stuff, the list scheduling, etc. \cite{schmitz2004}.}

\subsection{Temperature-Aware Energy Optimization}
The energy dissipation depends on a particular processing element, its active state, and a particular task:
\[
  E = P t = P \frac{NC}{f}
\]
where $NC$ is the number of clock cycles that the task requires.

The total power consumption of a task is usually treated as a sum of the dynamic power and the leakage power:
\begin{align*}
  & P = P_{dyn} + P_{leak} \\
  & P_{dyn} = C_{eff} V^2 f \\
  & P_{leak} = V I_{leak} = V (MT + N)
\end{align*}

In should be noted that the dynamic power consumption does not depend on temperature, it is a constant for each combination of a processing element with its active state and a task with its worse case number of clock cycles.

For the leakage part, a linear approximation, presented in \cite{liu2007}, is used. $M$ and $N$ are the leakage coefficients that are calculated at the reference temperature $T_{ref}$ using Taylor series expansion of the initial equation for the leakage current $I_{leak}$.

However, the concern about the leak of charge across the channel of the transistor seems to be decreasing with taking into the manufacturing process new types of transistors, e.g. the FinFET and the UTB.

\todo{Put some strong references here.}

\note{Therefore, we should not put too much stress on the leakage part. We should say that we can also take it into account, but it is not that necessary, it is not our main concern or contribution. Alexandru Andrei thinks so as well. Intel goes beyond 22nm with those transistors. May be we should even skip this leakage stuff?}

\todo{The energy part of the optimization problem formulation.}

\subsection{Temperature-Aware Reliability Optimization}
We address the thermal cycling (TC) component of common failure mechanisms, which also contain electromigration, time dependent dielectric breakdown, and stress migration \cite{xiang2010}.

\todo{The reliability part of the optimization problem formulation.}

\subsection{Performance Comparison} \label{sec:comparison}
\todo{Compare the performance with the dummy HotSpot approach.}
