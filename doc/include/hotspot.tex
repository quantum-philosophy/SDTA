We consider a well-known thermal model called HotSpot. HotSpot is a modeling methodology for developing compact thermal models based on the popular stacked-layer packaging scheme in modern very large-scale integration systems \cite{huang2006}. It describes thermal systems with the following widespread differential equation \cite{rao2008}:
\[
  C \cdot \frac{dT}{dt} + G \cdot T = P
\]
where $C$ is a diagonal matrix $n \times n$ of the thermal capacitance, $G$ is a symmetric matrix $n \times n$ of the thermal conductivity ($\frac{1}{R}$), $P$ is a $n \times 1$ vector of the power consumption, and $n$ is the number of thermal nodes.

For convenience, we move $GT$ to the right part of the equation and change the denotation:
\begin{equation} \label{eq:initial}
  C \cdot \frac{dT}{dt} = A \cdot T + B
\end{equation}

HotSpot comes with a tool of the same name to calculate this model. It uses the Runge-Kutta numerical method to solve \equref{eq:initial}.

If we are talking about a power profile, then we have a number of vectors $B$ for each time step. The profile can be represented with a matrix which rows are transposed vectors $B$, and the number of rows is the number of time steps:
\[
  BB =
    \left[
      \begin{array}{c}
        B_0^T \\
        \vdots \\
        B_{m-1}^T
      \end{array}
    \right] =
    \left[
      \begin{array}{ccc}
        \dindex{b}{0}{0} & \cdots & \dindex{b}{0}{cores-1} \\
        \vdots & \ddots & \vdots \\
        \dindex{b}{m-1}{0} & \cdots & \dindex{b}{m-1}{cores-1}
      \end{array}
    \right]
\]
where $m$ is the number of time steps, $b_{ij}$ id the power supplied to the $i$-th core on the $j$-th step.

One approach to find SSDTC would be to call HotSpot (the tool) with a very long power profile composed of some repeating chunk, but it would take much time to compute (see the comparison in \secref{sec:comparison}).
