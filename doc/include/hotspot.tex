We consider a well-known thermal model called HotSpot. HotSpot is a modeling methodology for developing compact thermal models based on the popular stacked-layer packaging scheme in modern very large-scale integration systems \cite{huang2006}. It describes thermal systems with the following differential equation \cite{rao2008}:
\begin{equation} \label{eq:initial}
  C \frac{dT}{dt} = AT + B
\end{equation}
where:
\begin{itemize}
  \item $C$ --- a diagonal matrix $n \times n$ that represents the capacitance,
  \item $A$ --- a symmetric matrix $n \times n$ that represents the conductivity,
  \item $B$ --- a $n \times 1$ vector that represents the power consumption,
  \item $n = 4 \; cores + 12$ --- the number of thermal nodes,
  \item $cores$ --- the number of cores that the processor has.
\end{itemize}

\note{May be we should use more conventional denotations: $-G$ instead of $A$ (over even more obvious $-\frac{1}{R}$), and $P$ instead of $B$?}

HotSpot comes with a tool of the same name to calculate this model. It uses the Runge-Kutta numerical method to solve \equref{eq:initial}.

If we are talking about a power profile, then we have a number of vectors $B$ for each time step. The profile can be represented with a matrix which rows are transposed vectors $B$, and the number of rows is the number of time steps:
\[
  BB =
    \left[
      \begin{array}{c}
        B_0^T \\
        \vdots \\
        B_{m-1}^T
      \end{array}
    \right] =
    \left[
      \begin{array}{ccc}
        \dindex{b}{0}{0} & \cdots & \dindex{b}{0}{cores-1} \\
        \vdots & \ddots & \vdots \\
        \dindex{b}{m-1}{0} & \cdots & \dindex{b}{m-1}{cores-1}
      \end{array}
    \right]
\]
where:
\begin{itemize}
  \item $m$ --- the number of time steps,
  \item $b_{ij}$ --- the power supplied to the $i$-th core on the $j$-th step.
\end{itemize}

One approach to find SSDTC would be to call HotSpot (the tool) with a very long power profile composed of some repeating chunk, but it would take much time to compute (see the comparison in \secref{sec:comparison}).
