HotSpot is a modeling methodology for developing compact thermal models based on the popular stacked-layer packaging scheme in modern very large-scale integration systems \cite{huang2006}. It describes thermal systems with the following differential equation \cite{rao2008}:
\begin{equation} \label{eq:initial}
  C \frac{dT}{dt} = AT + B
\end{equation}
where:
\begin{itemize}
  \item $C$ --- a diagonal matrix $n \times n$ that represents the capacitance,
  \item $A$ --- a symmetric matrix $n \times n$ that represents the conductivity,
  \item $B$ --- a vector $n \times 1$ that represents the power consumption,
  \item $n = 4 \; cores + 12$ --- the number of thermal nodes,
  \item $cores$ --- the number of cores that the processor has.
\end{itemize}

\note{May be we should use more conventional denotations like $C$, $R$, and $P$?}

HotSpot comes with a tool of the same name to calculate this model. It uses the Runge-Kutta numerical method to solve \equref{eq:initial}.
