Before going to the proposed solution, we first discuss what can be achieved with existing techniques without additional efforts.

\subsection{Iterative Simulation}
The transient temperature analysis can be employed to obtain a rough approximation of the SSDTP. In this case, a thermal simulator (e.g., HotSpot \cite{huang2006}) is called several times, each time starting with the same power profile and different temperature profile computed during the previous call. A common approach to perform this analysis is to solve \equref{eq:fourier-model} numerically, e.g., using the fourth-order Runge-Kutta method.

\image{hotspot-error}{80 230 80 230}{Normalized error of the transient temperature analysis through 200 successive iterations of a 1-second application relative to the SSDTP given for different dimensions of the die.}
The number of simulations required to reach the SSDTP is characterized by the thermal time constant of the system $\tau$ defined as the following:
\[
  \tau = G^{-1} \; C = R \; C
\]
where $R$ is the thermal resistance. For the die $\tau$ is found to be in order of milliseconds while for the thermal package in order of minutes \cite{rao2007}. Consequently, the number of iterations depends on a particular configuration of the die as well as on the application period. An example is given in \figref{fig:hotspot-error}, where 200 successive iterations of the transient temperature analysis are compared with the ground truth SSDTP for five different chips with the same application of 1 second.

\subsection{Steady-State Approximation}
\image{steady-state-approximation}{80 230 80 230}{Steady-state dynamic temperature profiles (SSDTP) and corresponding approximations using the steady-state temperature (SS) given for the same application proportionally scaled to fit into 100 milliseconds (the top graph) and 1 minute (the bottom graph).}
Another approach to compute the SSDTP is to perform the steady-state analysis \cite{huang2009} (not to be confused with the steady-state \emph{dynamic} temperature analysis). Instead of solving the system of differential equations given by \equref{eq:fourier-model}, we assume that during each time interval $\triangle t_i$ of constant power $P_i$ the whole system stays in its steady state. In this case the derivative \mbox{$dT/dt = 0$} and the temperature can be calculated in the following way:
\[
  T_i = G^{-1} P_i
\]
\image{steady-state-error}{80 230 80 230}{Normalized error of the steady-state approximation as a function of the processor area given for different application periods.}
The result is a stepwise temperature curve. An example of such an approximation can be observed in \figref{fig:steady-state-approximation}, where the approximation produces exactly the same curves for two different SSDTPs. The discussion about the thermal time constant from the previous subsection holds here as well. For instance, in \figref{fig:steady-state-error} we change the dimensions of the chip for different application periods and measure the normalized error relative to the SSDTP. It can be seen that the approximation does not necessary provide a good estimation and holds only when the assumption about the thermal balance on each time interval can reasonably be applied \cite{huang2009}. To conclude, the accuracy of the SS approximation is satisfactory only for a narrow range of cases.
