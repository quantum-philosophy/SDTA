\subsection{Iterative Simulation} \label{sec:hotspot-iterative-solution}
A rough approximation of the SSDTP can be obtained by running a temperature simulation over successive periods of the application until it can be assumed that the system has reached the thermal steady state. The simulator performs the transient temperature analysis where the common approach is to solve \equref{eq:fourier-model} numerically, for instance, using the fourth-order Runge-Kutta method \cite{press2007}.

The number of iterations required to reach the SSDTP depends on the thermal characteristics of the system. In order to illustrate this aspect, we have considered an application with the period of 0.5 $s$ running on five hypothetical platforms with core areas between 1 and 25 $mm^2$. The configuration of the die and thermal package can be found in the appendix (\tabref{tab:parameters}). We have run the temperature simulation with HotSpot \cite{huang2003} for 50 successive periods. The temperature profile in each period has been compared with the actual SSDTP, obtained with our analytical approach (\secref{sec:condensed-equation}), and the normalized root mean square error (NRMSE) has been calculated. The result is shown in \figref{fig:hotspot-error}. It can be observed that the number of successive periods over which the temperature simulation has to be performed, in order to achieve a satisfactory level of accuracy, is significant for the majority of configurations. For a 9 $mm^2$ die, for example, after 15 iterations, the NRMSE is still close to 20\%. This leads to large computation times, making it difficult to apply the technique inside an intensive optimization loop.

\subsection{Steady-State Approximation (SSA)} \label{sec:steady-state-approximation}
An approximation of the SSDTP has been proposed in \cite{huang2009}. Instead of solving the system of equations in \equref{eq:fourier-model}, it is assumed that during each time interval $\Delta t_i$, in which the power is constant, the system stays in its steady state. The derivative $d\v{T}/dt = 0$ and temperature can be calculated as $\v{T}_i = \m{G}^{-1} \v{P}_i$. The result is a stepwise temperature curve where each step corresponds to the steady-state temperature $\v{T}_i$ that would be reached if the constant power $\v{P}_i$ was applied for a sufficiently long time.

An example of such an approximation (SSA) along with the corresponding SSDTP for an application with 10 tasks and period of 0.1 $s$ is given in \figref{fig:steady-state-approximation}. The die area is 25 $mm^2$, the configuration of the chip is the same as in \tabref{tab:parameters}. The reduced accuracy of the SSA is due to the mismatch between the actual temperature within each interval $\Delta t_i$ and the hypothetical steady-state temperature. The inaccuracy depends on the thermal characteristics of the respective platform and on the application itself. To illustrate this, we have generated five applications with periods between 0.01 and 1~$s$ and computed approximated SSDTPs using the SSA for die areas between 1 and 25 $mm^2$. The NRMSE relative to the correct SSDTP is shown in \figref{fig:steady-state-error}. It can be seen that, e.g., for a die area of 10 $mm^2$ and a period of $100~ms$ the NRMSE with the SSA is close to 40\%.
