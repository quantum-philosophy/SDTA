Before going to the proposed solution, we first discuss what can be achieved with existing techniques without additional efforts.

\subsection{Iterative Simulation}
A rough approximation of the SSDTP can be obtained by constructing a long power profile with one repeating part and running a thermal simulator with this profile as an input. The simulator produces a temperature profile of the same length, and the last part of it is assumed to be the SSDTP. Another way is to call the simulator a number of times each time starting with the temperature computed during the previous call. In both cases the simulator performs the transient temperature analysis where the common approach is to solve \equref{eq:fourier-model} numerically, e.g., using the fourth-order Runge-Kutta method.

\image{hotspot-error}{0 0 0 0}{Error for 30 successive simulations using HotSpot relative to the ground truth SSDTP given for different combinations of the die area and maximal power. The example is done for a single core architecture running a 1-second application.}
The number of simulations required to reach the SSDTP, i.e., the time needed for the system to reach the thermal balance, depends on the thermal time constant of the system $\tau$ defined as the following:
\[
  \tau = G^{-1} \; C = R \; C
\]
where $R$ is the thermal resistance. For the die $\tau$ is found to be in order of milliseconds while for the thermal package in order of one minute \cite{rao2007}. The temperature will keep changing until all parts of the system are stabilized. Consequently, the number of iterations depends on a particular application, since a few simulations of an application with a short period relative to $\tau$ are not enough for the temperature to reach the steady state. An example is given in \figref{fig:hotspot-error}, where 30 successive iterations using the HotSpot simulator \cite{huang2006} are compared with the ground truth SSDTP for different combinations of the die area and maximal power. In can be seen that the number of simulations varies dramatically depending on a particular setup and produces results with a residual error.

\subsection{Steady-State Approximation}
\image{steady-state-approximation}{80 230 80 230}{An example of the steady-state approximation (the dashed orange line) of the SSDTP (the solid blue line) for one processing element running a 1-second application.}
Another approach is an approximation using the steady-state temperature mentioned in \cite{huang2009}. Instead of solving the system of differential equations given by \equref{eq:fourier-model}, we assume that during each time interval $\triangle t_i$ of constant power $P_i$ the whole system stays in its steady state. In this case the derivative \mbox{$dT/dt = 0$} and the temperature can be computed in the following way:
\[
  T_i = G^{-1} P_i
\]
The result is a stepwise temperature curve. An example of such an approximation along with the real SSDTP can be observed in \figref{fig:steady-state-approximation}. It can be seen that the approximation does not necessary provide a good estimation and holds only when the assumption about the thermal balance on each time interval can reasonably be applied. For instance, it is the case when the execution time of the tasks is much larger than the time needed to reach the steady-state temperature \cite{huang2009}. The discussion about the thermal time constant from the previous subsection holds here as well with a conclusion that the accuracy of the SS approximation is satisfactory only for a narrow range of configurations.
