Before going to the proposed solution, we first discuss the available state of the art techniques.

\subsection{Iterative Simulation}
A rough approximation of the SSDTP can be obtained by running a temperature simulator over successive periods of the application until it can be assumed that the system has reached the thermal steady state. The simulator performs the transient temperature analysis where the common approach is to solve \equref{eq:fourier-model} numerically, e.g., using the fourth-order Runge-Kutta method.

\image{hotspot-error}{80 230 80 230}{Normalized error of the transient temperature analysis through 200 successive iterations of a 1-second application relative to the SSDTP given for different dimensions of the die.}
The number of simulations required to reach the SSDTP depends on the thermal characteristics of the system. In order to illustrate this aspect, we have considered an application with the period of 1 second running on five hypothetical platforms with core areas between 1 and 25 $mm^2$. We have run the temperature simulation with HotSpot \cite{huang2006} for 200 successive periods of the application. The temperature profile in each period has been compared with the SSDTP and the Normalized Root Mean Square Error (NRMSE) has been calculated. The result is shown in \figref{fig:hotspot-error}. In can be observed that the number of simulations to achieve, for instance, the accuracy of 5\% is small only for a narrow range of configurations making it difficult to apply this technique inside an intensive optimization loop.

\subsection{Steady-State Approximation}
\image{steady-state-approximation}{80 230 80 230}{Steady-state dynamic temperature profiles (SSDTP) and corresponding approximations using the steady-state temperature (SS) given for the same application proportionally scaled to fit into 100 milliseconds (the top graph) and 1 minute (the bottom graph).}
Another approach to compute the SSDTP is to perform the steady-state analysis \cite{huang2009} (not to be confused with the steady-state \emph{dynamic} temperature analysis). Instead of solving the system of differential equations given by \equref{eq:fourier-model}, we assume that during each time interval $\Delta t_i$ of constant power $P_i$ the whole system stays in its steady state. In this case the derivative \mbox{$dT/dt = 0$} and the temperature can be calculated in the following way:
\[
  T_i = G^{-1} P_i
\]
\image{steady-state-error}{80 230 80 230}{Normalized error of the steady-state approximation as a function of the processor area given for different application periods.}
The result is a stepwise temperature curve. An example of such an approximation can be observed in \figref{fig:steady-state-approximation}, where the approximation produces exactly the same curves for two different SSDTPs. The discussion about the thermal time constant from the previous subsection holds here as well. For instance, in \figref{fig:steady-state-error} we change the dimensions of the chip for different application periods and measure the normalized error relative to the SSDTP. It can be seen that the approximation does not necessary provide a good estimation and holds only when the assumption about the thermal balance on each time interval can reasonably be applied \cite{huang2009}. To conclude, the accuracy of the SS approximation is satisfactory only for a narrow range of cases.
