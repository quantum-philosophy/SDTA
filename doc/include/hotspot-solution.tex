Before going to the proposed solution, we first discuss the available state of the art techniques.

\subsection{Iterative Simulation} \label{sec:hotspot-iterative-solution}
A rough approximation of the SSDTP can be obtained by running a temperature simulator over successive periods of the application until it can be assumed that the system has reached the thermal steady state. The simulator performs the transient temperature analysis where the common approach is to solve \equref{eq:fourier-model} numerically, e.g., using the fourth-order Runge-Kutta method \cite{press2007}.

\image{hotspot-error}{40 230 40 230}{Normalized RMSE of the TTA over 50 successive periods of a 0.5-second application given for different dimensions of the die.}
\begin{itable}{parameters}{|l|r|}
  {Parameters of the Die and Thermal Package}
  {The thermal parameters (i.e., thermal conductivity, specific heat, convection resistance and capacitance) used in our experiments are equal to the default ones found in HotSpot 5.0 \cite{huang2006}.}
  \hline
  Parameter & Value \\
  \hline
  \hline
  Die thickness                         & 0.15 $mm$ \\
  Thermal interface material thickness  & 0.02 $mm$ \\
  Heat spreader side                    &   20 $mm$ \\
  Heat spreader thickness               &    1 $mm$ \\
  Heat sink side                        &   30 $mm$ \\
  Heat sink thickness                   &   15 $mm$ \\
  Ambient temperature                   &   27 ${}^\circ C$ \\
  \hline
\end{itable}
The number of simulations required to reach the SSDTP depends on the thermal characteristics of the system. In order to illustrate this aspect, we have considered an application with the period of 0.5 seconds running on five hypothetical platforms with core areas between 1 and 25 $mm^2$. The configuration of the die itself and thermal package is given in \tabref{tab:parameters}. We have run the temperature simulation with HotSpot \cite{huang2006} for 50 successive periods of the application. The temperature profile in each period has been compared with the real SSDTP obtained for the same thermal circuit constructed by HotSpot and the normalized Root Mean Square Error (RMSE) has been calculated. The result is shown in \figref{fig:hotspot-error}. It can be observed that the number of successive periods over which the temperature simulation has to be performed in order to achieve a satisfactory level of accuracy is significant in the majority of configurations. The same observation can be made with respect to the period of the application where shorter periods require larger number of iterations. This leads to large computation times, making it difficult to apply the technique inside an intensive optimization loop.

\subsection{Steady-State Approximation} \label{sec:steady-state-approximation}
\image{steady-state-approximation}{40 230 40 230}{A steady-state dynamic temperature profile (SSDTP) and its approximation using the steady-state temperature (SS) of an application of 0.1 seconds.}
Another approach is an approximation of the SSDTP proposed in \cite{huang2009}. Instead of solving the system of differential equations given by \equref{eq:fourier-model}, it is assumed that during each time interval $\Delta t_i$ in which the power is constant the whole system stays in its steady state. In this case the derivative \mbox{$d\v{T}/dt = 0$} and the temperature can be calculated as $\v{T}_i = \m{G}^{-1} \v{P}_i$.
\image{steady-state-error}{40 230 40 230}{Normalized RMSE of the steady-state approximation as a function of the die area given for different application periods.}
The result is a stepwise temperature curve where each step corresponds to the steady-state temperature $\v{T}_i$ that would be reached if the constant power $\v{P}_i$ was applied for a sufficiently long time. An example of such an approximation along with the corresponding SSDTP for an application with 10 tasks and period of 0.1 seconds is given in \figref{fig:steady-state-approximation}. The die area is 25 $mm^2$, the configuration of the chip is given in \tabref{tab:parameters}. The reduced accuracy of the approximation is due to the mismatch between the actual temperature within each interval $\Delta t_i$ and the hypothetical steady-state temperature. The inaccuracy depends on the thermal characteristics of the respective platform and on the application itself. To illustrate this, we have generated five applications with periods between 0.01 and 1 seconds and computed approximated SSDTPs for die areas between 1 and 25 $mm^2$. The normalized RMSE relative to the correct SSDTP is shown in \figref{fig:steady-state-error}. It can be seen that the approximation can reasonably be applied only for very specific cases where the assumption concerting the steady-state temperature during each time interval is acceptable \cite{huang2009}.
