The first straight-forward way to solve the system is to use dense solvers such as the LU decomposition \cite{press2007}. However, a more advanced approach is to employ sparse solvers since the matrix of the system given in \equref{eq:system} is a sparse matrix. Therefore, algorithms specially designed for such cases are preferable, e.g., the Unsymmetric MultiFrontal method \cite{umfpack2004}. The computational complexity of the solution is proportionally to $N_s^3 N_n^3$ \cite{press2007} where $N_n$ is the number of nodes and $N_s$ is the number of steps in the power profile. The problem here is that the systems can be extremely large and not feasible to solve, and, if solved, the significance of solutions can be totally lost \cite{press2007}. This is especially the case when a high accuracy is required and a fine-graned power profile is obtained with an increased number of steps and/or a large number of thermal nodes, corresponding to a high level of details in the equivalent RC thermal circuit. Our experiments have shown that direct solvers are extremely slow and consume a large amount of memory. Therefore, we will not further consider them in the paper.
