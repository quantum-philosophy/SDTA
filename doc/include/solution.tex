In order to deal with the problem, we use the analytical approach. First of all, we need to be able to solve \equref{eq:initial} efficiently. The direct solution is:
\begin{equation} \label{eq:solution}
  T(t) = e^{C^{-1}At}T_{0} + (C^{-1}A)^{-1}(e^{C^{-1}At} - I)C^{-1}B
\end{equation}

Here we need to find the matrix exponential of $C^{-1} At$, but it would be easier to do if the matrix were symmetric, because a real symmetric matrix is diagonalizable and has independent (orthogonal) real eigenvectors:
\begin{align}
  & A = U \Lambda U^T \label{eq:eigenvalue-decomposition} \\
  & e^A = e^{U \Lambda U^T} = U e^{\Lambda} U^T \nonumber \\
  & e^{\Lambda} = \left[
      \begin{array}{ccc}
        e^{\lambda_0} & \cdots & 0 \\
        \vdots & \ddots & \vdots \\
        0 & \cdots & e^{\lambda_{n - 1}}
      \end{array}
    \right] \nonumber
\end{align}
where:
\begin{itemize}
  \item $A$ --- a real symmetric matrix,
  \item $U$ --- the eigenvectors of $A$,
  \item $\Lambda$ --- a diagonal matrix of the eigenvalues of $A$ ($\lambda_i$).
\end{itemize}

Therefore, we want to keep symmetry of the matrix which matrix exponential we are going to compute. In order to achieve this, we do the following substitution:
\begin{align*}
  Y & = C^{\frac{1}{2}} T \\
  D & = C^{-\frac{1}{2}} A C^{-\frac{1}{2}} \\
  E & = C^{-\frac{1}{2}} B
\end{align*}
with the result:
\begin{align}
  \frac{dY}{dt} & = DY + E \nonumber \\
  Y(t) & = e^{Dt}Y_{0} + D^{-1}(e^{Dt} - I)E \label{eq:modified_solution} \\
  T(t) & = C^{-\frac{1}{2}} Y(t) \nonumber
\end{align}

Now $D$ is a symmetric matrix, hence, it will be easy to find the matrix exponential of $D t$ using the eigenvalue decomposition (\equref{eq:eigenvalue-decomposition}):
\[
  e^{D t} = U e^{t \Lambda} U^T = \left[
      \begin{array}{ccc}
        e^{t \lambda_0} & \cdots & 0 \\
        \vdots & \ddots & \vdots \\
        0 & \cdots & e^{t \lambda_{n - 1}}
      \end{array}
    \right]
\]

Each line of the power profile $B$ contains power values for all cores, denoted as $B_i$, at the same period of time $t_i$. Each step of the iterative process we get a vector of temperature values for all cores according to \equref{eq:modified_solution}:
\begin{equation} \label{eq:recurrent}
  Y_{i+1} = K_i Y_i + G_i B_i
\end{equation}
where:
\begin{itemize}
  \item $K_i = e^{Dt_i}$,
  \item $G_i = D^{-1}(e^{Dt_i} - I) C^{-\frac{1}{2}}$.
\end{itemize}

Since we perform the eigenvalue decomposition of D (\equref{eq:eigenvalue-decomposition}), $D^{-1}$ can be efficiently computed in the following way:
\[
  D^{-1} = U \Lambda^{-1} U^T = U \left[
      \begin{array}{ccc}
        \frac{1}{\lambda_0} & \cdots & 0 \\
        \vdots & \ddots & \vdots \\
        0 & \cdots & \frac{1}{\lambda_{n - 1}}
      \end{array}
    \right] U^T \\
\]
therefore:
\begin{align*}
  G_i & = U \Lambda^{-1} U^T (U e^{t_i \Lambda} U^T - U U^T) C^{-\frac{1}{2}} = \\
      & = U \left[
        \begin{array}{ccc}
          \frac{e^{t_i \lambda_0} - 1}{\lambda_0} & \cdots & 0 \\
          \vdots & \ddots & \vdots \\
          0 & \cdots & \frac{e^{t_i \lambda_{n - 1}} - 1}{\lambda_{n - 1}}
        \end{array}
      \right] U^T C^{-\frac{1}{2}}
\end{align*}

If the time intervals are equal, i.e. the distance in time between two vectors of power values stays the same, the iterative process can be described as the following:
\[
  Y_{i+1} = KY_i + GB_i
\]
where:
\begin{itemize}
  \item $t_i = t_s, \forall i$ --- the sampling interval,
  \item $K = e^{Dt_s}$,
  \item $G = D^{-1}(e^{Dt_s}-I) C^{-\frac{1}{2}}$.
\end{itemize}

It should be noticed that $K$ and $G$ are constants, since they depend only on the matrices $A$, $C$, and the sampling interval $t_s$.  Both $Y_i$ and $B_i$ are vectors $n \times 1$.

Therefore, in order to find SSDTC, we need to solve the following system of linear equations:
\[
  \begin{cases}
    K_0 Y_0 - Y_1 & = -Q_0 \\
    ... \\
    K_{m-1} Y_{m-1} - Y_{m} & = -Q_{m-1}
  \end{cases}
\]
where $Q_i = G_i B_i$.

Also we have one boundary condition that ensures the temperature to be the same on both sides of the curve:
\begin{equation} \label{eq:boundary-condition}
  Y_0 = Y_m
\end{equation}

Taking it into account, we get:
\[
  \begin{cases}
    K_0 Y_0 - Y_1 & =-Q_0 \\
    ... \\
    -Y_0 + K_{m-1} Y_{m-1} & = -Q_{m-1}
  \end{cases}
\]

In this notation, the system can be written as:
\begin{equation} \label{eq:system}
  AA \; YY = BB
\end{equation}
where:
\begin{itemize}
  \item
    $AA = \left[
      \begin{array}{ccccc}
        K_0 & -I & 0 & \cdots & 0 \\
        0 & K_1 & -I &  & \vdots \\
        \vdots &  & \ddots & -I & 0 \\
        0 &  &  & K_{m-2} & -I \\
        -I & 0 & \cdots & 0 & K_{m-1}
      \end{array}
    \right]$,

  \item
    $YY = \left[
      \begin{array}{c}
        Y_0 \\
        Y_1 \\
        \vdots \\
        Y_{m-2} \\
        Y_{m-1}
      \end{array}
    \right]$,

  \item $BB = \left[
    \begin{array}{c}
      -Q_0 \\
      -Q_1 \\
      \vdots \\
      -Q_{m-2} \\
      -Q_{m-1}
    \end{array}
  \right]$.
\end{itemize}

$AA$ is a square matrix $nm \times nm$. $YY$ and $BB$ are vectors $nm \times 1$. This is the system that can give us desired SSDTC. It should be observed that if the sampling interval is constant, the block diagonal of the matrix $AA$ is composed of the same blocks $K$.

Such systems could be extremely big, therefore, we need to find a fast and accurate way to solve them.
