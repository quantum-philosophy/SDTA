\image{sparseness-of-system}{50 200 50 200}{The sparseness of the system that we need to solve in order to obtain the SSDTP. Each blue point corresponds to a non-zero element of the matrix. All non-zero elements are located on the block diagonal of the matrix, one subdiagonal, and one subdiagonal in the left bottom corner.}
One may notice that the matrix $\mathbb{A}$ is an extremely sparse matrix with a very specific structure that can be observed in \figref{fig:sparseness-of-system}. The matrix has non-zero elements only on its block diagonal (composed of matrices $N_n \times N_n$), one subdiagonal just above the block diagonal, and one subdiagonal in the left bottom corner. Therefore, instead of the dense LU decomposition we can apply algorithms that are specially designed for such cases.
In our experiments we use the UMFPACK library, a set of routines for solving unsymmetric sparse linear systems based on the Unsymmetric MultiFrontal method (UMF) \cite{umfpack2004}.
