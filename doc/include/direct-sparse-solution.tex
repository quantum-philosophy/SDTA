\image{sparseness-of-system}{80 210 80 210}{Sparseness of the system of linear equations for the SSDTP calculation. Each blue point corresponds to a non-zero element of the matrix of the system.}
One may notice that the matrix of the system given in \equref{eq:system} is an extremely sparse matrix with a very specific structure that can be observed in \figref{fig:sparseness-of-system}. The matrix has non-zero elements only on its block diagonal (composed of $N_n \times N_n$ matrices), one subdiagonal just above the block diagonal, and one subdiagonal in the left bottom corner. Therefore, instead of the dense LU decomposition we can apply algorithms that are specially designed for such cases, e.g., the Unsymmetric MultiFrontal method \cite{umfpack2004}. Our experiments shown that direct sparse solvers also become extremely slow when systems are large. Moveover, all direct solvers consume a lot of memory, since they operate on full matrices.
