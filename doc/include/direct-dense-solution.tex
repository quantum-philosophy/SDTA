The first straight-forward way to solve the system is to use dense solvers such as the LU decomposition. The problem here is that such systems could be extremely large, especially when we want to achieve a higher level of accuracy and, therefore, the power profile contains a lot of steps $N_s$. Each new step produces $N_n$ new equations in the system given by \equref{eq:system}. The complexity grows very rapidly with the number of processing elements $N_p$. For instanse, the simpliest model implemented in HotSpot uses the following relation:
\[
  N_n = 4 \times N_p + 12
\]
Therefore, each new processing element increases each matrix $K_i$ by 4 rows and 4 columns, and each vector $Y_i$ and $Q_i$ by 4 elements. As an example, if the power profile for a single-processor system is composed of 1000 steps, then having the same discretization but with one additional core results in a linear system with 4000 additional equations. All in all, a fast and accurate approach to solve \equref{eq:system} is required.
