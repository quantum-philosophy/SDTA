The proposed calculation of the SSDTA can be used in a wide range of optimization procedures. One of them is reliability optimization that we discuss in this section. We perform a temperature-aware task mapping and scheduling in order to address the thermal cycling fatigue. Both mapping and scheduling are based on a genetic algorithm \cite{schmitz2004}.

\isubsection{Application Model} \label{sec:application-model}
The periodic application is modeled as a task graph $G = (V, \: E, \: \period)$ where $V$ is a set of $N_t$ tasks (vertices of the graph), $E$ is a set of data dependencies between tasks (edges), and $\period$ is the period of the application, which we assume to be equal to the deadline. Each pair of a task $v_i \in V$ and processing element $\pi_j \in \Pi$ is characterized by a tuple $(N_{clock \: ij}, C_{eff \; ij})$, where $N_{clock \: ij}$ is the number of clock cycles and $C_{eff \; ij}$ is the effective switched capacitance.

\isubsection{Temperature-Aware Reliability Model} \label{sec:reliability-model}
We address temperature-driven failure mechanisms with the reliability model presented in \cite{huang2009, xiang2010}. In this paper, our particular focus is on the thermal cycling (TC) fatigue, which is directly connected to the temperature variations. The derivation of the model is given in the appendix (\appref{reliability-optimization}).

Assuming the TC fatigue, the parameters affecting reliability are the amplitude and number of thermal cycles as well as the maximal temperature. A thermal cycle is a time interval in which the temperature starts from a certain value and, after reaching an extremum, returns back.

The mean time to failure (MTTF) of one processing element in the system can be estimated as the following:
\begin{align} \label{eq:one-mttf}
  \theta = \frac{\period}{\sum_{i=0}^{N_m - 1} \frac{1}{N_{c \: i}}}
\end{align}
where $N_m$ is the number of thermal cycles during the application period $\period$. $N_{c \: i}$ characterizes $i$th thermal cycle and is calculated according to the following expression:
\begin{equation} \label{eq:cycles-to-failure}
  N_c = A (\Delta T - \Delta T_0)^{-b} e^{\frac{E_a}{k T_{max}}}
\end{equation}
where $\Delta T$ is the thermal cycle excursion (the distance between the minimal and maximal temperatures) and $T_{max}$ is the maximal temperature during the thermal cycle.

It can be seen that the computation requires the identification of the thermal cycles with their amplitudes and maximal temperatures. All the prerequisites are captured by the SSDTP, which is needed as an input to the reliability optimization.


\isubsection{Problem Formulation and Optimization} \label{sec:reliability-problem}
The objective of the optimization procedure is to prolong the lifetime of the system by varying the mapping and scheduling of the application being executed. The problem formulation is the following.

Given:

\begin{itemize}
  \item A multiprocessor system $\Pi$ (\secref{sec:architecture-model}).
  \item A periodic application $G$ (\secref{sec:application-model}).
  \item The floorplan of the chip at the desired level of details (\secref{sec:thermal-model}), configuration of the thermal package, and thermal parameters (\secref{sec:problem}).
  \item The parameters of the reliability model (\secref{sec:reliability-model}), i.e., the constants $A$, $\Delta T_0$, $b$, $E_a$ (see \equref{eq:cycles-to-failure}).
\end{itemize}

Maximize:
\begin{align}
  & F = \min_{i = 0}^{N_p - 1} \theta_i \label{eq:fitness-function} \\
  & s.t. \nonumber \\
  & Duration(G) \leq \period \label{eq:deadline} \\
  & T_{ij} \leq T_{max}, \; i = 0 \cdots N_s, j = 0 \dots N_p \label{eq:t-max}
\end{align}
where $\theta_i$ is the MTTF of the $i$th processing element given by \equref{eq:one-mttf}, $Duration(G)$ denotes the execution time of the application mapped and scheduled onto the platform, $\period$ is the period of the application, and $T_{ij}$ are temperature values in the SSDTP. \equref{eq:deadline} imposes the first constrain on the optimization where the application is to meet its deadline, which we assume to be equal to the period. \equref{eq:t-max} enforses the second constrain on the maximal temperature in the temperature profile $\mathbb{T} = \{ T_{ij} \}$.

The optimization procedure is held by a genetic algorithm (GA) \cite{schmitz2004} with the fitness function given by \equref{eq:fitness-function}. Each chromosome is a vector of $2 \times N_t$ elements, where the first half encodes priorities of the tasks and the second represents a mapping. The population contains $4 \times N_t$ individuals that are initialized partially randomly and partially based on the mobility of the tasks \cite{schmitz2004}. Each generation, a number of individuals, called parents, are chosen for breeding by the tournament selection with the number of competitors proportional to the population size. The parents undergo the 2-point crossover with $0.8$ probability and uniform mutation with $0.05$ probability. The evolution mechanism follows the elitism model where the best individual always survives. The stopping condition is an absence of improvement within $200$ successive generations.

The fitness of a chromosome, \equref{eq:fitness-function}, is evaluated in a number of steps. First, the decoded priorities and mapping are given to a list scheduler that produces schedules for each of the cores. If the schedules do not respect the deadline of the application, the solution is penalized proportionally to the delay and is not further evaluated; otherwise, based on the parameters of the architecture and tasks, a power profile is obtained and the corresponding SSDTP is computed by the CE method. If the SSDTP violates the temparature constrain given by \equref{eq:t-max}, the solution is penalized proportionaly to the amount of violation and not further processed; otherwise, the MTTF of each core is estimated according to \equref{eq:one-mttf} and the fitness function $F$ is computed.

