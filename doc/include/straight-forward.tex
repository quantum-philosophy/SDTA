The first straight-forward way to solve the system in \equref{eq:system} is to use dense solvers such as the LU decomposition \cite{press2007}. However, a more advanced approach is to employ sparse solvers since the matrix of the system is a sparse matrix. Therefore, algorithms specially designed for such cases are preferable, e.g., the unsymmetric multifrontal method \cite{umfpack2004}. The computational complexity of the solution is proportional to $N_s^3 N_n^3$ \cite{press2007} where $N_n$ is the number of nodes and $N_s$ is the number of steps in the power profile. The problem here is that the systems to solve can be extremely large, in particular due to $N_s$. Our experiments have shown that direct solvers are extremely slow and consume a large amount of memory. Therefore, we do not consider them in the paper.

The overall matrix of the system in \equref{eq:system} is, in fact, a block Toeplitz matrix. To be more specific, the matrix is a block-circulant matrix where each block row vector is rotated one block element to the right relative to the preceding block row vector. This leads to a wide range of possible techniques to solve the system, e.g., the fast Fourier transform (FFT) \cite{mazancourt1983} that we include in our experiments in \secref{sec:results-ssdtp}.

Another possible technique is iterative methods for solving systems of linear equations (e.g., Jacobi, Gauss--Seidel, Successive Overrelaxation) \cite{press2007}. These methods are designed to overcome problems of direct solvers and, consequently, they are applicable for extremely large systems. However, the most important issue with these methods is their convergence. In our experiments we did not observe any advantages of using these methods compared to the other approaches considered in this paper. Therefore, they are excluded from the discussion.
