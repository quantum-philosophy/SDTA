\subsection{SSDTP Computation Performance} \label{sec:results-ssdtp}
\image{performance}{50 200 50 200}{A performance comparison on the semilogarithmic scale of three methods for the SSDTP calculation: the HotSpot simulator, the Unsymmetric MultiFrontal method, and the Condensed Equation method. HotSpot is restricted to one iteration. The performance is given for different application periods with a constant sampling interval of $1 ms$. One second on the horizontal axis corresponds to 1000 steps in the power profile.}
We have discussed several ways to perform the steady-state dynamic temperature profile calculation:
\begin{itemize}
  \item The iterative HotSpot simulation.
  \item The steady-state approximation using HotSpot.
  \item Direct dense solutions, namely, the dense LU decomposition (the Gaussian elimination).
  \item Direct sparse solutions with such techniques as the Unsymmetric MultiFrontal method.
  \item The solution based on the Condensed Equation method.
  \item Solutions for Toeplitz matrices, and in particular for block-circulant matrices, e.g., the Fast Fourier Transform.
  \item Iterative solution of the linear system (e.g. the Jacobi, Gauss–Seidel methods, Successive over-relaxation methods, etc.).
\end{itemize}

In should be noted that the first two solutions with the HotSpot simulator have nothing to do with the system of linear equations given by \equref{eq:system}, which is not the case with the other approaches. The iterative simulation, steady-state approximation, and iterative methods for linear systems are not analytical and produce approximated solutions (see \secref{sec:hotspot-solution}). The LU decomposition is not a competitor to the rest of the analytical approaches, since it does not take into account any special properties of the system. Moreover, direct solvers (in this case, the LU decomposition and the UMF method) are known to consume a lot of memory and, therefore, have problems with considerably large systems of linear equations. The CE method and the FFT seem to be the most promising solutions, although, according to our experiments, the former is significantly faster. The explanation to this is the following. The approach based on the fact that the matrix of the system is a block-circulant matrix is aware of the recurrent nature of the system, but it does not consider its sparseness, while the CE has both of these features.

Now we shall compare three of the methods listed above: the HotSpot simulator, the Unsymmetric MultiFrontal method, and the Condensed Equation method. We restrict HotSpot to \emph{one iteration} in order to have a cleaner comparison, since this number varies dramatically between different application periods and significantly depends on the stopping condition (we calculate the error relative to the CE method as to the ground truth). The comparison is given on \figref{fig:performance} where we vary the period of the simulated application keeping the sampling interval constant and equal to $1 ms$. For instance, an application with an 8-second period, 8000 power steps, and $(4 \times 1 + 12) \times 8000 = 128000$ linear equations requires 0.725, 2.509, and 0.0085 seconds to compute\footnote{All the experiments are done on a Linux machine with Intel Core i7-2600 (3.4GHz, 4 cores, 8 threads) and 8Gb of RAM.} for one iteration of HotSpot, the UMF method, and the CE method, correspondingly. Therefore, the CE method is approximately \emph{85 times faster} than one iteration of the HotSpot simulator, and around \emph{300 times faster} than the UMF method. The application period proportionally corresponds to the number of steps in the power profile (one second equals to 1000 steps in the power profile). Hence, we would see the same curves, if we were investigating dependency on the power profile discretization. The same holds for the number of processing elements, since, at the end of the day, the only difference is in the number of linear equations that the system (\equref{eq:system}) contains.

\subsection{Reliability Optimization}
TO BE DONE.
