\subsection{SSDTP Computation Performance} \label{sec:results-ssdtp}
In this subsection we investigate the scalability properties of the proposed solution based on the condensed equation method and compare it with one transient temperature simulation (TTS) of the application period, which is not sufficient to reach the SSDTP as it was shown in \secref{sec:hotspot-solution}. For the transient temperature analysis (TTA), the HotSpot thermal simulator is used.

\image{scaling-time}{80 230 80 230}{Scalability with the application period for a quad-core architecture. The sampling interval is fixed to 1 millisecond where 1 second on the horizontal axis corresponds to 1000 steps in the power profile. The comparison is given on the semilogarithmic scale.}
\begin{itable}{scaling-time}{|r|r|r|r|r|}
  {Scalability with the application period shown in \figref{fig:scaling-time}.}
  {$\mathcal{T}$ --- the application period, CE --- the Condensed Equation method, TTS --- one Transient Temperature Simulation, NRMSE --- the Normalized Root Mean Square Error.}
  \hline
  $\mathcal{T}$, s & CE, ms & TTS, ms & Speedup, $\times$ & NRMSE, \% \\
  \hline
  \hline
  0.05 &  0.18 &   10.24 & 56.93 & 25.8 \\
   0.1 &  0.35 &   20.26 & 58.30 & 19.0 \\
   0.5 &  1.63 &   97.36 & 59.73 & 9.65 \\
     1 &  3.23 &  193.31 & 59.80 & 7.80 \\
     2 &  6.48 &  382.59 & 59.08 & 6.46 \\
     3 &  9.58 &  573.15 & 59.83 & 5.79 \\
     4 & 12.78 &  770.09 & 60.25 & 5.34 \\
     5 & 16.10 &  964.75 & 59.92 & 5.00 \\
     6 & 19.32 & 1146.87 & 59.36 & 4.72 \\
     7 & 22.51 & 1335.26 & 59.31 & 4.49 \\
     8 & 25.69 & 1536.91 & 59.82 & 4.28 \\
     9 & 28.94 & 1729.39 & 59.76 & 4.09 \\
    10 & 32.65 & 1921.14 & 58.83 & 3.93 \\
  \hline
\end{itable}
First, we vary the application period keeping the sampling interval constant and equal to 1 millisecond. The comparison for a quad-core architecture is given in \figref{fig:scaling-time} and \tabref{tab:scaling-time}. It can be seen that the CE method is roughly 60 times faster than one iteration of the TTA\footnote{All the experiments are done on a Linux machine with Intel\textregistered\ Core\texttrademark\ i7-2600 (3.4GHz, 4 cores, 8 threads) and 8Gb of RAM.}. The application period proportionally corresponds to the number of steps in the power profile (one second is equal to 1000 steps in the power profile in the above-mentioned example). Hence, we would see the same curves, if we were investigating the dependency on the power profile discretization.

\image{scaling-cores}{80 230 80 230}{Scalability with the number of cores. The application period is fixed to 1 second that corresponds to 1000 steps in the power profile. The comparison is given on the semilogarithmic scale.}
\begin{itable}{scaling-cores}{|r|r|r|r|r|}
  {Scalability with the number of cores shown in \figref{fig:scaling-cores}.}
  {$N_p$ --- the number of processing elements (cores), CE --- the Condensed Equation method, TTS --- one Transient Temperature Simulation, NRMSE --- the Normalized Root Mean Square Error.}
  \hline
  $N_p$ & CE, ms & TTS, ms & Speedup, $\times$ & NRMSE, \% \\
  \hline
  \hline
    1 &    0.99 &    97.00 & 97.93 & 25.3 \\
   10 &   16.46 &   632.50 & 38.44 & 40.6 \\
   20 &   46.00 &  1761.75 & 38.30 & 69.5 \\
   30 &   98.49 &  3472.21 & 35.25 & 10.2 \\
   40 &  172.69 &  5827.77 & 33.75 & 130  \\
   50 &  266.08 &  8771.93 & 32.97 & 142  \\
   60 &  380.95 & 12235.91 & 32.12 & 185  \\
   70 &  517.04 & 16363.54 & 31.65 & 220  \\
   80 &  675.21 & 21104.98 & 31.26 & 245  \\
   90 &  856.20 & 26415.77 & 30.85 & 277  \\
  100 & 1058.35 & 32329.09 & 30.55 & 308  \\
  \hline
\end{itable}
The second part of the comparison is the scalability with the number of processing elements shown in \figref{fig:scaling-cores} and \tabref{tab:scaling-cores}. It can be seen that the difference between computation times of the CE method and one TTS becomes smaller when the number of cores is increasing. At the same time the mismatch between the SSDTP and temperature profile produced by one TTA dramatically increases, which means that larger number of the TTA iterations is required to reach the same level of accuracy.

\subsection{Reliability Optimization}
The experimental setup for the reliability optimization is the following. We consider 10 distinct levels of difficulty of the problem characterized by pairs $(N_p, N_t)$ where $N_p$ is the number of processing elements and $N_t$ is the number of tasks. For each level we generate 20 random task graphs of a similar structure \cite{dick1998}, apply the optimization procedure, and state the average improvement of the lifetime. The parameters of the architecture, i.e., the frequency $f$, voltage $V$, and number of gates $N_{gates}$, are also generated randomly. For counting thermal cycles the rainflow counting method described in \cite{xiang2010} is employed.

The optimization procedure is held by a genetic algorithm \cite{schmitz2004}. One half of the initial population is generated based on the mobility of the tasks, the second is generated randomly.

\begin{itable}{difficulty-levels}{|r|r|r|r|r|}
  {Reliability optimization}
  {$N_p$ --- the number of cores, $N_t$ --- the number of tasks in the periodic application, Time --- the computational time, MTTF --- the improvement of the MTTF relative to the initial solution.}
  \hline
  Level & $N_p$ & $N_t$ & Time, m & MTTF, \% \\
  \hline
   1 &  1 &  30 & 0 & 0 \\
   2 &  1 &  60 & 0 & 0 \\
   3 &  2 &  60 & 0 & 0 \\
   4 &  2 &  90 & 0 & 0 \\
   5 &  4 &  90 & 0 & 0 \\
   6 &  4 & 150 & 0 & 0 \\
   7 &  8 & 150 & 0 & 0 \\
   8 &  8 & 200 & 0 & 0 \\
   9 & 16 & 200 & 0 & 0 \\
  10 & 16 & 250 & 0 & 0 \\
  \hline
\end{itable}

\todo{Go on.}
