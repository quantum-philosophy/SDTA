\iimage{task-graph}{0 0 0 0}{Motivational example with 6 tasks, labeled from ``T0'' to ``T5'', running on a heterogeneous dual-core architecture. The execution times of the tasks vary between cores and are given on the figure along with the period of the application.}
\image{motivation}{100 240 100 240}{Three alternative combinations of mapping and scheduling (the left side) of the application shown in \figref{fig:task-graph} onto a dual-core architecture and corresponding steady-state dynamic temperature curves (the right side) for each of the cores.}
Consider a toy application with six tasks, denoted ``T0''--``T5'', and a heterogeneous architecture with two cores, labeled ``PE0'' and ``PE1''. The task graph of the application is given in \figref{fig:task-graph}. The initial mapping, schedule, and resulting SSDTP are shown at the top of \figref{fig:motivation}, where PE0 is experiencing three thermal cycles. If we change the allocation of T5 and move it to PE1, we achieve two thermal cycles of PE0 instead of three. Finally, if we vary the schedule as well and change the order of T1 and T3, the number of cycles of PE0 becomes one. Using the reliability model from \secref{sec:reliability}, we observe improvements in the MTTF of 43\% and 53\%, respectively, relative to the initial configuration.
