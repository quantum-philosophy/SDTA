\iimage{task-graph}{-30 0 -30 0}{Motivational example with 6 tasks, labeled from ``T0'' to ``T5'', running on a heterogeneous dual-core architecture.}
\image{motivation}{100 240 100 240}{Three alternative combinations of mapping and scheduling (the left side) of the application shown in \figref{fig:task-graph} onto a dual-core architecture and corresponding steady-state dynamic temperature curves (the right side) for each of the cores.}
Consider an application with six tasks, denoted ``T0''--``T5'', and a heterogeneous architecture with two cores, labeled ``PE0'' and ``PE1''. The task graph of the application is given in \figref{fig:task-graph} along with the execution times for both cores. The period of the application is 0.06 seconds. A start mapping, schedule, and the resulting SSDTP are shown at the top of \figref{fig:motivation} where the hight of a task represents its relative dynamic power consumption. In can be observed that initially PE0 is experiencing three thermal cycles\footnote{The cycle counting method is simplified here in order have a cleaner explanation. The rainflow counting method \cite{xiang2010} would produce a set of half and/or full cycles depending on the relative amplitudes of fluctuations.}. If we change the allocation of T5 and move it to PE1, we achieve two thermal cycles of PE0 instead of three. Finally, if we vary the schedule as well and change the order of T1 and T3, the number of cycles of PE0 becomes one. Using the reliability model from \secref{sec:reliability-model}, we observe improvements in the MTTF of 44.69\% and 54.53\%, respectively, relative to the initial configuration.
