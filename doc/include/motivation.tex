\iimage{task-graph}{0 0 0 0}{Motivational example with 4 tasks, labeled from ``T0'' to ``T3'', running on a heterogeneous dual-core architecture. The execution times of the tasks vary between cores and are given on the figure. The period of the application is 50 milliseconds.}
\image{motivation}{80 230 80 230}{Two alternative combinations of mapping and scheduling (the left side) of the application shown in \figref{fig:task-graph} onto a dual-core architecture and corresponding steady-state dynamic temperature curves (the right side) for each of the cores.}
Consider a toy application with four tasks, denoted ``T0''--``T3'', and a heterogeneous architecture with two cores, labeled ``PE0'' and ``PE1''. The task graph of the application is given in \figref{fig:task-graph} and two possible combinations of mapping and scheduling are depicted in \figref{fig:motivation}. It can be seen that in the first case the core ``PE0'' is experiencing two thermal cycles and only one cycle with the second configuration. Using the reliability model from \secref{sec:reliability}, we observe a $24$-percent improvement in the lifetime of the system. The difference is in the placement of the task ``T3'', which will be assigned to ``PE0'' if we are guided only by its execution time, since on ``PE0'' the task can be completed faster than on ``PE1''.
