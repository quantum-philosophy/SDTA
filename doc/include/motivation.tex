\iimage{task-graph}{0 0 0 0}{A task graph of an application with 6 tasks.}
\image{motivation}{50 200 50 200}{Two examples of mapping and scheduling (the left side) of the application shown on \figref{fig:task-graph} onto a dual-core architecture and the corresponding steady-state dynamic temperature curves (the right side) of the cores.}
\begin{itable}{motivational-setup}{|r|r|r|r|r|}
  {Setup of the task graph shown in \figref{fig:task-graph}}
  {$\tau$ --- the execution time of the task, ASAP --- as soon as possible, ALAP --- as late as possible, Mobility --- difference between the ALAP and ASAP.}
  \hline
  Task & $\tau$, s & ASAP, s & ALAP, s & Mobility \\
  \hline
  \hline
   0 &  0.04 & 0.00 & 0.08 & 0.08 \\
   1 &  0.03 & 0.04 & 0.12 & 0.08 \\
   2 &  0.07 & 0.04 & 0.17 & 0.13 \\
   3 &  0.04 & 0.07 & 0.15 & 0.08 \\
   4 &  0.04 & 0.11 & 0.08 & 0.08 \\
   5 &  0.05 & 0.11 & 0.20 & 0.09 \\
  \hline
\end{itable}
Consider a toy application with 6 tasks mapped onto a homogenious dual-core architecture. The task graph of the application is given in \figref{fig:task-graph}. Let us map and schedule the application using a variant of the list scheduler with priorities based on the mobility of the tasks, i.e., the difference between the As Soon As Possible (ASAP) and As Late As Possible (ALAP) start times \cite{schmitz2004}. The mapping is done along with the scheduling where we assing the earliest available core for the task being processed. All the parameters are shown in \tabref{tab:motivational-setup}. The resulting and alternative mapping, schedule, and SSDTP for each core are depicted in the top and bottom of \figref{fig:motivation}, respectively. It can be seen that in the first case, the core labeled ``Core1'' is experiencing two thermal cycles and only one cycle with the second configuration.
